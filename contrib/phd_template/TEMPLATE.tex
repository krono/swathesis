\documentclass[%
   %draft,     % Entwurfsstadium
   final,      % fertiges Dokument
%%%% --- Schriftgröße ---
   fontsize=12pt,
   %smallheadings,    % kleine Überschriften
   %normalheadings,   % normale Überschriften
   %bigheadings,       % große Überschriften
%%%% --- Sprache ---
%   ngerman,           % wird an andere Pakete weitergereicht
%%%% === Seitengröße ===
   % letterpaper,
   % legalpaper,
   % executivepaper,
   paper=a4,
   % a5paper,
   % landscap,
%%%% === Optionen für den Satzspiegel ===
   BCOR=5mm,          % Zusaetzlicher Rand auf der Innenseite
   %DIV11,            % Seitengroesse (siehe Koma Skript Dokumentation !)
   DIV=calc,         % automatische Berechnung einer guten Zeilenlaenge
   1.1headlines,     % Zeilenanzahl der Kopfzeilen
   %headinclude,     % Kopf einbeziehen
   %headexclude,      % Kopf nicht einbeziehen
   %footinclude,     % Fuss einbeziehen
   %footexclude,      % Fuss nicht einbeziehen
   %mpinclude,       % Margin einbeziehen
   %mpexclude,        % Margin nicht einbeziehen
   pagesize,         % Schreibt die Papiergroesse in die Datei.
                     % Wichtig für Konvertierungen
%%%% === Layout ===
   %oneside,         % einseitiges Layout
   twoside,          % Seitenraender für zweiseitiges Layout
   onecolumn,        % Einspaltig
   %twocolumn,       % Zweispaltig
   %openany,         % Kapitel beginnen auf jeder Seite
   open=right,        % Kapitel beginnen immer auf der rechten Seite
                     % (macht nur bei 'twoside' Sinn)
   %cleardoubleplain,    % leere, linke Seite mit Seitenstil 'plain'
   %cleardoubleempty,    % leere, linke Seite mit Seitenstil 'empty'
   titlepage,        % Titel als einzelne Seite ('titlepage' Umgebung)
   %notitlepage,     % Titel in Seite integriert
%%%% --- Absatzeinzug ---
   %                 % Absatzabstand: Einzeilig,
   %parskip,         % Freiraum in letzter Zeile: 1em
   %parskip*,        % Freiraum in letzter Zeile: Viertel einer Zeile
   %parskip+,        % Freiraum in letzter Zeile: Drittel einer Zeile
   %parskip-,        % Freiraum in letzter Zeile: keine Vorkehrungen
   %                 % Absatzabstand: Halbzeilig
   %halfparskip,     % Freiraum in letzter Zeile: 1em
   %halfparskip*,    % Freiraum in letzter Zeile: Viertel einer Zeile
   %halfparskip+,    % Freiraum in letzter Zeile: Drittel einer Zeile
   %halfparskip,     % Freiraum in letzter Zeile: keine Vorkehrungen
   %                 % Absatzabstand: keiner
   %parindent,        % Eingerückt (Standard)
%%%% --- Kolumnentitel ---
   headsepline,      % Linie unter Kolumnentitel
   %headnosepline,   % keine Linie unter Kolumnentitel
   %footsepline,     % Linie unter Fussnote
   %footnosepline,   % keine Linie unter Fussnote
%%%% --- Kapitel ---
   %chapterprefix,   % Ausgabe von 'Kapitel:'
   %nochapterprefix,  % keine Ausgabe von 'Kapitel:'
%%%% === Verzeichnisse (TOC, LOF, LOT, BIB) ===
   %liststotoc,      % Tabellen & Abbildungsverzeichnis ins TOC
   %idxtotoc,        % Index ins TOC
   %bibtotoc,         % Bibliographie ins TOC
   %bibtotocnumbered, % Bibliographie im TOC nummeriert
   %liststotocnumbered, % Alle Verzeichnisse im TOC nummeriert
   %tocindent,        % eingereuckte Gliederung
   %tocleft,         % Tabellenartige TOC
   %listsindent,      % eingereuckte LOT, LOF
   %listsleft,       % Tabellenartige LOT, LOF
   %pointednumbers,  % Überschriftnummerierung mit Punkt, siehe DUDEN !
   numbers=noenddot, % Überschriftnummerierung ohne Punkt, siehe DUDEN !
   %openbib,         % alternative Formatierung des Literaturverzeichnisses
%%%% === Matheformeln ===
   %leqno,           % Formelnummern links
   fleqn,            % Formeln werden linksbuendig angezeigt
]{scrbook}%     Klassen: scrartcl, scrreprt, scrbook

\makeatletter
\@ifpackagelater{scrbase}{2013/12/19}{%
  \KOMAoption{titlepage}{firstiscover}
}{
  \KOMAoption{titlepage}{yes}
}
\renewcommand*\toc@heading{%
 \chapter*{\contentsname}%
 \@mkboth{\MakeMarkcase{\contentsname}}{\MakeMarkcase{\contentsname}}%
}
\renewcommand*{\scr@float@listhead@warning}{%
}
\renewcommand*{\scr@float@addtolists@warning}{%
  \global\let\scr@float@addtolists@warning\relax
}
\makeatother
\listfiles
\usepackage{datenumber}
\usepackage[utf8x]{inputenc}
\usepackage{graphicx}


\def\svnId{$$Id:$$}
\makeindex

%% ------------------------------------------------------------------------
%%% Packages for LaTeX - programming
%
% Define commands that don't eat spaces.
\usepackage{xspace}
% IfThenElse
\usepackage{ifthen}
%%% Doc: ftp://tug.ctan.org/pub/tex-archive/macros/latex/contrib/oberdiek/ifpdf.sty
% command for testing for pdf-creation
\usepackage{ifpdf} %\ifpdf \else \fi

%%% Internal Commands: ----------------------------------------------
\makeatletter
%
\providecommand{\IfPackageLoaded}[2]{\@ifpackageloaded{#1}{#2}{}}
\providecommand{\IfPackageNotLoaded}[2]{\@ifpackageloaded{#1}{}{#2}}
\providecommand{\IfElsePackageLoaded}[3]{\@ifpackageloaded{#1}{#2}{#3}}
%
\newboolean{chapteravailable}%
\setboolean{chapteravailable}{false}%

\ifcsname chapter\endcsname
  \setboolean{chapteravailable}{true}%
\else
  \setboolean{chapteravailable}{false}%
\fi


\providecommand{\IfChapterDefined}[1]{\ifthenelse{\boolean{chapteravailable}}{#1}{}}%
\providecommand{\IfElseChapterDefined}[2]{\ifthenelse{\boolean{chapteravailable}}{#1}{#2}}%

\providecommand{\IfDefined}[2]{%
\ifcsname #1\endcsname
   #2 %
\else
     % do nothing
\fi
}

\providecommand{\IfElseDefined}[3]{%
\ifcsname #1\endcsname
   #2 %
\else
   #3 %
\fi
}

\providecommand{\IfElseUnDefined}[3]{%
\ifcsname #1\endcsname
   #3 %
\else
   #2 %
\fi
}


%
% Check for 'draft' mode - commands.
\newcommand{\IfNotDraft}[1]{\ifx\@draft\@undefined #1 \fi}
\newcommand{\IfNotDraftElse}[2]{\ifx\@draft\@undefined #1 \else #2 \fi}
\newcommand{\IfDraft}[1]{\ifx\@draft\@undefined \else #1 \fi}
%

% Definde frontmatter, mainmatter and backmatter if not defined
\@ifundefined{frontmatter}{%
   \newcommand{\frontmatter}{%
      %In Roemischen Buchstaben nummerieren (i, ii, iii)
      \pagenumbering{roman}
   }
}{}
\@ifundefined{mainmatter}{%
   % scrpage2 benoetigt den folgenden switch
   % wenn \mainmatter definiert ist.
   \newif\if@mainmatter\@mainmattertrue
   \newcommand{\mainmatter}{%
      % -- Seitennummerierung auf Arabische Zahlen zuruecksetzen (1,2,3)
      \pagenumbering{arabic}%
      \setcounter{page}{1}%
   }
}{}
\@ifundefined{backmatter}{%
   \newcommand{\backmatter}{
      %In Roemischen Buchstaben nummerieren (i, ii, iii)
      \pagenumbering{roman}
   }
}{}

% Pakete speichern die spaeter geladen werden sollen
\newcommand{\LoadPackagesNow}{}
\newcommand{\LoadPackageLater}[1]{%
   \g@addto@macro{\LoadPackagesNow}{%
      \usepackage{#1}%
   }%
}



\makeatother
%%% ----------------------------------------------------------------
%%
%%%% Doc: www.cs.brown.edu/system/software/latex/doc/calc.pdf
%% Calculation with LaTeX
%\usepackage{calc}
%
%%%% Doc: ftp://tug.ctan.org/pub/tex-archive/macros/latex/required/babel/babel.pdf
%% Languagesetting
%\usepackage[
%%	german,
%%	ngerman,
%	english,
%%	frensh,
%]{babel}

% \usepackage[%
%	final,
%	%draft % do not include images (faster)
% ]{graphicx}



\usepackage{marginnote}


%% Doc: (inside relsize.sty )
%% ftp://tug.ctan.org/pub/tex-archive/macros/latex/contrib/misc/relsize.sty
%  Set the font size relative to the current font size
%\usepackage{relsize}

%% Doc: ftp://tug.ctan.org/pub/tex-archive/macros/latex/contrib/ms/ragged2e.pdf
% Besserer Flatternsatz (Linksbuendig, statt Blocksatz)
%\usepackage{ragged2e}
%% Fonts
\usepackage{lmodern}


%% ~~~~~~~~~~~~~~~~~~~~~~~~~~~~~~~~~~~~~~~~~~~~~~~~~~~~~~~~~~~~~~~~~~~~~~~~
%% PDF related packages
%% ~~~~~~~~~~~~~~~~~~~~~~~~~~~~~~~~~~~~~~~~~~~~~~~~~~~~~~~~~~~~~~~~~~~~~~~~
%
%%% Doc: ftp://tug.ctan.org/pub/tex-archive/macros/latex/contrib/microtype/microtype.pdf
% Optischer Randausgleich mit pdfTeX
\ifpdf
\usepackage[%
	expansion=true, % better typography, but with much larger PDF file.
	protrusion=true
]{microtype}
\fi
%


%%% Doc: ftp://tug.ctan.org/pub/tex-archive/macros/latex/contrib/oberdiek/hypcap.pdf
% Links auf Gleitumgebungen springen nicht zur Beschriftung,
% sondern zum Anfang der Gleitumgebung
\IfPackageLoaded{hyperref}{%
	\usepackage[figure]{hypcap}
}



% Pakete Laden die nach Hyperref geladen werden sollen
\LoadPackagesNow % (ltxtable, tabularx)

% ~~~~~~~~~~~~~~~~~~~~~~~~~~~~~~~~~~~~~~~~~~~~~~~~~~~~~~~~~~~~~~~~~~~~~~~~
% figures and placement
% ~~~~~~~~~~~~~~~~~~~~~~~~~~~~~~~~~~~~~~~~~~~~~~~~~~~~~~~~~~~~~~~~~~~~~~~~

%% Bilder und Graphiken ==================================================

%%% Doc: only dtx Package
%\usepackage{float}             % Stellt die Option [H] f�r Floats zur Verf�gung

%%% Doc: No Documentation
%b\usepackage{flafter}          % Floats immer erst nach der Referenz setzen

% Defines a \FloatBarrier command, beyond which floats may not
% pass; useful, for example, to ensure all floats for a section
% appear before the next \section command.
%\usepackage[
%	section		% "\section" command will be redefined with "\FloatBarrier"
%]{placeins}
%
%%% Doc: ftp://tug.ctan.org/pub/tex-archive/macros/latex/contrib/subfig/subfig.pdf
% Incompatible: loads package capt-of. Loading of 'capt-of' afterwards will fail therefor
%\usepackage{subfig} % Layout wird weiter unten festgelegt !


% ~~~~~~~~~~~~~~~~~~~~~~~~~~~~~~~~~~~~~~~~~~~~~~~~~~~~~~~~~~~~~~~~~~~~~~~~
% verbatim packages
% ~~~~~~~~~~~~~~~~~~~~~~~~~~~~~~~~~~~~~~~~~~~~~~~~~~~~~~~~~~~~~~~~~~~~~~~~

%%% Doc: ftp://tug.ctan.org/pub/tex-archive/macros/latex/contrib/upquote/upquote.sty
\usepackage{upquote} % Setzt "richtige" Quotes in verbatim-Umgebung


%%% Doc: No documentation - documented in 'The LaTeX Companion'
% \usepackage{fancybox}   % for shadowbox, ovalbox

%%% Doc: ftp://tug.ctan.org/pub/tex-archive/macros/latex/contrib/misc/framed.sty
% \usepackage{framed}
% \renewcommand\FrameCommand{\fcolorbox{black}{shadecolor}}

\makeatletter
\IfPackageLoaded{framed}{%
   \IfPackageLoaded{marginnote}{%
������\begingroup
         \g@addto@macro\framed{%
���������   \let\marginnoteleftadjust\FrameSep
������������   \let\marginnoterightadjust\FrameSep
         }
      \makeatother
� }
}
\makeatother
%


\iffalse % CLASH mit dissertation.tex

% Layout mit 'geometry'
%%% Doc: ftp://tug.ctan.org/pub/tex-archive/macros/latex/contrib/geometry/manual.pdf
\usepackage{geometry}
\geometry{%
%%% Paper Groesse
   a4paper, % Andere a0paper, a1paper, a2paper, a3paper, , a5paper, a6paper,
            % b0paper, b1paper, b2paper, b3paper, b4paper, b5paper, b6paper
            % letterpaper, executivepaper, legalpaper
   %screen,  % a special paper size with (W,H) = (225mm,180mm)
   %paperwidth=,
   %paperheight=,
   %papersize=, %{ width , height }
   %landscape,  % Querformat
   portrait,    % Hochformat
%%% Koerper Groesse
   %hscale=,      % ratio of width of total body to \paperwidth
                  % hscale=0.8 is equivalent to width=0.8\paperwidth. (0.7 by default)
   %vscale=,      % ratio of height of total body to \paperheight
                  % vscale=0.9 is equivalent to height=0.9\paperheight.
   %scale=,       % ratio of total body to the paper. scale={ h-scale , v-scale }
   %totalwidth=,    % width of total body % (Generally, width >= textwidth)
   %totalheight=,   % height of total body, excluding header and footer by default
   %total=,        % total={ width , height }
   %textwidth=,    % modifies \textwidth, the width of body
   %textheight=,   % modifies \textheight, the height of body
   %body=,        % { width , height } sets both \textwidth and \textheight of the body of page.
   lines=44,       % enables users to specify \textheight by the number of lines.
   %includehead,  % includes the head of the page, \headheight and \headsep, into total body.
   %includefoot,  % includes the foot of the page, \footskip, into body.
   %includeheadfoot, % sets both includehead and includefoot to true
   %includemp,    % includes the margin notes, \marginparwidth and \marginparsep, into body
   %includeall,   % sets both includeheadfoot and includemp to true.
   %ignorehead,   % disregards the head of the page, headheight and headsep in determining vertical layout
   %ignorefoot,   % disregards the foot of page, footskip, in determining vertical layout
   %ignoreheadfoot, % sets both ignorehead and ignorefoot to true.
   %ignoremp,     % disregards the marginal notes in determining the horizontal margins
   ignoreall,     % sets both ignoreheadfoot and ignoremp to true
   heightrounded, % This option rounds \textheight to n-times (n: an integer) of \baselineskip
   %hdivide=,     % { left margin , width , right margin }
                  % Note that you should not specify all of the three parameters
   %vdivide=,     % { top margin , height , bottom margin }
   %divide=,      % ={A,B,C} %  is interpreted as hdivide={A,B,C} and vdivide={A,B,C}.
%%% Margin
   %left=,        % left margin (for oneside) or inner margin (for twoside) of total body
                  % alias: lmargin, inner
   %right=,       % right or outer margin of total body
                  % alias: rmargin outer
   %top=,         % top margin of the page.
                  % Alias : tmargin
   %bottom=,      % bottom margin of the page
                  % Alias : bmargin
   %hmargin=,     % left and right margin. hmargin={ left margin , right margin }
   %vmargin=,     % top and bottom margin. vmargin={ top margin , bottom margin }
   %margin=,      % margin={A,B} is equivalent to hmargin={A,B} and vmargin={A,B}
   %hmarginratio, % horizontal margin ratio of left (inner) to right (outer).
   %vmarginratio, % vertical margin ratio of top to bottom.
   %marginratio,  % marginratio={ horizontal ratio , vertical ratio }
   %hcentering,   % sets auto-centering horizontally and is equivalent to hmarginratio=1:1
   %vcentering,   % sets auto-centering vertically and is equivalent to vmarginratio=1:1
   centering,    % sets auto-centering and is equivalent to marginratio=1:1
   twoside,       % switches on twoside mode with left and right margins swapped on verso pages.
   %asymmetric,   % implements a twosided layout in which margins are not swapped on alternate pages
                  % and in which the marginal notes stay always on the same side.
   bindingoffset=5mm,  % removes a specified space for binding
%%% Dimensionen
   %headheight=,  % Alias:  head
   %headsep=,     % separation between header and text
   %footskip=,    % distance separation between baseline of last line of text and baseline of footer
                  % Alias: foot
   %nohead,       % eliminates spaces for the head of the page
                  % equivalent to both \headheight=0pt and \headsep=0pt.
   %nofoot,       % eliminates spaces for the foot of the page
                  % equivalent to \footskip=0pt.
   %noheadfoot,   % equivalent to nohead and nofoot.
   %footnotesep=, % changes the dimension \skip\footins,.
                  % separation between the bottom of text body and the top of footnote text
   marginparwidth=0pt, % width of the marginal notes
                  % Alias: marginpar
   %marginparsep=,% separation between body and marginal notes.
   %nomarginpar,  % shrinks spaces for marginal notes to 0pt
   %columnsep=,   % the separation between two columns in twocolumn mode.
   %hoffset=,
   %voffset=,
   %offset=,      % horizontal and vertical offset.
                  % offset={ hoffset , voffset }
   %twocolumn,    % twocolumn=false denotes onecolumn
   %twoside,
   textwidth=400pt,   % sets \textwidth directly
   %textheight=600pt,  % sets \textheight directly
   %reversemp,    % makes the marginal notes appear in the left (inner) margin
                  % Alias: reversemarginpar
} % Endif

\fi


\raggedbottom     % Variable Seitenhoehen zulassen



%% Schriften (Sections )==================================================

\IfElsePackageLoaded{fourier}{
   \newcommand\SectionFontStyle{\rmfamily}
}{
   \newcommand\SectionFontStyle{\sffamily}
}

% -- Koma Schriften --
\IfChapterDefined{%
   \setkomafont{chapter}{\huge\SectionFontStyle}    % Chapter
}
\setkomafont{descriptionlabel}{\SectionFontStyle}
\setkomafont{sectioning}{\SectionFontStyle} %  % Titelzeilen % \bfseries
\setkomafont{pagenumber}{\bfseries\SectionFontStyle}             % Seitenzahl
\setkomafont{pageheadfoot}{\small\sffamily}        % Kopfzeile \setkomafont{pagehead}{\small\sffamily}
%\setkomafont{pagefoot}{\small\sffamily}        % Kopfzeile
%\setkomafont{descriptionlabel}{\itshape}        % Kopfzeile
%

\renewcommand*{\raggedsection}{\raggedright} % Titelzeile linksbuendig, haengend
%
%% UeberSchriften (Chapter und Sections) =================================
% -- Ueberschriften komlett Umdefinieren --
%%% Doc: ftp://tug.ctan.org/pub/tex-archive/macros/latex/contrib/titlesec/titlesec.pdf

%\usepackage{titlesec} 

%\titleformat{\chapter}[block]
%{\vspace{-5pc}\usekomafont{chapter}\Large \color{black}}
%{\Huge \textsubscript{\thechapter} \filright }
%{1px}
%{\titlerule\vspace{.5pc}\\ \LARGE  }   % {before}[after] (before chaptertitle and after)
%  [\color{black} \vspace{0.6pc} \filright {\titlerule}]


%% Captions (Schrift, Aussehen) ==========================================

% % Folgende Befehle werden durch das Paket caption und subfig ersetzt !
% \setcapindent{1em} % Einrueckung der Beschriftung
% \setkomafont{caption}{\color{black}\small\sffamily\RaggedRight}  % Schrift f�r Caption
% \setkomafont{captionlabel}{\color{black}\small}   % Schrift f�r 'Abbildung' usw.

%%% Doc: ftp://tug.ctan.org/pub/tex-archive/macros/latex/contrib/caption/caption.pdf
\usepackage{caption}
% Aussehen der Captions
\captionsetup{
   margin = 10pt,
   font = {small,rm},
   labelfont = {small,bf},
   format = plain, % oder 'hang'
   indention = 0em,  % Einruecken der Beschriftung
   labelsep = colon, %period, space, quad, newline
   justification = RaggedRight, % justified, centering
   singlelinecheck = true, % false (true=bei einer Zeile immer zentrieren)
   position = bottom %top
}
%%% Bugfix Workaround
\DeclareCaptionOption{parskip}[]{}
\DeclareCaptionOption{parindent}[]{}

% Aussehen der Captions f�r subfigures (subfig-Paket)
\IfPackageLoaded{subfig}{
 \captionsetup[subfloat]{%
   margin = 10pt,
   font = {small,rm},
   labelfont = {small,bf},
   format = plain, % oder 'hang'
   indention = 0em,  % Einruecken der Beschriftung
   labelsep = space, %period, space, quad, newline
   justification = RaggedRight, % justified, centering
   singlelinecheck = true, % false (true=bei einer Zeile immer zentrieren)
   position = bottom, %top
   labelformat = parens % simple, empty % Wie die Bezeichnung gesetzt wird
 }
}

\usepackage{rotating}


% Malte
\newcommand{\bv}{behavioral variations\xspace}
\newcommand{\Bv}{Behavioral variations\xspace}
\newcommand{\BV}{Behavioral Variations\xspace}

\newcommand{\forsubject}{\code{subject}\xspace}
\newcommand{\context}{\code{context}\xspace}
\newcommand{\LPAREN}{\code{\textbf{(}}}
\newcommand{\RPAREN}{\code{\textbf{)}}}
\newcommand{\LBRACE}{\code{\textbf{\{}}}
\newcommand{\RBRACE}{\code{\textbf{\}}}}
\newcommand{\LBRACKET}{\code{\textbf{[}}}
\newcommand{\RBRACKET}{\code{\textbf{]}}}
%\newcommand{\ind}{\hspace*{1.5em}}
\newcommand{\nl}{\newline}
\newcommand{\OR}{~\textbar}
\newcommand{\LBr}{ \texttt{(} }
\newcommand{\RBr}{\texttt{)}}
\newcommand{\LB}{\texttt{\{~}}
\newcommand{\RB}{\texttt{\}}}
%\newcommand{\OR}{\textbar~}
\newcommand{\SEMICOLON}{\texttt{;}}

\newcommand{\supervisor}[2] { %
 %\begin{tabular}{l l}
 \par{%
   \begin{minipage}[t]{3cm}
     \large #1 :
   \end{minipage}\begin{minipage}[t]{10cm}
     \large #2 
   \end{minipage}
 }
 }




 %TEX root = main.tex
%%%%%%%%%%%%%%%%%%%%%%%%%%%%%%%%%%%%%%%%%%%%%%%%%%%%%%%%%%%%%%%%%%%%%%%%%%%%%%%%%%%%%%%%%%%%%%%%%%%
%
% This file contains:
%
% Comment Mode Support 
% ==================
% The following features are only active while in comment mode. To enter this mode, simply 
% set \commentmode in the preamble. 
%
% \documentclass{...}
%  %TEX root = main.tex
%%%%%%%%%%%%%%%%%%%%%%%%%%%%%%%%%%%%%%%%%%%%%%%%%%%%%%%%%%%%%%%%%%%%%%%%%%%%%%%%%%%%%%%%%%%%%%%%%%%
%
% This file contains:
%
% Comment Mode Support 
% ==================
% The following features are only active while in comment mode. To enter this mode, simply 
% set \commentmode in the preamble. 
%
% \documentclass{...}
%  %TEX root = main.tex
%%%%%%%%%%%%%%%%%%%%%%%%%%%%%%%%%%%%%%%%%%%%%%%%%%%%%%%%%%%%%%%%%%%%%%%%%%%%%%%%%%%%%%%%%%%%%%%%%%%
%
% This file contains:
%
% Comment Mode Support 
% ==================
% The following features are only active while in comment mode. To enter this mode, simply 
% set \commentmode in the preamble. 
%
% \documentclass{...}
% \input{auxiliary}
% \commentmode
%
% - Draft Notice and Svn Info in a Framed Header on Every Page. This feature is automatically 
%   activated while in comment mode, but needs an additional declaration in the preamble to set 
%   the SVN Id: \def\svnId{$$Id:$$}
% 
%   \documentclass{...}
%   \def\svnId{$$Id:$$}
%   \input{auxiliary}
%   \commentmode
%
% - Comments Labeled with Author Id. Define your own author id before \begin{document}:
%   \newcommand\myAuthorID[1]{\nbnote{MAID}{#1}}
%   Use it anywhere in the text: 
%   ... \myAuthorID{this is a comment} ... 
%
% Listings Package Configuration for Several Languages
% ====================================================
% - Syntax highlighting for JCop, ContextJ, LogicAJ, LogicAJ2, GenTL, Prolog
% - Code makro to define code snippets independently of their appeariance. Use:
%   \code{public void foo(){}}
%
% Published Notice on Title Page
% ============================= 
% Paper reference feature to place a publication note on top of the title page of local downloads. 
% Use the following command after \begin{document}: 
% \paperref{<framed box width>}{<height>}{This paper appeared at the n-th XY Conference on Z}
%
% last modified: MA, 2009-08-17_14-47
%%%%%%%%%%%%%%%%%%%%%%%%%%%%%%%%%%%%%%%%%%%%%%%%%%%%%%%%%%%%%%%%%%%%%%%%%%%%%%%%%%%%%%%%%%%%%%%%%%%
\usepackage{amsmath}
\usepackage[usenames,dvipsnames]{color}

\usepackage{framed}
\usepackage{xspace}
\usepackage[hyphens]{url}
%\urlstyle{same}
\urlstyle{tt}
%\usepackage{caption}
\DeclareMathAlphabet{\mathpzc}{OT1}{pzc}{m}{it}
% General %%%%%%%%%%%%%%%%%%%%%%%%%%%%%%%%%%%%%%%%%%%%%%%%%%%%%%%%%%%%%%%%%%%%%%%%%%%%%%%%%%%%%%%%%

% API Table %%%%%%%%%%%%%%%%%%%%%%%%%%%%%%%%%%%%%%%%%%%%%%%%%%%%%%%%%%%%%%%%%%%%%%%%%%%%%%%
\usepackage{longtable}
\usepackage{multirow}

\newenvironment{api2}[1]{%
\begin{tabular}{p{.5cm} p{13.5cm}}%
\hline%
%\multicolumn{2}{|c|}{\code{#1}}\\ \hline %
}%
{\hline%
\end{tabular}%
}
\newcommand{\apidesctxt}[2]{\multicolumn{2}{l}{\lstinline[language=JCop, basicstyle=\listingsfont]{public #1}} \\ & #2 \\}

\newenvironment{api}[1]{%
\begin{longtable}{p{.5cm} p{12.5cm}}%
%\hline%
%\multicolumn{2}{|c|}{\code{#1}}\\ \hline %
}%
{%\hline%
\end{longtable}%
}

\newcommand{\apidesc}[2]{\multicolumn{2}{l}{\lstinline[language=JCop, basicstyle=\listingsfont]{public #1}} \\ & \footnotesize{#2} \\}


% COMMENTS & DRAFT  %%%%%%%%%%%%%%%%%%%%%%%%%%%%%%%%%%%%%%%%%%%%%%%%%%%%%%%%%%%%%%%%%%%%%%%%%%%%%%%

\usepackage{ednotes}
\usepackage{amssymb}
\usepackage{datetime}
\usepackage{everypage}

% date and timeformat
\newdateformat{simpledate}{\THEYEAR-\twodigit{\THEMONTH}-\twodigit{\THEDAY}}
\newtimeformat{simpletime}{\twodigit{\THEHOUR}:\twodigit{\THEMINUTE}}
\newcommand{\timestamp}{\simpledate\today~\simpletime}

% header for draft notice
\newcommand{\draftheader}{%
  %\framebox[\textwidth][t]{%
	  \centerline	
}

% the default definitions of \nbnote and \svnversion are empty	
\newcommand{\nbnote}[2]{}
\newcommand{\svnversion}{}

% the actions perfomed while in comment mode
\newcommand{\commentmode}{
		% \usepackage{draftwatermark} \SetWatermarkLightness{0.70} \SetWatermarkText{\Huge DRAFT}   	
		\renewcommand{\nbnote}[2]{%
		  \fbox{\bfseries\sffamily\scriptsize##1}
		  {\sf\small $\blacktriangleright$ \textit##2 $\blacktriangleleft$}
		  %{\sf\small$\blacktriangleright$\textit{##2}$\blacktriangleleft$}%
		}
	 \renewcommand{\svnversion}{\emph{\footnotesize\svnId}}
	 \AddEverypageHook{\paperref{21cm}{-3mm}{\draftheader}}      
	 \AddEverypageHook{\paperref{21cm}{28.5cm}{\draftheader}}      
}


\newcommand\todo[1]{\nbnote{TO DO}{#1}}

\newcommand{\here}{\nbnote{***}{CONTINUE HERE}}
\newcommand\fix[1]{\nbnote{FIX}{#1}}
\newcommand\jl[1]{\nbnote{JL}{#1}}

\newcommand\rw[1]{\color{blue}#1\color{black}}

% dangerous!
% \DeclareOption{comments}{\commentmode}
% \ProcessOptions 

% LISTINGS  %%%%%%%%%%%%%%%%%%%%%%%%%%%%%%%%%%%%%%%%%%%%%%%%%%%%%%%%%%%%%%%%%%%%%%%%%%%%%%%%%%%%%%%

\usepackage[formats]{listings}
\usepackage[T1]{fontenc}


%\usepackage{pifont}
%\usepackage{winfonts}
%\usepackage{windingbats}
%\usepackage{winpifont}
\usepackage{soul}
%\DisableLigatures{encoding = T1, family = courier }
%\usepackage{beramono}
\usepackage[scaled=.84]{DejaVuSansMono}
%\linespread{1.5}  % mathpazo

%\fontfamily{courier} \selectfont 

\newcommand{\code}[1]{\mbox{\texttt{#1}}} % code font
%\let\code\lstinline
\newcommand{\scode}[1]{\texttt{\small{#1}}} % code font
%\newcommand{\code}[1]{%
%  \lstinline[language=JCop,%
%  					 basicstyle={\ttfamily \footnotesize \selectfont},%
%             keywordstyle={\bfseries \color[rgb]{0,0,0}}]!#1!} % code font
\newcommand{\biglistingsfont}{ \ttfamily \normalsize \selectfont }
\newcommand{\listingsfont}{ \ttfamily \scriptsize \selectfont }
\newcommand{\tinylistingsfont}{\rmfamily \fontsize{6.5}{6.5} \selectfont}
%\newcommand{\listingsfont}{ \fontfamily{courier} \scalebox{.5}[1] \scriptsize \selectfont }
%\newcommand{\tinylistingsfont}{\fontfamily{courier} \fontsize{6.5}{5}  \selectfont \scalebox{.5}[1]}

%\newcommand{\listingsfont}{\ttfamily \scriptsize}

% COLORS
% eclipse keywords: \color[rgb]{0.49,0.00,0.33}
% eclipse strings:  \color[rgb]{0.16,0.2,1}
% eclipse comments: \color[rgb]{0.247,0.498,0.372},


% -- JCop  ----------------------------------------------------------------------------------------

\newcommand{\layerclass}{\code{jcop.lang.Layer}\xspace}
\newcommand{\with}{\code{with}\xspace}
\newcommand{\without}{\code{without}\xspace}
\newcommand{\withoutall}{\code{withoutall}\xspace}
\newcommand{\layer}{\code{layer}\xspace}
\newcommand{\before}{\code{before}\xspace}
\newcommand{\after}{\code{after}\xspace}
\newcommand{\finalmod}{\code{final}\xspace}
\newcommand{\abstractmod}{\code{abstract}\xspace}
\newcommand{\proceed}{\code{proceed}\xspace} 
\newcommand{\ind}{\hspace*{1.5em}}



\lstdefinelanguage{ContextJS} {
      classoffset=0,
      keywords={cop, proceed, withLayer, withoutLayer, with, break,%
                layerClass, layerObject, createLayer, %
        case, catch, continue, else, extends,%
                false, finally, function, for, if, var, instanceof,%
                new, %
                return, super, this, throw,%
                true, try, undefined, while, },
      morecomment=[l]//,%    
      morecomment=[s]{/*}{*/},%
      morestring=[b]",%
      % basicstyle=\listingsfont,
      % basicstyle= \ttfamily \scriptsize \selectfont
      keywordstyle = \color[rgb]{0.54,0.27,0.00},
      %commentstyle = \color[rgb]{0.0,0.533,0.8},
      stringstyle = \color[rgb]{0.0,0.533,0.8},
      identifierstyle=,
      showstringspaces=false,
      captionpos=b,
      tabsize=2,
      aboveskip=5mm,
      %numbersep=-10pt,      
      extendedchars=true,
      frame=tb,
      framexleftmargin=0pt,
      framexrightmargin=-2pt,
      numbers=left,
      numberstyle=\tiny    
}


% -- None ---

\lstdefinelanguage{todo} {
  basicstyle=\color{MidnightBlue}\scriptsize\ttfamily,
  breaklines=true,
  xleftmargin=0.5cm,
  extendedchars=true,
  frame=none
}

\lstdefinelanguage{none} {
  keywords={},
  alsoletter={+},
  morekeywords={}
  frame=none
}

% -----------
  
  % general config
  \lstset{
    framexrightmargin=0pt,      
    numberstyle=\tiny,
    numbersep=5pt,      
    frame=tb,
    basicstyle=\linespread{1} \listingsfont, % print whole listing small
    keywordstyle = \bfseries \color[rgb]{0.49,0.00,0.33},%\color[rgb]{0.84,0.27,0.40},
    commentstyle = \color[rgb]{0.247,0.498,0.372},
    stringstyle = \color[rgb]{0.16,0.2,1},    
    captionpos=b,
    tabsize=2,
    aboveskip=5mm,
    showlines=true,
    showstringspaces=false,
    escapechar=\§,
    breakautoindent=false    
  }
  
  
  \newcommand{\lstnolinenumbers}{
  	\lstset{
  	  numbers=none,   	  
      framexleftmargin=4pt,
      xleftmargin=4pt,   
    }
  }
  	
  
  \newcommand{\lstlinenumbers}{ 
  	\lstset{
  	  numbers=left,
      framexleftmargin=15pt,
      xleftmargin=15pt,            
    }
}



% Paperref ==========================================================================
% includes a published note at the top of the title page for local copies

\usepackage[absolute]{textpos}
%\setlength{\TPHorizModule}{0.8\textwidth}
\definecolor{shadecolor}{rgb}{0.8,0.8,0.8} 

\newcommand{\paperref}[3]{%  
  \setlength{\TPHorizModule}{#1}
  \textblockorigin{0mm}{#2}
  \begin{textblock}{1}(0,0)    
      \begin{shaded}#3\end{shaded}    
  \end{textblock}
}



% \usepackage{verbatim}% http://ctan.org/pkg/verbatim
% \newenvironment{note}{\endgraf\color{MidnightBlue}\scriptsize\verbatim}{\endverbatim}

\newenvironment{rewrite}{\endgraf\color{MidnightBlue}}{}


%\usepackage{fancyvrb, xstring}
%\DefineVerbatimEnvironment{note}{Verbatim}{
%    formatcom=\color{MidnightBlue}\scriptsize
%}

%\renewcommand{\FancyVerbFormatLine}[1]{
%   {\StrSubstitute[1]{#1}{☐}{-}}
%}

% \usepackage{listings}

\lstnewenvironment{note}%
  {%
    %\minipage{\textwidth} 
    \lstset{
      basicstyle=\color{MidnightBlue}\scriptsize\ttfamily,
      breaklines=true,
      xleftmargin=0.5cm,
      extendedchars=true,
      frame=none
    }
  }
  {
    %\endminipage
  }

\usepackage{grfext}
\PrependGraphicsExtensions*{.pdf}


% \usepackage[toc]{glossaries}
% \makeglossaries

\usepackage[nopostdot,nonumberlist,nomain,acronym,xindy%,toc
]{glossaries} % nomain, if you define glossaries in a file, and you use \include{INP-00-glossary}
\makeglossaries
\usepackage[xindy]{imakeidx}
\makeindex

\renewcommand*{\glsnamefont}[1]{\textbf{#1}}



% \commentmode
%
% - Draft Notice and Svn Info in a Framed Header on Every Page. This feature is automatically 
%   activated while in comment mode, but needs an additional declaration in the preamble to set 
%   the SVN Id: \def\svnId{$$Id:$$}
% 
%   \documentclass{...}
%   \def\svnId{$$Id:$$}
%    %TEX root = main.tex
%%%%%%%%%%%%%%%%%%%%%%%%%%%%%%%%%%%%%%%%%%%%%%%%%%%%%%%%%%%%%%%%%%%%%%%%%%%%%%%%%%%%%%%%%%%%%%%%%%%
%
% This file contains:
%
% Comment Mode Support 
% ==================
% The following features are only active while in comment mode. To enter this mode, simply 
% set \commentmode in the preamble. 
%
% \documentclass{...}
% \input{auxiliary}
% \commentmode
%
% - Draft Notice and Svn Info in a Framed Header on Every Page. This feature is automatically 
%   activated while in comment mode, but needs an additional declaration in the preamble to set 
%   the SVN Id: \def\svnId{$$Id:$$}
% 
%   \documentclass{...}
%   \def\svnId{$$Id:$$}
%   \input{auxiliary}
%   \commentmode
%
% - Comments Labeled with Author Id. Define your own author id before \begin{document}:
%   \newcommand\myAuthorID[1]{\nbnote{MAID}{#1}}
%   Use it anywhere in the text: 
%   ... \myAuthorID{this is a comment} ... 
%
% Listings Package Configuration for Several Languages
% ====================================================
% - Syntax highlighting for JCop, ContextJ, LogicAJ, LogicAJ2, GenTL, Prolog
% - Code makro to define code snippets independently of their appeariance. Use:
%   \code{public void foo(){}}
%
% Published Notice on Title Page
% ============================= 
% Paper reference feature to place a publication note on top of the title page of local downloads. 
% Use the following command after \begin{document}: 
% \paperref{<framed box width>}{<height>}{This paper appeared at the n-th XY Conference on Z}
%
% last modified: MA, 2009-08-17_14-47
%%%%%%%%%%%%%%%%%%%%%%%%%%%%%%%%%%%%%%%%%%%%%%%%%%%%%%%%%%%%%%%%%%%%%%%%%%%%%%%%%%%%%%%%%%%%%%%%%%%
\usepackage{amsmath}
\usepackage[usenames,dvipsnames]{color}

\usepackage{framed}
\usepackage{xspace}
\usepackage[hyphens]{url}
%\urlstyle{same}
\urlstyle{tt}
%\usepackage{caption}
\DeclareMathAlphabet{\mathpzc}{OT1}{pzc}{m}{it}
% General %%%%%%%%%%%%%%%%%%%%%%%%%%%%%%%%%%%%%%%%%%%%%%%%%%%%%%%%%%%%%%%%%%%%%%%%%%%%%%%%%%%%%%%%%

% API Table %%%%%%%%%%%%%%%%%%%%%%%%%%%%%%%%%%%%%%%%%%%%%%%%%%%%%%%%%%%%%%%%%%%%%%%%%%%%%%%
\usepackage{longtable}
\usepackage{multirow}

\newenvironment{api2}[1]{%
\begin{tabular}{p{.5cm} p{13.5cm}}%
\hline%
%\multicolumn{2}{|c|}{\code{#1}}\\ \hline %
}%
{\hline%
\end{tabular}%
}
\newcommand{\apidesctxt}[2]{\multicolumn{2}{l}{\lstinline[language=JCop, basicstyle=\listingsfont]{public #1}} \\ & #2 \\}

\newenvironment{api}[1]{%
\begin{longtable}{p{.5cm} p{12.5cm}}%
%\hline%
%\multicolumn{2}{|c|}{\code{#1}}\\ \hline %
}%
{%\hline%
\end{longtable}%
}

\newcommand{\apidesc}[2]{\multicolumn{2}{l}{\lstinline[language=JCop, basicstyle=\listingsfont]{public #1}} \\ & \footnotesize{#2} \\}


% COMMENTS & DRAFT  %%%%%%%%%%%%%%%%%%%%%%%%%%%%%%%%%%%%%%%%%%%%%%%%%%%%%%%%%%%%%%%%%%%%%%%%%%%%%%%

\usepackage{ednotes}
\usepackage{amssymb}
\usepackage{datetime}
\usepackage{everypage}

% date and timeformat
\newdateformat{simpledate}{\THEYEAR-\twodigit{\THEMONTH}-\twodigit{\THEDAY}}
\newtimeformat{simpletime}{\twodigit{\THEHOUR}:\twodigit{\THEMINUTE}}
\newcommand{\timestamp}{\simpledate\today~\simpletime}

% header for draft notice
\newcommand{\draftheader}{%
  %\framebox[\textwidth][t]{%
	  \centerline	
}

% the default definitions of \nbnote and \svnversion are empty	
\newcommand{\nbnote}[2]{}
\newcommand{\svnversion}{}

% the actions perfomed while in comment mode
\newcommand{\commentmode}{
		% \usepackage{draftwatermark} \SetWatermarkLightness{0.70} \SetWatermarkText{\Huge DRAFT}   	
		\renewcommand{\nbnote}[2]{%
		  \fbox{\bfseries\sffamily\scriptsize##1}
		  {\sf\small $\blacktriangleright$ \textit##2 $\blacktriangleleft$}
		  %{\sf\small$\blacktriangleright$\textit{##2}$\blacktriangleleft$}%
		}
	 \renewcommand{\svnversion}{\emph{\footnotesize\svnId}}
	 \AddEverypageHook{\paperref{21cm}{-3mm}{\draftheader}}      
	 \AddEverypageHook{\paperref{21cm}{28.5cm}{\draftheader}}      
}


\newcommand\todo[1]{\nbnote{TO DO}{#1}}

\newcommand{\here}{\nbnote{***}{CONTINUE HERE}}
\newcommand\fix[1]{\nbnote{FIX}{#1}}
\newcommand\jl[1]{\nbnote{JL}{#1}}

\newcommand\rw[1]{\color{blue}#1\color{black}}

% dangerous!
% \DeclareOption{comments}{\commentmode}
% \ProcessOptions 

% LISTINGS  %%%%%%%%%%%%%%%%%%%%%%%%%%%%%%%%%%%%%%%%%%%%%%%%%%%%%%%%%%%%%%%%%%%%%%%%%%%%%%%%%%%%%%%

\usepackage[formats]{listings}
\usepackage[T1]{fontenc}


%\usepackage{pifont}
%\usepackage{winfonts}
%\usepackage{windingbats}
%\usepackage{winpifont}
\usepackage{soul}
%\DisableLigatures{encoding = T1, family = courier }
%\usepackage{beramono}
\usepackage[scaled=.84]{DejaVuSansMono}
%\linespread{1.5}  % mathpazo

%\fontfamily{courier} \selectfont 

\newcommand{\code}[1]{\mbox{\texttt{#1}}} % code font
%\let\code\lstinline
\newcommand{\scode}[1]{\texttt{\small{#1}}} % code font
%\newcommand{\code}[1]{%
%  \lstinline[language=JCop,%
%  					 basicstyle={\ttfamily \footnotesize \selectfont},%
%             keywordstyle={\bfseries \color[rgb]{0,0,0}}]!#1!} % code font
\newcommand{\biglistingsfont}{ \ttfamily \normalsize \selectfont }
\newcommand{\listingsfont}{ \ttfamily \scriptsize \selectfont }
\newcommand{\tinylistingsfont}{\rmfamily \fontsize{6.5}{6.5} \selectfont}
%\newcommand{\listingsfont}{ \fontfamily{courier} \scalebox{.5}[1] \scriptsize \selectfont }
%\newcommand{\tinylistingsfont}{\fontfamily{courier} \fontsize{6.5}{5}  \selectfont \scalebox{.5}[1]}

%\newcommand{\listingsfont}{\ttfamily \scriptsize}

% COLORS
% eclipse keywords: \color[rgb]{0.49,0.00,0.33}
% eclipse strings:  \color[rgb]{0.16,0.2,1}
% eclipse comments: \color[rgb]{0.247,0.498,0.372},


% -- JCop  ----------------------------------------------------------------------------------------

\newcommand{\layerclass}{\code{jcop.lang.Layer}\xspace}
\newcommand{\with}{\code{with}\xspace}
\newcommand{\without}{\code{without}\xspace}
\newcommand{\withoutall}{\code{withoutall}\xspace}
\newcommand{\layer}{\code{layer}\xspace}
\newcommand{\before}{\code{before}\xspace}
\newcommand{\after}{\code{after}\xspace}
\newcommand{\finalmod}{\code{final}\xspace}
\newcommand{\abstractmod}{\code{abstract}\xspace}
\newcommand{\proceed}{\code{proceed}\xspace} 
\newcommand{\ind}{\hspace*{1.5em}}



\lstdefinelanguage{ContextJS} {
      classoffset=0,
      keywords={cop, proceed, withLayer, withoutLayer, with, break,%
                layerClass, layerObject, createLayer, %
        case, catch, continue, else, extends,%
                false, finally, function, for, if, var, instanceof,%
                new, %
                return, super, this, throw,%
                true, try, undefined, while, },
      morecomment=[l]//,%    
      morecomment=[s]{/*}{*/},%
      morestring=[b]",%
      % basicstyle=\listingsfont,
      % basicstyle= \ttfamily \scriptsize \selectfont
      keywordstyle = \color[rgb]{0.54,0.27,0.00},
      %commentstyle = \color[rgb]{0.0,0.533,0.8},
      stringstyle = \color[rgb]{0.0,0.533,0.8},
      identifierstyle=,
      showstringspaces=false,
      captionpos=b,
      tabsize=2,
      aboveskip=5mm,
      %numbersep=-10pt,      
      extendedchars=true,
      frame=tb,
      framexleftmargin=0pt,
      framexrightmargin=-2pt,
      numbers=left,
      numberstyle=\tiny    
}


% -- None ---

\lstdefinelanguage{todo} {
  basicstyle=\color{MidnightBlue}\scriptsize\ttfamily,
  breaklines=true,
  xleftmargin=0.5cm,
  extendedchars=true,
  frame=none
}

\lstdefinelanguage{none} {
  keywords={},
  alsoletter={+},
  morekeywords={}
  frame=none
}

% -----------
  
  % general config
  \lstset{
    framexrightmargin=0pt,      
    numberstyle=\tiny,
    numbersep=5pt,      
    frame=tb,
    basicstyle=\linespread{1} \listingsfont, % print whole listing small
    keywordstyle = \bfseries \color[rgb]{0.49,0.00,0.33},%\color[rgb]{0.84,0.27,0.40},
    commentstyle = \color[rgb]{0.247,0.498,0.372},
    stringstyle = \color[rgb]{0.16,0.2,1},    
    captionpos=b,
    tabsize=2,
    aboveskip=5mm,
    showlines=true,
    showstringspaces=false,
    escapechar=\§,
    breakautoindent=false    
  }
  
  
  \newcommand{\lstnolinenumbers}{
  	\lstset{
  	  numbers=none,   	  
      framexleftmargin=4pt,
      xleftmargin=4pt,   
    }
  }
  	
  
  \newcommand{\lstlinenumbers}{ 
  	\lstset{
  	  numbers=left,
      framexleftmargin=15pt,
      xleftmargin=15pt,            
    }
}



% Paperref ==========================================================================
% includes a published note at the top of the title page for local copies

\usepackage[absolute]{textpos}
%\setlength{\TPHorizModule}{0.8\textwidth}
\definecolor{shadecolor}{rgb}{0.8,0.8,0.8} 

\newcommand{\paperref}[3]{%  
  \setlength{\TPHorizModule}{#1}
  \textblockorigin{0mm}{#2}
  \begin{textblock}{1}(0,0)    
      \begin{shaded}#3\end{shaded}    
  \end{textblock}
}



% \usepackage{verbatim}% http://ctan.org/pkg/verbatim
% \newenvironment{note}{\endgraf\color{MidnightBlue}\scriptsize\verbatim}{\endverbatim}

\newenvironment{rewrite}{\endgraf\color{MidnightBlue}}{}


%\usepackage{fancyvrb, xstring}
%\DefineVerbatimEnvironment{note}{Verbatim}{
%    formatcom=\color{MidnightBlue}\scriptsize
%}

%\renewcommand{\FancyVerbFormatLine}[1]{
%   {\StrSubstitute[1]{#1}{☐}{-}}
%}

% \usepackage{listings}

\lstnewenvironment{note}%
  {%
    %\minipage{\textwidth} 
    \lstset{
      basicstyle=\color{MidnightBlue}\scriptsize\ttfamily,
      breaklines=true,
      xleftmargin=0.5cm,
      extendedchars=true,
      frame=none
    }
  }
  {
    %\endminipage
  }

\usepackage{grfext}
\PrependGraphicsExtensions*{.pdf}


% \usepackage[toc]{glossaries}
% \makeglossaries

\usepackage[nopostdot,nonumberlist,nomain,acronym,xindy%,toc
]{glossaries} % nomain, if you define glossaries in a file, and you use \include{INP-00-glossary}
\makeglossaries
\usepackage[xindy]{imakeidx}
\makeindex

\renewcommand*{\glsnamefont}[1]{\textbf{#1}}



%   \commentmode
%
% - Comments Labeled with Author Id. Define your own author id before \begin{document}:
%   \newcommand\myAuthorID[1]{\nbnote{MAID}{#1}}
%   Use it anywhere in the text: 
%   ... \myAuthorID{this is a comment} ... 
%
% Listings Package Configuration for Several Languages
% ====================================================
% - Syntax highlighting for JCop, ContextJ, LogicAJ, LogicAJ2, GenTL, Prolog
% - Code makro to define code snippets independently of their appeariance. Use:
%   \code{public void foo(){}}
%
% Published Notice on Title Page
% ============================= 
% Paper reference feature to place a publication note on top of the title page of local downloads. 
% Use the following command after \begin{document}: 
% \paperref{<framed box width>}{<height>}{This paper appeared at the n-th XY Conference on Z}
%
% last modified: MA, 2009-08-17_14-47
%%%%%%%%%%%%%%%%%%%%%%%%%%%%%%%%%%%%%%%%%%%%%%%%%%%%%%%%%%%%%%%%%%%%%%%%%%%%%%%%%%%%%%%%%%%%%%%%%%%
\usepackage{amsmath}
\usepackage[usenames,dvipsnames]{color}

\usepackage{framed}
\usepackage{xspace}
\usepackage[hyphens]{url}
%\urlstyle{same}
\urlstyle{tt}
%\usepackage{caption}
\DeclareMathAlphabet{\mathpzc}{OT1}{pzc}{m}{it}
% General %%%%%%%%%%%%%%%%%%%%%%%%%%%%%%%%%%%%%%%%%%%%%%%%%%%%%%%%%%%%%%%%%%%%%%%%%%%%%%%%%%%%%%%%%

% API Table %%%%%%%%%%%%%%%%%%%%%%%%%%%%%%%%%%%%%%%%%%%%%%%%%%%%%%%%%%%%%%%%%%%%%%%%%%%%%%%
\usepackage{longtable}
\usepackage{multirow}

\newenvironment{api2}[1]{%
\begin{tabular}{p{.5cm} p{13.5cm}}%
\hline%
%\multicolumn{2}{|c|}{\code{#1}}\\ \hline %
}%
{\hline%
\end{tabular}%
}
\newcommand{\apidesctxt}[2]{\multicolumn{2}{l}{\lstinline[language=JCop, basicstyle=\listingsfont]{public #1}} \\ & #2 \\}

\newenvironment{api}[1]{%
\begin{longtable}{p{.5cm} p{12.5cm}}%
%\hline%
%\multicolumn{2}{|c|}{\code{#1}}\\ \hline %
}%
{%\hline%
\end{longtable}%
}

\newcommand{\apidesc}[2]{\multicolumn{2}{l}{\lstinline[language=JCop, basicstyle=\listingsfont]{public #1}} \\ & \footnotesize{#2} \\}


% COMMENTS & DRAFT  %%%%%%%%%%%%%%%%%%%%%%%%%%%%%%%%%%%%%%%%%%%%%%%%%%%%%%%%%%%%%%%%%%%%%%%%%%%%%%%

\usepackage{ednotes}
\usepackage{amssymb}
\usepackage{datetime}
\usepackage{everypage}

% date and timeformat
\newdateformat{simpledate}{\THEYEAR-\twodigit{\THEMONTH}-\twodigit{\THEDAY}}
\newtimeformat{simpletime}{\twodigit{\THEHOUR}:\twodigit{\THEMINUTE}}
\newcommand{\timestamp}{\simpledate\today~\simpletime}

% header for draft notice
\newcommand{\draftheader}{%
  %\framebox[\textwidth][t]{%
	  \centerline	
}

% the default definitions of \nbnote and \svnversion are empty	
\newcommand{\nbnote}[2]{}
\newcommand{\svnversion}{}

% the actions perfomed while in comment mode
\newcommand{\commentmode}{
		% \usepackage{draftwatermark} \SetWatermarkLightness{0.70} \SetWatermarkText{\Huge DRAFT}   	
		\renewcommand{\nbnote}[2]{%
		  \fbox{\bfseries\sffamily\scriptsize##1}
		  {\sf\small $\blacktriangleright$ \textit##2 $\blacktriangleleft$}
		  %{\sf\small$\blacktriangleright$\textit{##2}$\blacktriangleleft$}%
		}
	 \renewcommand{\svnversion}{\emph{\footnotesize\svnId}}
	 \AddEverypageHook{\paperref{21cm}{-3mm}{\draftheader}}      
	 \AddEverypageHook{\paperref{21cm}{28.5cm}{\draftheader}}      
}


\newcommand\todo[1]{\nbnote{TO DO}{#1}}

\newcommand{\here}{\nbnote{***}{CONTINUE HERE}}
\newcommand\fix[1]{\nbnote{FIX}{#1}}
\newcommand\jl[1]{\nbnote{JL}{#1}}

\newcommand\rw[1]{\color{blue}#1\color{black}}

% dangerous!
% \DeclareOption{comments}{\commentmode}
% \ProcessOptions 

% LISTINGS  %%%%%%%%%%%%%%%%%%%%%%%%%%%%%%%%%%%%%%%%%%%%%%%%%%%%%%%%%%%%%%%%%%%%%%%%%%%%%%%%%%%%%%%

\usepackage[formats]{listings}
\usepackage[T1]{fontenc}


%\usepackage{pifont}
%\usepackage{winfonts}
%\usepackage{windingbats}
%\usepackage{winpifont}
\usepackage{soul}
%\DisableLigatures{encoding = T1, family = courier }
%\usepackage{beramono}
\usepackage[scaled=.84]{DejaVuSansMono}
%\linespread{1.5}  % mathpazo

%\fontfamily{courier} \selectfont 

\newcommand{\code}[1]{\mbox{\texttt{#1}}} % code font
%\let\code\lstinline
\newcommand{\scode}[1]{\texttt{\small{#1}}} % code font
%\newcommand{\code}[1]{%
%  \lstinline[language=JCop,%
%  					 basicstyle={\ttfamily \footnotesize \selectfont},%
%             keywordstyle={\bfseries \color[rgb]{0,0,0}}]!#1!} % code font
\newcommand{\biglistingsfont}{ \ttfamily \normalsize \selectfont }
\newcommand{\listingsfont}{ \ttfamily \scriptsize \selectfont }
\newcommand{\tinylistingsfont}{\rmfamily \fontsize{6.5}{6.5} \selectfont}
%\newcommand{\listingsfont}{ \fontfamily{courier} \scalebox{.5}[1] \scriptsize \selectfont }
%\newcommand{\tinylistingsfont}{\fontfamily{courier} \fontsize{6.5}{5}  \selectfont \scalebox{.5}[1]}

%\newcommand{\listingsfont}{\ttfamily \scriptsize}

% COLORS
% eclipse keywords: \color[rgb]{0.49,0.00,0.33}
% eclipse strings:  \color[rgb]{0.16,0.2,1}
% eclipse comments: \color[rgb]{0.247,0.498,0.372},


% -- JCop  ----------------------------------------------------------------------------------------

\newcommand{\layerclass}{\code{jcop.lang.Layer}\xspace}
\newcommand{\with}{\code{with}\xspace}
\newcommand{\without}{\code{without}\xspace}
\newcommand{\withoutall}{\code{withoutall}\xspace}
\newcommand{\layer}{\code{layer}\xspace}
\newcommand{\before}{\code{before}\xspace}
\newcommand{\after}{\code{after}\xspace}
\newcommand{\finalmod}{\code{final}\xspace}
\newcommand{\abstractmod}{\code{abstract}\xspace}
\newcommand{\proceed}{\code{proceed}\xspace} 
\newcommand{\ind}{\hspace*{1.5em}}



\lstdefinelanguage{ContextJS} {
      classoffset=0,
      keywords={cop, proceed, withLayer, withoutLayer, with, break,%
                layerClass, layerObject, createLayer, %
        case, catch, continue, else, extends,%
                false, finally, function, for, if, var, instanceof,%
                new, %
                return, super, this, throw,%
                true, try, undefined, while, },
      morecomment=[l]//,%    
      morecomment=[s]{/*}{*/},%
      morestring=[b]",%
      % basicstyle=\listingsfont,
      % basicstyle= \ttfamily \scriptsize \selectfont
      keywordstyle = \color[rgb]{0.54,0.27,0.00},
      %commentstyle = \color[rgb]{0.0,0.533,0.8},
      stringstyle = \color[rgb]{0.0,0.533,0.8},
      identifierstyle=,
      showstringspaces=false,
      captionpos=b,
      tabsize=2,
      aboveskip=5mm,
      %numbersep=-10pt,      
      extendedchars=true,
      frame=tb,
      framexleftmargin=0pt,
      framexrightmargin=-2pt,
      numbers=left,
      numberstyle=\tiny    
}


% -- None ---

\lstdefinelanguage{todo} {
  basicstyle=\color{MidnightBlue}\scriptsize\ttfamily,
  breaklines=true,
  xleftmargin=0.5cm,
  extendedchars=true,
  frame=none
}

\lstdefinelanguage{none} {
  keywords={},
  alsoletter={+},
  morekeywords={}
  frame=none
}

% -----------
  
  % general config
  \lstset{
    framexrightmargin=0pt,      
    numberstyle=\tiny,
    numbersep=5pt,      
    frame=tb,
    basicstyle=\linespread{1} \listingsfont, % print whole listing small
    keywordstyle = \bfseries \color[rgb]{0.49,0.00,0.33},%\color[rgb]{0.84,0.27,0.40},
    commentstyle = \color[rgb]{0.247,0.498,0.372},
    stringstyle = \color[rgb]{0.16,0.2,1},    
    captionpos=b,
    tabsize=2,
    aboveskip=5mm,
    showlines=true,
    showstringspaces=false,
    escapechar=\§,
    breakautoindent=false    
  }
  
  
  \newcommand{\lstnolinenumbers}{
  	\lstset{
  	  numbers=none,   	  
      framexleftmargin=4pt,
      xleftmargin=4pt,   
    }
  }
  	
  
  \newcommand{\lstlinenumbers}{ 
  	\lstset{
  	  numbers=left,
      framexleftmargin=15pt,
      xleftmargin=15pt,            
    }
}



% Paperref ==========================================================================
% includes a published note at the top of the title page for local copies

\usepackage[absolute]{textpos}
%\setlength{\TPHorizModule}{0.8\textwidth}
\definecolor{shadecolor}{rgb}{0.8,0.8,0.8} 

\newcommand{\paperref}[3]{%  
  \setlength{\TPHorizModule}{#1}
  \textblockorigin{0mm}{#2}
  \begin{textblock}{1}(0,0)    
      \begin{shaded}#3\end{shaded}    
  \end{textblock}
}



% \usepackage{verbatim}% http://ctan.org/pkg/verbatim
% \newenvironment{note}{\endgraf\color{MidnightBlue}\scriptsize\verbatim}{\endverbatim}

\newenvironment{rewrite}{\endgraf\color{MidnightBlue}}{}


%\usepackage{fancyvrb, xstring}
%\DefineVerbatimEnvironment{note}{Verbatim}{
%    formatcom=\color{MidnightBlue}\scriptsize
%}

%\renewcommand{\FancyVerbFormatLine}[1]{
%   {\StrSubstitute[1]{#1}{☐}{-}}
%}

% \usepackage{listings}

\lstnewenvironment{note}%
  {%
    %\minipage{\textwidth} 
    \lstset{
      basicstyle=\color{MidnightBlue}\scriptsize\ttfamily,
      breaklines=true,
      xleftmargin=0.5cm,
      extendedchars=true,
      frame=none
    }
  }
  {
    %\endminipage
  }

\usepackage{grfext}
\PrependGraphicsExtensions*{.pdf}


% \usepackage[toc]{glossaries}
% \makeglossaries

\usepackage[nopostdot,nonumberlist,nomain,acronym,xindy%,toc
]{glossaries} % nomain, if you define glossaries in a file, and you use \include{INP-00-glossary}
\makeglossaries
\usepackage[xindy]{imakeidx}
\makeindex

\renewcommand*{\glsnamefont}[1]{\textbf{#1}}



% \commentmode
%
% - Draft Notice and Svn Info in a Framed Header on Every Page. This feature is automatically 
%   activated while in comment mode, but needs an additional declaration in the preamble to set 
%   the SVN Id: \def\svnId{$$Id:$$}
% 
%   \documentclass{...}
%   \def\svnId{$$Id:$$}
%    %TEX root = main.tex
%%%%%%%%%%%%%%%%%%%%%%%%%%%%%%%%%%%%%%%%%%%%%%%%%%%%%%%%%%%%%%%%%%%%%%%%%%%%%%%%%%%%%%%%%%%%%%%%%%%
%
% This file contains:
%
% Comment Mode Support 
% ==================
% The following features are only active while in comment mode. To enter this mode, simply 
% set \commentmode in the preamble. 
%
% \documentclass{...}
%  %TEX root = main.tex
%%%%%%%%%%%%%%%%%%%%%%%%%%%%%%%%%%%%%%%%%%%%%%%%%%%%%%%%%%%%%%%%%%%%%%%%%%%%%%%%%%%%%%%%%%%%%%%%%%%
%
% This file contains:
%
% Comment Mode Support 
% ==================
% The following features are only active while in comment mode. To enter this mode, simply 
% set \commentmode in the preamble. 
%
% \documentclass{...}
% \input{auxiliary}
% \commentmode
%
% - Draft Notice and Svn Info in a Framed Header on Every Page. This feature is automatically 
%   activated while in comment mode, but needs an additional declaration in the preamble to set 
%   the SVN Id: \def\svnId{$$Id:$$}
% 
%   \documentclass{...}
%   \def\svnId{$$Id:$$}
%   \input{auxiliary}
%   \commentmode
%
% - Comments Labeled with Author Id. Define your own author id before \begin{document}:
%   \newcommand\myAuthorID[1]{\nbnote{MAID}{#1}}
%   Use it anywhere in the text: 
%   ... \myAuthorID{this is a comment} ... 
%
% Listings Package Configuration for Several Languages
% ====================================================
% - Syntax highlighting for JCop, ContextJ, LogicAJ, LogicAJ2, GenTL, Prolog
% - Code makro to define code snippets independently of their appeariance. Use:
%   \code{public void foo(){}}
%
% Published Notice on Title Page
% ============================= 
% Paper reference feature to place a publication note on top of the title page of local downloads. 
% Use the following command after \begin{document}: 
% \paperref{<framed box width>}{<height>}{This paper appeared at the n-th XY Conference on Z}
%
% last modified: MA, 2009-08-17_14-47
%%%%%%%%%%%%%%%%%%%%%%%%%%%%%%%%%%%%%%%%%%%%%%%%%%%%%%%%%%%%%%%%%%%%%%%%%%%%%%%%%%%%%%%%%%%%%%%%%%%
\usepackage{amsmath}
\usepackage[usenames,dvipsnames]{color}

\usepackage{framed}
\usepackage{xspace}
\usepackage[hyphens]{url}
%\urlstyle{same}
\urlstyle{tt}
%\usepackage{caption}
\DeclareMathAlphabet{\mathpzc}{OT1}{pzc}{m}{it}
% General %%%%%%%%%%%%%%%%%%%%%%%%%%%%%%%%%%%%%%%%%%%%%%%%%%%%%%%%%%%%%%%%%%%%%%%%%%%%%%%%%%%%%%%%%

% API Table %%%%%%%%%%%%%%%%%%%%%%%%%%%%%%%%%%%%%%%%%%%%%%%%%%%%%%%%%%%%%%%%%%%%%%%%%%%%%%%
\usepackage{longtable}
\usepackage{multirow}

\newenvironment{api2}[1]{%
\begin{tabular}{p{.5cm} p{13.5cm}}%
\hline%
%\multicolumn{2}{|c|}{\code{#1}}\\ \hline %
}%
{\hline%
\end{tabular}%
}
\newcommand{\apidesctxt}[2]{\multicolumn{2}{l}{\lstinline[language=JCop, basicstyle=\listingsfont]{public #1}} \\ & #2 \\}

\newenvironment{api}[1]{%
\begin{longtable}{p{.5cm} p{12.5cm}}%
%\hline%
%\multicolumn{2}{|c|}{\code{#1}}\\ \hline %
}%
{%\hline%
\end{longtable}%
}

\newcommand{\apidesc}[2]{\multicolumn{2}{l}{\lstinline[language=JCop, basicstyle=\listingsfont]{public #1}} \\ & \footnotesize{#2} \\}


% COMMENTS & DRAFT  %%%%%%%%%%%%%%%%%%%%%%%%%%%%%%%%%%%%%%%%%%%%%%%%%%%%%%%%%%%%%%%%%%%%%%%%%%%%%%%

\usepackage{ednotes}
\usepackage{amssymb}
\usepackage{datetime}
\usepackage{everypage}

% date and timeformat
\newdateformat{simpledate}{\THEYEAR-\twodigit{\THEMONTH}-\twodigit{\THEDAY}}
\newtimeformat{simpletime}{\twodigit{\THEHOUR}:\twodigit{\THEMINUTE}}
\newcommand{\timestamp}{\simpledate\today~\simpletime}

% header for draft notice
\newcommand{\draftheader}{%
  %\framebox[\textwidth][t]{%
	  \centerline	
}

% the default definitions of \nbnote and \svnversion are empty	
\newcommand{\nbnote}[2]{}
\newcommand{\svnversion}{}

% the actions perfomed while in comment mode
\newcommand{\commentmode}{
		% \usepackage{draftwatermark} \SetWatermarkLightness{0.70} \SetWatermarkText{\Huge DRAFT}   	
		\renewcommand{\nbnote}[2]{%
		  \fbox{\bfseries\sffamily\scriptsize##1}
		  {\sf\small $\blacktriangleright$ \textit##2 $\blacktriangleleft$}
		  %{\sf\small$\blacktriangleright$\textit{##2}$\blacktriangleleft$}%
		}
	 \renewcommand{\svnversion}{\emph{\footnotesize\svnId}}
	 \AddEverypageHook{\paperref{21cm}{-3mm}{\draftheader}}      
	 \AddEverypageHook{\paperref{21cm}{28.5cm}{\draftheader}}      
}


\newcommand\todo[1]{\nbnote{TO DO}{#1}}

\newcommand{\here}{\nbnote{***}{CONTINUE HERE}}
\newcommand\fix[1]{\nbnote{FIX}{#1}}
\newcommand\jl[1]{\nbnote{JL}{#1}}

\newcommand\rw[1]{\color{blue}#1\color{black}}

% dangerous!
% \DeclareOption{comments}{\commentmode}
% \ProcessOptions 

% LISTINGS  %%%%%%%%%%%%%%%%%%%%%%%%%%%%%%%%%%%%%%%%%%%%%%%%%%%%%%%%%%%%%%%%%%%%%%%%%%%%%%%%%%%%%%%

\usepackage[formats]{listings}
\usepackage[T1]{fontenc}


%\usepackage{pifont}
%\usepackage{winfonts}
%\usepackage{windingbats}
%\usepackage{winpifont}
\usepackage{soul}
%\DisableLigatures{encoding = T1, family = courier }
%\usepackage{beramono}
\usepackage[scaled=.84]{DejaVuSansMono}
%\linespread{1.5}  % mathpazo

%\fontfamily{courier} \selectfont 

\newcommand{\code}[1]{\mbox{\texttt{#1}}} % code font
%\let\code\lstinline
\newcommand{\scode}[1]{\texttt{\small{#1}}} % code font
%\newcommand{\code}[1]{%
%  \lstinline[language=JCop,%
%  					 basicstyle={\ttfamily \footnotesize \selectfont},%
%             keywordstyle={\bfseries \color[rgb]{0,0,0}}]!#1!} % code font
\newcommand{\biglistingsfont}{ \ttfamily \normalsize \selectfont }
\newcommand{\listingsfont}{ \ttfamily \scriptsize \selectfont }
\newcommand{\tinylistingsfont}{\rmfamily \fontsize{6.5}{6.5} \selectfont}
%\newcommand{\listingsfont}{ \fontfamily{courier} \scalebox{.5}[1] \scriptsize \selectfont }
%\newcommand{\tinylistingsfont}{\fontfamily{courier} \fontsize{6.5}{5}  \selectfont \scalebox{.5}[1]}

%\newcommand{\listingsfont}{\ttfamily \scriptsize}

% COLORS
% eclipse keywords: \color[rgb]{0.49,0.00,0.33}
% eclipse strings:  \color[rgb]{0.16,0.2,1}
% eclipse comments: \color[rgb]{0.247,0.498,0.372},


% -- JCop  ----------------------------------------------------------------------------------------

\newcommand{\layerclass}{\code{jcop.lang.Layer}\xspace}
\newcommand{\with}{\code{with}\xspace}
\newcommand{\without}{\code{without}\xspace}
\newcommand{\withoutall}{\code{withoutall}\xspace}
\newcommand{\layer}{\code{layer}\xspace}
\newcommand{\before}{\code{before}\xspace}
\newcommand{\after}{\code{after}\xspace}
\newcommand{\finalmod}{\code{final}\xspace}
\newcommand{\abstractmod}{\code{abstract}\xspace}
\newcommand{\proceed}{\code{proceed}\xspace} 
\newcommand{\ind}{\hspace*{1.5em}}



\lstdefinelanguage{ContextJS} {
      classoffset=0,
      keywords={cop, proceed, withLayer, withoutLayer, with, break,%
                layerClass, layerObject, createLayer, %
        case, catch, continue, else, extends,%
                false, finally, function, for, if, var, instanceof,%
                new, %
                return, super, this, throw,%
                true, try, undefined, while, },
      morecomment=[l]//,%    
      morecomment=[s]{/*}{*/},%
      morestring=[b]",%
      % basicstyle=\listingsfont,
      % basicstyle= \ttfamily \scriptsize \selectfont
      keywordstyle = \color[rgb]{0.54,0.27,0.00},
      %commentstyle = \color[rgb]{0.0,0.533,0.8},
      stringstyle = \color[rgb]{0.0,0.533,0.8},
      identifierstyle=,
      showstringspaces=false,
      captionpos=b,
      tabsize=2,
      aboveskip=5mm,
      %numbersep=-10pt,      
      extendedchars=true,
      frame=tb,
      framexleftmargin=0pt,
      framexrightmargin=-2pt,
      numbers=left,
      numberstyle=\tiny    
}


% -- None ---

\lstdefinelanguage{todo} {
  basicstyle=\color{MidnightBlue}\scriptsize\ttfamily,
  breaklines=true,
  xleftmargin=0.5cm,
  extendedchars=true,
  frame=none
}

\lstdefinelanguage{none} {
  keywords={},
  alsoletter={+},
  morekeywords={}
  frame=none
}

% -----------
  
  % general config
  \lstset{
    framexrightmargin=0pt,      
    numberstyle=\tiny,
    numbersep=5pt,      
    frame=tb,
    basicstyle=\linespread{1} \listingsfont, % print whole listing small
    keywordstyle = \bfseries \color[rgb]{0.49,0.00,0.33},%\color[rgb]{0.84,0.27,0.40},
    commentstyle = \color[rgb]{0.247,0.498,0.372},
    stringstyle = \color[rgb]{0.16,0.2,1},    
    captionpos=b,
    tabsize=2,
    aboveskip=5mm,
    showlines=true,
    showstringspaces=false,
    escapechar=\§,
    breakautoindent=false    
  }
  
  
  \newcommand{\lstnolinenumbers}{
  	\lstset{
  	  numbers=none,   	  
      framexleftmargin=4pt,
      xleftmargin=4pt,   
    }
  }
  	
  
  \newcommand{\lstlinenumbers}{ 
  	\lstset{
  	  numbers=left,
      framexleftmargin=15pt,
      xleftmargin=15pt,            
    }
}



% Paperref ==========================================================================
% includes a published note at the top of the title page for local copies

\usepackage[absolute]{textpos}
%\setlength{\TPHorizModule}{0.8\textwidth}
\definecolor{shadecolor}{rgb}{0.8,0.8,0.8} 

\newcommand{\paperref}[3]{%  
  \setlength{\TPHorizModule}{#1}
  \textblockorigin{0mm}{#2}
  \begin{textblock}{1}(0,0)    
      \begin{shaded}#3\end{shaded}    
  \end{textblock}
}



% \usepackage{verbatim}% http://ctan.org/pkg/verbatim
% \newenvironment{note}{\endgraf\color{MidnightBlue}\scriptsize\verbatim}{\endverbatim}

\newenvironment{rewrite}{\endgraf\color{MidnightBlue}}{}


%\usepackage{fancyvrb, xstring}
%\DefineVerbatimEnvironment{note}{Verbatim}{
%    formatcom=\color{MidnightBlue}\scriptsize
%}

%\renewcommand{\FancyVerbFormatLine}[1]{
%   {\StrSubstitute[1]{#1}{☐}{-}}
%}

% \usepackage{listings}

\lstnewenvironment{note}%
  {%
    %\minipage{\textwidth} 
    \lstset{
      basicstyle=\color{MidnightBlue}\scriptsize\ttfamily,
      breaklines=true,
      xleftmargin=0.5cm,
      extendedchars=true,
      frame=none
    }
  }
  {
    %\endminipage
  }

\usepackage{grfext}
\PrependGraphicsExtensions*{.pdf}


% \usepackage[toc]{glossaries}
% \makeglossaries

\usepackage[nopostdot,nonumberlist,nomain,acronym,xindy%,toc
]{glossaries} % nomain, if you define glossaries in a file, and you use \include{INP-00-glossary}
\makeglossaries
\usepackage[xindy]{imakeidx}
\makeindex

\renewcommand*{\glsnamefont}[1]{\textbf{#1}}



% \commentmode
%
% - Draft Notice and Svn Info in a Framed Header on Every Page. This feature is automatically 
%   activated while in comment mode, but needs an additional declaration in the preamble to set 
%   the SVN Id: \def\svnId{$$Id:$$}
% 
%   \documentclass{...}
%   \def\svnId{$$Id:$$}
%    %TEX root = main.tex
%%%%%%%%%%%%%%%%%%%%%%%%%%%%%%%%%%%%%%%%%%%%%%%%%%%%%%%%%%%%%%%%%%%%%%%%%%%%%%%%%%%%%%%%%%%%%%%%%%%
%
% This file contains:
%
% Comment Mode Support 
% ==================
% The following features are only active while in comment mode. To enter this mode, simply 
% set \commentmode in the preamble. 
%
% \documentclass{...}
% \input{auxiliary}
% \commentmode
%
% - Draft Notice and Svn Info in a Framed Header on Every Page. This feature is automatically 
%   activated while in comment mode, but needs an additional declaration in the preamble to set 
%   the SVN Id: \def\svnId{$$Id:$$}
% 
%   \documentclass{...}
%   \def\svnId{$$Id:$$}
%   \input{auxiliary}
%   \commentmode
%
% - Comments Labeled with Author Id. Define your own author id before \begin{document}:
%   \newcommand\myAuthorID[1]{\nbnote{MAID}{#1}}
%   Use it anywhere in the text: 
%   ... \myAuthorID{this is a comment} ... 
%
% Listings Package Configuration for Several Languages
% ====================================================
% - Syntax highlighting for JCop, ContextJ, LogicAJ, LogicAJ2, GenTL, Prolog
% - Code makro to define code snippets independently of their appeariance. Use:
%   \code{public void foo(){}}
%
% Published Notice on Title Page
% ============================= 
% Paper reference feature to place a publication note on top of the title page of local downloads. 
% Use the following command after \begin{document}: 
% \paperref{<framed box width>}{<height>}{This paper appeared at the n-th XY Conference on Z}
%
% last modified: MA, 2009-08-17_14-47
%%%%%%%%%%%%%%%%%%%%%%%%%%%%%%%%%%%%%%%%%%%%%%%%%%%%%%%%%%%%%%%%%%%%%%%%%%%%%%%%%%%%%%%%%%%%%%%%%%%
\usepackage{amsmath}
\usepackage[usenames,dvipsnames]{color}

\usepackage{framed}
\usepackage{xspace}
\usepackage[hyphens]{url}
%\urlstyle{same}
\urlstyle{tt}
%\usepackage{caption}
\DeclareMathAlphabet{\mathpzc}{OT1}{pzc}{m}{it}
% General %%%%%%%%%%%%%%%%%%%%%%%%%%%%%%%%%%%%%%%%%%%%%%%%%%%%%%%%%%%%%%%%%%%%%%%%%%%%%%%%%%%%%%%%%

% API Table %%%%%%%%%%%%%%%%%%%%%%%%%%%%%%%%%%%%%%%%%%%%%%%%%%%%%%%%%%%%%%%%%%%%%%%%%%%%%%%
\usepackage{longtable}
\usepackage{multirow}

\newenvironment{api2}[1]{%
\begin{tabular}{p{.5cm} p{13.5cm}}%
\hline%
%\multicolumn{2}{|c|}{\code{#1}}\\ \hline %
}%
{\hline%
\end{tabular}%
}
\newcommand{\apidesctxt}[2]{\multicolumn{2}{l}{\lstinline[language=JCop, basicstyle=\listingsfont]{public #1}} \\ & #2 \\}

\newenvironment{api}[1]{%
\begin{longtable}{p{.5cm} p{12.5cm}}%
%\hline%
%\multicolumn{2}{|c|}{\code{#1}}\\ \hline %
}%
{%\hline%
\end{longtable}%
}

\newcommand{\apidesc}[2]{\multicolumn{2}{l}{\lstinline[language=JCop, basicstyle=\listingsfont]{public #1}} \\ & \footnotesize{#2} \\}


% COMMENTS & DRAFT  %%%%%%%%%%%%%%%%%%%%%%%%%%%%%%%%%%%%%%%%%%%%%%%%%%%%%%%%%%%%%%%%%%%%%%%%%%%%%%%

\usepackage{ednotes}
\usepackage{amssymb}
\usepackage{datetime}
\usepackage{everypage}

% date and timeformat
\newdateformat{simpledate}{\THEYEAR-\twodigit{\THEMONTH}-\twodigit{\THEDAY}}
\newtimeformat{simpletime}{\twodigit{\THEHOUR}:\twodigit{\THEMINUTE}}
\newcommand{\timestamp}{\simpledate\today~\simpletime}

% header for draft notice
\newcommand{\draftheader}{%
  %\framebox[\textwidth][t]{%
	  \centerline	
}

% the default definitions of \nbnote and \svnversion are empty	
\newcommand{\nbnote}[2]{}
\newcommand{\svnversion}{}

% the actions perfomed while in comment mode
\newcommand{\commentmode}{
		% \usepackage{draftwatermark} \SetWatermarkLightness{0.70} \SetWatermarkText{\Huge DRAFT}   	
		\renewcommand{\nbnote}[2]{%
		  \fbox{\bfseries\sffamily\scriptsize##1}
		  {\sf\small $\blacktriangleright$ \textit##2 $\blacktriangleleft$}
		  %{\sf\small$\blacktriangleright$\textit{##2}$\blacktriangleleft$}%
		}
	 \renewcommand{\svnversion}{\emph{\footnotesize\svnId}}
	 \AddEverypageHook{\paperref{21cm}{-3mm}{\draftheader}}      
	 \AddEverypageHook{\paperref{21cm}{28.5cm}{\draftheader}}      
}


\newcommand\todo[1]{\nbnote{TO DO}{#1}}

\newcommand{\here}{\nbnote{***}{CONTINUE HERE}}
\newcommand\fix[1]{\nbnote{FIX}{#1}}
\newcommand\jl[1]{\nbnote{JL}{#1}}

\newcommand\rw[1]{\color{blue}#1\color{black}}

% dangerous!
% \DeclareOption{comments}{\commentmode}
% \ProcessOptions 

% LISTINGS  %%%%%%%%%%%%%%%%%%%%%%%%%%%%%%%%%%%%%%%%%%%%%%%%%%%%%%%%%%%%%%%%%%%%%%%%%%%%%%%%%%%%%%%

\usepackage[formats]{listings}
\usepackage[T1]{fontenc}


%\usepackage{pifont}
%\usepackage{winfonts}
%\usepackage{windingbats}
%\usepackage{winpifont}
\usepackage{soul}
%\DisableLigatures{encoding = T1, family = courier }
%\usepackage{beramono}
\usepackage[scaled=.84]{DejaVuSansMono}
%\linespread{1.5}  % mathpazo

%\fontfamily{courier} \selectfont 

\newcommand{\code}[1]{\mbox{\texttt{#1}}} % code font
%\let\code\lstinline
\newcommand{\scode}[1]{\texttt{\small{#1}}} % code font
%\newcommand{\code}[1]{%
%  \lstinline[language=JCop,%
%  					 basicstyle={\ttfamily \footnotesize \selectfont},%
%             keywordstyle={\bfseries \color[rgb]{0,0,0}}]!#1!} % code font
\newcommand{\biglistingsfont}{ \ttfamily \normalsize \selectfont }
\newcommand{\listingsfont}{ \ttfamily \scriptsize \selectfont }
\newcommand{\tinylistingsfont}{\rmfamily \fontsize{6.5}{6.5} \selectfont}
%\newcommand{\listingsfont}{ \fontfamily{courier} \scalebox{.5}[1] \scriptsize \selectfont }
%\newcommand{\tinylistingsfont}{\fontfamily{courier} \fontsize{6.5}{5}  \selectfont \scalebox{.5}[1]}

%\newcommand{\listingsfont}{\ttfamily \scriptsize}

% COLORS
% eclipse keywords: \color[rgb]{0.49,0.00,0.33}
% eclipse strings:  \color[rgb]{0.16,0.2,1}
% eclipse comments: \color[rgb]{0.247,0.498,0.372},


% -- JCop  ----------------------------------------------------------------------------------------

\newcommand{\layerclass}{\code{jcop.lang.Layer}\xspace}
\newcommand{\with}{\code{with}\xspace}
\newcommand{\without}{\code{without}\xspace}
\newcommand{\withoutall}{\code{withoutall}\xspace}
\newcommand{\layer}{\code{layer}\xspace}
\newcommand{\before}{\code{before}\xspace}
\newcommand{\after}{\code{after}\xspace}
\newcommand{\finalmod}{\code{final}\xspace}
\newcommand{\abstractmod}{\code{abstract}\xspace}
\newcommand{\proceed}{\code{proceed}\xspace} 
\newcommand{\ind}{\hspace*{1.5em}}



\lstdefinelanguage{ContextJS} {
      classoffset=0,
      keywords={cop, proceed, withLayer, withoutLayer, with, break,%
                layerClass, layerObject, createLayer, %
        case, catch, continue, else, extends,%
                false, finally, function, for, if, var, instanceof,%
                new, %
                return, super, this, throw,%
                true, try, undefined, while, },
      morecomment=[l]//,%    
      morecomment=[s]{/*}{*/},%
      morestring=[b]",%
      % basicstyle=\listingsfont,
      % basicstyle= \ttfamily \scriptsize \selectfont
      keywordstyle = \color[rgb]{0.54,0.27,0.00},
      %commentstyle = \color[rgb]{0.0,0.533,0.8},
      stringstyle = \color[rgb]{0.0,0.533,0.8},
      identifierstyle=,
      showstringspaces=false,
      captionpos=b,
      tabsize=2,
      aboveskip=5mm,
      %numbersep=-10pt,      
      extendedchars=true,
      frame=tb,
      framexleftmargin=0pt,
      framexrightmargin=-2pt,
      numbers=left,
      numberstyle=\tiny    
}


% -- None ---

\lstdefinelanguage{todo} {
  basicstyle=\color{MidnightBlue}\scriptsize\ttfamily,
  breaklines=true,
  xleftmargin=0.5cm,
  extendedchars=true,
  frame=none
}

\lstdefinelanguage{none} {
  keywords={},
  alsoletter={+},
  morekeywords={}
  frame=none
}

% -----------
  
  % general config
  \lstset{
    framexrightmargin=0pt,      
    numberstyle=\tiny,
    numbersep=5pt,      
    frame=tb,
    basicstyle=\linespread{1} \listingsfont, % print whole listing small
    keywordstyle = \bfseries \color[rgb]{0.49,0.00,0.33},%\color[rgb]{0.84,0.27,0.40},
    commentstyle = \color[rgb]{0.247,0.498,0.372},
    stringstyle = \color[rgb]{0.16,0.2,1},    
    captionpos=b,
    tabsize=2,
    aboveskip=5mm,
    showlines=true,
    showstringspaces=false,
    escapechar=\§,
    breakautoindent=false    
  }
  
  
  \newcommand{\lstnolinenumbers}{
  	\lstset{
  	  numbers=none,   	  
      framexleftmargin=4pt,
      xleftmargin=4pt,   
    }
  }
  	
  
  \newcommand{\lstlinenumbers}{ 
  	\lstset{
  	  numbers=left,
      framexleftmargin=15pt,
      xleftmargin=15pt,            
    }
}



% Paperref ==========================================================================
% includes a published note at the top of the title page for local copies

\usepackage[absolute]{textpos}
%\setlength{\TPHorizModule}{0.8\textwidth}
\definecolor{shadecolor}{rgb}{0.8,0.8,0.8} 

\newcommand{\paperref}[3]{%  
  \setlength{\TPHorizModule}{#1}
  \textblockorigin{0mm}{#2}
  \begin{textblock}{1}(0,0)    
      \begin{shaded}#3\end{shaded}    
  \end{textblock}
}



% \usepackage{verbatim}% http://ctan.org/pkg/verbatim
% \newenvironment{note}{\endgraf\color{MidnightBlue}\scriptsize\verbatim}{\endverbatim}

\newenvironment{rewrite}{\endgraf\color{MidnightBlue}}{}


%\usepackage{fancyvrb, xstring}
%\DefineVerbatimEnvironment{note}{Verbatim}{
%    formatcom=\color{MidnightBlue}\scriptsize
%}

%\renewcommand{\FancyVerbFormatLine}[1]{
%   {\StrSubstitute[1]{#1}{☐}{-}}
%}

% \usepackage{listings}

\lstnewenvironment{note}%
  {%
    %\minipage{\textwidth} 
    \lstset{
      basicstyle=\color{MidnightBlue}\scriptsize\ttfamily,
      breaklines=true,
      xleftmargin=0.5cm,
      extendedchars=true,
      frame=none
    }
  }
  {
    %\endminipage
  }

\usepackage{grfext}
\PrependGraphicsExtensions*{.pdf}


% \usepackage[toc]{glossaries}
% \makeglossaries

\usepackage[nopostdot,nonumberlist,nomain,acronym,xindy%,toc
]{glossaries} % nomain, if you define glossaries in a file, and you use \include{INP-00-glossary}
\makeglossaries
\usepackage[xindy]{imakeidx}
\makeindex

\renewcommand*{\glsnamefont}[1]{\textbf{#1}}



%   \commentmode
%
% - Comments Labeled with Author Id. Define your own author id before \begin{document}:
%   \newcommand\myAuthorID[1]{\nbnote{MAID}{#1}}
%   Use it anywhere in the text: 
%   ... \myAuthorID{this is a comment} ... 
%
% Listings Package Configuration for Several Languages
% ====================================================
% - Syntax highlighting for JCop, ContextJ, LogicAJ, LogicAJ2, GenTL, Prolog
% - Code makro to define code snippets independently of their appeariance. Use:
%   \code{public void foo(){}}
%
% Published Notice on Title Page
% ============================= 
% Paper reference feature to place a publication note on top of the title page of local downloads. 
% Use the following command after \begin{document}: 
% \paperref{<framed box width>}{<height>}{This paper appeared at the n-th XY Conference on Z}
%
% last modified: MA, 2009-08-17_14-47
%%%%%%%%%%%%%%%%%%%%%%%%%%%%%%%%%%%%%%%%%%%%%%%%%%%%%%%%%%%%%%%%%%%%%%%%%%%%%%%%%%%%%%%%%%%%%%%%%%%
\usepackage{amsmath}
\usepackage[usenames,dvipsnames]{color}

\usepackage{framed}
\usepackage{xspace}
\usepackage[hyphens]{url}
%\urlstyle{same}
\urlstyle{tt}
%\usepackage{caption}
\DeclareMathAlphabet{\mathpzc}{OT1}{pzc}{m}{it}
% General %%%%%%%%%%%%%%%%%%%%%%%%%%%%%%%%%%%%%%%%%%%%%%%%%%%%%%%%%%%%%%%%%%%%%%%%%%%%%%%%%%%%%%%%%

% API Table %%%%%%%%%%%%%%%%%%%%%%%%%%%%%%%%%%%%%%%%%%%%%%%%%%%%%%%%%%%%%%%%%%%%%%%%%%%%%%%
\usepackage{longtable}
\usepackage{multirow}

\newenvironment{api2}[1]{%
\begin{tabular}{p{.5cm} p{13.5cm}}%
\hline%
%\multicolumn{2}{|c|}{\code{#1}}\\ \hline %
}%
{\hline%
\end{tabular}%
}
\newcommand{\apidesctxt}[2]{\multicolumn{2}{l}{\lstinline[language=JCop, basicstyle=\listingsfont]{public #1}} \\ & #2 \\}

\newenvironment{api}[1]{%
\begin{longtable}{p{.5cm} p{12.5cm}}%
%\hline%
%\multicolumn{2}{|c|}{\code{#1}}\\ \hline %
}%
{%\hline%
\end{longtable}%
}

\newcommand{\apidesc}[2]{\multicolumn{2}{l}{\lstinline[language=JCop, basicstyle=\listingsfont]{public #1}} \\ & \footnotesize{#2} \\}


% COMMENTS & DRAFT  %%%%%%%%%%%%%%%%%%%%%%%%%%%%%%%%%%%%%%%%%%%%%%%%%%%%%%%%%%%%%%%%%%%%%%%%%%%%%%%

\usepackage{ednotes}
\usepackage{amssymb}
\usepackage{datetime}
\usepackage{everypage}

% date and timeformat
\newdateformat{simpledate}{\THEYEAR-\twodigit{\THEMONTH}-\twodigit{\THEDAY}}
\newtimeformat{simpletime}{\twodigit{\THEHOUR}:\twodigit{\THEMINUTE}}
\newcommand{\timestamp}{\simpledate\today~\simpletime}

% header for draft notice
\newcommand{\draftheader}{%
  %\framebox[\textwidth][t]{%
	  \centerline	
}

% the default definitions of \nbnote and \svnversion are empty	
\newcommand{\nbnote}[2]{}
\newcommand{\svnversion}{}

% the actions perfomed while in comment mode
\newcommand{\commentmode}{
		% \usepackage{draftwatermark} \SetWatermarkLightness{0.70} \SetWatermarkText{\Huge DRAFT}   	
		\renewcommand{\nbnote}[2]{%
		  \fbox{\bfseries\sffamily\scriptsize##1}
		  {\sf\small $\blacktriangleright$ \textit##2 $\blacktriangleleft$}
		  %{\sf\small$\blacktriangleright$\textit{##2}$\blacktriangleleft$}%
		}
	 \renewcommand{\svnversion}{\emph{\footnotesize\svnId}}
	 \AddEverypageHook{\paperref{21cm}{-3mm}{\draftheader}}      
	 \AddEverypageHook{\paperref{21cm}{28.5cm}{\draftheader}}      
}


\newcommand\todo[1]{\nbnote{TO DO}{#1}}

\newcommand{\here}{\nbnote{***}{CONTINUE HERE}}
\newcommand\fix[1]{\nbnote{FIX}{#1}}
\newcommand\jl[1]{\nbnote{JL}{#1}}

\newcommand\rw[1]{\color{blue}#1\color{black}}

% dangerous!
% \DeclareOption{comments}{\commentmode}
% \ProcessOptions 

% LISTINGS  %%%%%%%%%%%%%%%%%%%%%%%%%%%%%%%%%%%%%%%%%%%%%%%%%%%%%%%%%%%%%%%%%%%%%%%%%%%%%%%%%%%%%%%

\usepackage[formats]{listings}
\usepackage[T1]{fontenc}


%\usepackage{pifont}
%\usepackage{winfonts}
%\usepackage{windingbats}
%\usepackage{winpifont}
\usepackage{soul}
%\DisableLigatures{encoding = T1, family = courier }
%\usepackage{beramono}
\usepackage[scaled=.84]{DejaVuSansMono}
%\linespread{1.5}  % mathpazo

%\fontfamily{courier} \selectfont 

\newcommand{\code}[1]{\mbox{\texttt{#1}}} % code font
%\let\code\lstinline
\newcommand{\scode}[1]{\texttt{\small{#1}}} % code font
%\newcommand{\code}[1]{%
%  \lstinline[language=JCop,%
%  					 basicstyle={\ttfamily \footnotesize \selectfont},%
%             keywordstyle={\bfseries \color[rgb]{0,0,0}}]!#1!} % code font
\newcommand{\biglistingsfont}{ \ttfamily \normalsize \selectfont }
\newcommand{\listingsfont}{ \ttfamily \scriptsize \selectfont }
\newcommand{\tinylistingsfont}{\rmfamily \fontsize{6.5}{6.5} \selectfont}
%\newcommand{\listingsfont}{ \fontfamily{courier} \scalebox{.5}[1] \scriptsize \selectfont }
%\newcommand{\tinylistingsfont}{\fontfamily{courier} \fontsize{6.5}{5}  \selectfont \scalebox{.5}[1]}

%\newcommand{\listingsfont}{\ttfamily \scriptsize}

% COLORS
% eclipse keywords: \color[rgb]{0.49,0.00,0.33}
% eclipse strings:  \color[rgb]{0.16,0.2,1}
% eclipse comments: \color[rgb]{0.247,0.498,0.372},


% -- JCop  ----------------------------------------------------------------------------------------

\newcommand{\layerclass}{\code{jcop.lang.Layer}\xspace}
\newcommand{\with}{\code{with}\xspace}
\newcommand{\without}{\code{without}\xspace}
\newcommand{\withoutall}{\code{withoutall}\xspace}
\newcommand{\layer}{\code{layer}\xspace}
\newcommand{\before}{\code{before}\xspace}
\newcommand{\after}{\code{after}\xspace}
\newcommand{\finalmod}{\code{final}\xspace}
\newcommand{\abstractmod}{\code{abstract}\xspace}
\newcommand{\proceed}{\code{proceed}\xspace} 
\newcommand{\ind}{\hspace*{1.5em}}



\lstdefinelanguage{ContextJS} {
      classoffset=0,
      keywords={cop, proceed, withLayer, withoutLayer, with, break,%
                layerClass, layerObject, createLayer, %
        case, catch, continue, else, extends,%
                false, finally, function, for, if, var, instanceof,%
                new, %
                return, super, this, throw,%
                true, try, undefined, while, },
      morecomment=[l]//,%    
      morecomment=[s]{/*}{*/},%
      morestring=[b]",%
      % basicstyle=\listingsfont,
      % basicstyle= \ttfamily \scriptsize \selectfont
      keywordstyle = \color[rgb]{0.54,0.27,0.00},
      %commentstyle = \color[rgb]{0.0,0.533,0.8},
      stringstyle = \color[rgb]{0.0,0.533,0.8},
      identifierstyle=,
      showstringspaces=false,
      captionpos=b,
      tabsize=2,
      aboveskip=5mm,
      %numbersep=-10pt,      
      extendedchars=true,
      frame=tb,
      framexleftmargin=0pt,
      framexrightmargin=-2pt,
      numbers=left,
      numberstyle=\tiny    
}


% -- None ---

\lstdefinelanguage{todo} {
  basicstyle=\color{MidnightBlue}\scriptsize\ttfamily,
  breaklines=true,
  xleftmargin=0.5cm,
  extendedchars=true,
  frame=none
}

\lstdefinelanguage{none} {
  keywords={},
  alsoletter={+},
  morekeywords={}
  frame=none
}

% -----------
  
  % general config
  \lstset{
    framexrightmargin=0pt,      
    numberstyle=\tiny,
    numbersep=5pt,      
    frame=tb,
    basicstyle=\linespread{1} \listingsfont, % print whole listing small
    keywordstyle = \bfseries \color[rgb]{0.49,0.00,0.33},%\color[rgb]{0.84,0.27,0.40},
    commentstyle = \color[rgb]{0.247,0.498,0.372},
    stringstyle = \color[rgb]{0.16,0.2,1},    
    captionpos=b,
    tabsize=2,
    aboveskip=5mm,
    showlines=true,
    showstringspaces=false,
    escapechar=\§,
    breakautoindent=false    
  }
  
  
  \newcommand{\lstnolinenumbers}{
  	\lstset{
  	  numbers=none,   	  
      framexleftmargin=4pt,
      xleftmargin=4pt,   
    }
  }
  	
  
  \newcommand{\lstlinenumbers}{ 
  	\lstset{
  	  numbers=left,
      framexleftmargin=15pt,
      xleftmargin=15pt,            
    }
}



% Paperref ==========================================================================
% includes a published note at the top of the title page for local copies

\usepackage[absolute]{textpos}
%\setlength{\TPHorizModule}{0.8\textwidth}
\definecolor{shadecolor}{rgb}{0.8,0.8,0.8} 

\newcommand{\paperref}[3]{%  
  \setlength{\TPHorizModule}{#1}
  \textblockorigin{0mm}{#2}
  \begin{textblock}{1}(0,0)    
      \begin{shaded}#3\end{shaded}    
  \end{textblock}
}



% \usepackage{verbatim}% http://ctan.org/pkg/verbatim
% \newenvironment{note}{\endgraf\color{MidnightBlue}\scriptsize\verbatim}{\endverbatim}

\newenvironment{rewrite}{\endgraf\color{MidnightBlue}}{}


%\usepackage{fancyvrb, xstring}
%\DefineVerbatimEnvironment{note}{Verbatim}{
%    formatcom=\color{MidnightBlue}\scriptsize
%}

%\renewcommand{\FancyVerbFormatLine}[1]{
%   {\StrSubstitute[1]{#1}{☐}{-}}
%}

% \usepackage{listings}

\lstnewenvironment{note}%
  {%
    %\minipage{\textwidth} 
    \lstset{
      basicstyle=\color{MidnightBlue}\scriptsize\ttfamily,
      breaklines=true,
      xleftmargin=0.5cm,
      extendedchars=true,
      frame=none
    }
  }
  {
    %\endminipage
  }

\usepackage{grfext}
\PrependGraphicsExtensions*{.pdf}


% \usepackage[toc]{glossaries}
% \makeglossaries

\usepackage[nopostdot,nonumberlist,nomain,acronym,xindy%,toc
]{glossaries} % nomain, if you define glossaries in a file, and you use \include{INP-00-glossary}
\makeglossaries
\usepackage[xindy]{imakeidx}
\makeindex

\renewcommand*{\glsnamefont}[1]{\textbf{#1}}



%   \commentmode
%
% - Comments Labeled with Author Id. Define your own author id before \begin{document}:
%   \newcommand\myAuthorID[1]{\nbnote{MAID}{#1}}
%   Use it anywhere in the text: 
%   ... \myAuthorID{this is a comment} ... 
%
% Listings Package Configuration for Several Languages
% ====================================================
% - Syntax highlighting for JCop, ContextJ, LogicAJ, LogicAJ2, GenTL, Prolog
% - Code makro to define code snippets independently of their appeariance. Use:
%   \code{public void foo(){}}
%
% Published Notice on Title Page
% ============================= 
% Paper reference feature to place a publication note on top of the title page of local downloads. 
% Use the following command after \begin{document}: 
% \paperref{<framed box width>}{<height>}{This paper appeared at the n-th XY Conference on Z}
%
% last modified: MA, 2009-08-17_14-47
%%%%%%%%%%%%%%%%%%%%%%%%%%%%%%%%%%%%%%%%%%%%%%%%%%%%%%%%%%%%%%%%%%%%%%%%%%%%%%%%%%%%%%%%%%%%%%%%%%%
\usepackage{amsmath}
\usepackage[usenames,dvipsnames]{color}

\usepackage{framed}
\usepackage{xspace}
\usepackage[hyphens]{url}
%\urlstyle{same}
\urlstyle{tt}
%\usepackage{caption}
\DeclareMathAlphabet{\mathpzc}{OT1}{pzc}{m}{it}
% General %%%%%%%%%%%%%%%%%%%%%%%%%%%%%%%%%%%%%%%%%%%%%%%%%%%%%%%%%%%%%%%%%%%%%%%%%%%%%%%%%%%%%%%%%

% API Table %%%%%%%%%%%%%%%%%%%%%%%%%%%%%%%%%%%%%%%%%%%%%%%%%%%%%%%%%%%%%%%%%%%%%%%%%%%%%%%
\usepackage{longtable}
\usepackage{multirow}

\newenvironment{api2}[1]{%
\begin{tabular}{p{.5cm} p{13.5cm}}%
\hline%
%\multicolumn{2}{|c|}{\code{#1}}\\ \hline %
}%
{\hline%
\end{tabular}%
}
\newcommand{\apidesctxt}[2]{\multicolumn{2}{l}{\lstinline[language=JCop, basicstyle=\listingsfont]{public #1}} \\ & #2 \\}

\newenvironment{api}[1]{%
\begin{longtable}{p{.5cm} p{12.5cm}}%
%\hline%
%\multicolumn{2}{|c|}{\code{#1}}\\ \hline %
}%
{%\hline%
\end{longtable}%
}

\newcommand{\apidesc}[2]{\multicolumn{2}{l}{\lstinline[language=JCop, basicstyle=\listingsfont]{public #1}} \\ & \footnotesize{#2} \\}


% COMMENTS & DRAFT  %%%%%%%%%%%%%%%%%%%%%%%%%%%%%%%%%%%%%%%%%%%%%%%%%%%%%%%%%%%%%%%%%%%%%%%%%%%%%%%

\usepackage{ednotes}
\usepackage{amssymb}
\usepackage{datetime}
\usepackage{everypage}

% date and timeformat
\newdateformat{simpledate}{\THEYEAR-\twodigit{\THEMONTH}-\twodigit{\THEDAY}}
\newtimeformat{simpletime}{\twodigit{\THEHOUR}:\twodigit{\THEMINUTE}}
\newcommand{\timestamp}{\simpledate\today~\simpletime}

% header for draft notice
\newcommand{\draftheader}{%
  %\framebox[\textwidth][t]{%
	  \centerline	
}

% the default definitions of \nbnote and \svnversion are empty	
\newcommand{\nbnote}[2]{}
\newcommand{\svnversion}{}

% the actions perfomed while in comment mode
\newcommand{\commentmode}{
		% \usepackage{draftwatermark} \SetWatermarkLightness{0.70} \SetWatermarkText{\Huge DRAFT}   	
		\renewcommand{\nbnote}[2]{%
		  \fbox{\bfseries\sffamily\scriptsize##1}
		  {\sf\small $\blacktriangleright$ \textit##2 $\blacktriangleleft$}
		  %{\sf\small$\blacktriangleright$\textit{##2}$\blacktriangleleft$}%
		}
	 \renewcommand{\svnversion}{\emph{\footnotesize\svnId}}
	 \AddEverypageHook{\paperref{21cm}{-3mm}{\draftheader}}      
	 \AddEverypageHook{\paperref{21cm}{28.5cm}{\draftheader}}      
}


\newcommand\todo[1]{\nbnote{TO DO}{#1}}

\newcommand{\here}{\nbnote{***}{CONTINUE HERE}}
\newcommand\fix[1]{\nbnote{FIX}{#1}}
\newcommand\jl[1]{\nbnote{JL}{#1}}

\newcommand\rw[1]{\color{blue}#1\color{black}}

% dangerous!
% \DeclareOption{comments}{\commentmode}
% \ProcessOptions 

% LISTINGS  %%%%%%%%%%%%%%%%%%%%%%%%%%%%%%%%%%%%%%%%%%%%%%%%%%%%%%%%%%%%%%%%%%%%%%%%%%%%%%%%%%%%%%%

\usepackage[formats]{listings}
\usepackage[T1]{fontenc}


%\usepackage{pifont}
%\usepackage{winfonts}
%\usepackage{windingbats}
%\usepackage{winpifont}
\usepackage{soul}
%\DisableLigatures{encoding = T1, family = courier }
%\usepackage{beramono}
\usepackage[scaled=.84]{DejaVuSansMono}
%\linespread{1.5}  % mathpazo

%\fontfamily{courier} \selectfont 

\newcommand{\code}[1]{\mbox{\texttt{#1}}} % code font
%\let\code\lstinline
\newcommand{\scode}[1]{\texttt{\small{#1}}} % code font
%\newcommand{\code}[1]{%
%  \lstinline[language=JCop,%
%  					 basicstyle={\ttfamily \footnotesize \selectfont},%
%             keywordstyle={\bfseries \color[rgb]{0,0,0}}]!#1!} % code font
\newcommand{\biglistingsfont}{ \ttfamily \normalsize \selectfont }
\newcommand{\listingsfont}{ \ttfamily \scriptsize \selectfont }
\newcommand{\tinylistingsfont}{\rmfamily \fontsize{6.5}{6.5} \selectfont}
%\newcommand{\listingsfont}{ \fontfamily{courier} \scalebox{.5}[1] \scriptsize \selectfont }
%\newcommand{\tinylistingsfont}{\fontfamily{courier} \fontsize{6.5}{5}  \selectfont \scalebox{.5}[1]}

%\newcommand{\listingsfont}{\ttfamily \scriptsize}

% COLORS
% eclipse keywords: \color[rgb]{0.49,0.00,0.33}
% eclipse strings:  \color[rgb]{0.16,0.2,1}
% eclipse comments: \color[rgb]{0.247,0.498,0.372},


% -- JCop  ----------------------------------------------------------------------------------------

\newcommand{\layerclass}{\code{jcop.lang.Layer}\xspace}
\newcommand{\with}{\code{with}\xspace}
\newcommand{\without}{\code{without}\xspace}
\newcommand{\withoutall}{\code{withoutall}\xspace}
\newcommand{\layer}{\code{layer}\xspace}
\newcommand{\before}{\code{before}\xspace}
\newcommand{\after}{\code{after}\xspace}
\newcommand{\finalmod}{\code{final}\xspace}
\newcommand{\abstractmod}{\code{abstract}\xspace}
\newcommand{\proceed}{\code{proceed}\xspace} 
\newcommand{\ind}{\hspace*{1.5em}}



\lstdefinelanguage{ContextJS} {
      classoffset=0,
      keywords={cop, proceed, withLayer, withoutLayer, with, break,%
                layerClass, layerObject, createLayer, %
        case, catch, continue, else, extends,%
                false, finally, function, for, if, var, instanceof,%
                new, %
                return, super, this, throw,%
                true, try, undefined, while, },
      morecomment=[l]//,%    
      morecomment=[s]{/*}{*/},%
      morestring=[b]",%
      % basicstyle=\listingsfont,
      % basicstyle= \ttfamily \scriptsize \selectfont
      keywordstyle = \color[rgb]{0.54,0.27,0.00},
      %commentstyle = \color[rgb]{0.0,0.533,0.8},
      stringstyle = \color[rgb]{0.0,0.533,0.8},
      identifierstyle=,
      showstringspaces=false,
      captionpos=b,
      tabsize=2,
      aboveskip=5mm,
      %numbersep=-10pt,      
      extendedchars=true,
      frame=tb,
      framexleftmargin=0pt,
      framexrightmargin=-2pt,
      numbers=left,
      numberstyle=\tiny    
}


% -- None ---

\lstdefinelanguage{todo} {
  basicstyle=\color{MidnightBlue}\scriptsize\ttfamily,
  breaklines=true,
  xleftmargin=0.5cm,
  extendedchars=true,
  frame=none
}

\lstdefinelanguage{none} {
  keywords={},
  alsoletter={+},
  morekeywords={}
  frame=none
}

% -----------
  
  % general config
  \lstset{
    framexrightmargin=0pt,      
    numberstyle=\tiny,
    numbersep=5pt,      
    frame=tb,
    basicstyle=\linespread{1} \listingsfont, % print whole listing small
    keywordstyle = \bfseries \color[rgb]{0.49,0.00,0.33},%\color[rgb]{0.84,0.27,0.40},
    commentstyle = \color[rgb]{0.247,0.498,0.372},
    stringstyle = \color[rgb]{0.16,0.2,1},    
    captionpos=b,
    tabsize=2,
    aboveskip=5mm,
    showlines=true,
    showstringspaces=false,
    escapechar=\§,
    breakautoindent=false    
  }
  
  
  \newcommand{\lstnolinenumbers}{
  	\lstset{
  	  numbers=none,   	  
      framexleftmargin=4pt,
      xleftmargin=4pt,   
    }
  }
  	
  
  \newcommand{\lstlinenumbers}{ 
  	\lstset{
  	  numbers=left,
      framexleftmargin=15pt,
      xleftmargin=15pt,            
    }
}



% Paperref ==========================================================================
% includes a published note at the top of the title page for local copies

\usepackage[absolute]{textpos}
%\setlength{\TPHorizModule}{0.8\textwidth}
\definecolor{shadecolor}{rgb}{0.8,0.8,0.8} 

\newcommand{\paperref}[3]{%  
  \setlength{\TPHorizModule}{#1}
  \textblockorigin{0mm}{#2}
  \begin{textblock}{1}(0,0)    
      \begin{shaded}#3\end{shaded}    
  \end{textblock}
}



% \usepackage{verbatim}% http://ctan.org/pkg/verbatim
% \newenvironment{note}{\endgraf\color{MidnightBlue}\scriptsize\verbatim}{\endverbatim}

\newenvironment{rewrite}{\endgraf\color{MidnightBlue}}{}


%\usepackage{fancyvrb, xstring}
%\DefineVerbatimEnvironment{note}{Verbatim}{
%    formatcom=\color{MidnightBlue}\scriptsize
%}

%\renewcommand{\FancyVerbFormatLine}[1]{
%   {\StrSubstitute[1]{#1}{☐}{-}}
%}

% \usepackage{listings}

\lstnewenvironment{note}%
  {%
    %\minipage{\textwidth} 
    \lstset{
      basicstyle=\color{MidnightBlue}\scriptsize\ttfamily,
      breaklines=true,
      xleftmargin=0.5cm,
      extendedchars=true,
      frame=none
    }
  }
  {
    %\endminipage
  }

\usepackage{grfext}
\PrependGraphicsExtensions*{.pdf}


% \usepackage[toc]{glossaries}
% \makeglossaries

\usepackage[nopostdot,nonumberlist,nomain,acronym,xindy%,toc
]{glossaries} % nomain, if you define glossaries in a file, and you use \include{INP-00-glossary}
\makeglossaries
\usepackage[xindy]{imakeidx}
\makeindex

\renewcommand*{\glsnamefont}[1]{\textbf{#1}}




% \usepackage{todonotes}

\usepackage{makeidx} 
%\usepackage{glossaries}
%\usepackage{marginnote}
\usepackage{mparhack}
\usepackage[framed]{ntheorem}
\theoremstyle{plain}

\definecolor{gray}{rgb}{0.6,0.6,0.6}
\renewcommand*\FrameCommand{{\color{gray}\vrule width 5pt \hspace{10pt}}}
\newframedtheorem{definition}{Definition}
\usepackage{bibentry}

%\iftrue
\iffalse % #TODO
\usepackage{eso-pic}
\usepackage{xcolor}
\newsavebox{\draftPageLine}
  % \newsavebox{\draftWatermark}
  \AddToShipoutPicture{%
  %  \AtPageLowerLeft{\usebox{\draftWatermark}}
    \AtPageUpperLeft{%
      \raisebox{-\height}[\height][0pt]{\usebox{\draftPageLine}}}%
    \AtPageLowerLeft{%
      \raisebox{\depth}[\height][0pt]{\usebox{\draftPageLine}}}%
  }

  \def\draftFont{\rmfamily\bfseries}
  \AtBeginDocument{
    \def\draftInfo{%
      {\draftFont
      Draft Draft Draft%
      \hspace*{3cm}\today\hspace*{3cm}%
      Draft Draft Draft%
      }}
      % Draft revision\gitVtags: \gitAbbrevHash{} %
      % (\gitCommitterIsoDate) \gitCommitterName}
    % \sbox{\draftWatermark}{%
    %   \includegraphics[width=\paperwidth]{img/draft}}
    \sbox{\draftPageLine}{%
      \colorbox{black!10}{%
        % enlarge box vertically by 2/3 lines
        \raisebox{0pt}%
        [\dimexpr .33\baselineskip + \height]%
        [\dimexpr .33\baselineskip + \depth]{%
          \makebox[\paperwidth]{\color{black!50}\draftInfo}}}}
  }
\fi

%
%\marginparsepmm
% \newcommand{\note}[1]{}%{\marginpar{\footnotesize{\textit{#1}}}}

\nobibliography*
%\renewcommand{\baselinestretch}{1}
%\parindent 0cm			%Kein Einzug bei neuen Absatz
%\setlength{\parskip}{2ex}
\setcounter{tocdepth}{1}
\setcounter{secnumdepth}{3}

\addtokomafont{disposition}{\rmfamily}
\addtokomafont{paragraph}{\bfseries}
\setkomafont{pagination}{\normalfont\small}

\renewcommand{\chaptermark}[1]{\markboth{#1}{}}
\renewcommand{\sectionmark}[1]{\markright{#1}{}}
\newcommand{\soabooks}{YourBook\xspace}
\newcounter{pbsavefigurecounter}
% 1: pos [tbh]
% 2: width of the left
% 3: width of the right
% 4: code left
% 5: code right
% 6: caption
% 7: label
%
\newcommand{\listingcaption}[2]{
\vspace{-4mm}
\renewcommand{\figurename}{Listing}
\setcounter{pbsavefigurecounter}{\value{figure}}\setcounter{figure}{\value{lstlisting}}
\caption{#1}
\label{#2}
\setcounter{lstlisting}{\value{figure}}
\setcounter{figure}{\value{pbsavefigurecounter}}}


%\usepackage[amsmath,thmmarks,framed]{ntheorem}
%\theoremstyle{plain}
%\theorembodyfont{\small}

%\usepackage{color}
\definecolor{yellow}{rgb}{1,0.88,0}
%\pagecolor{yellow}
%\renewcommand*\FrameCommand{{\color{gray}\vrule width 5pt \hspace{10pt}}}
%\newframedtheorem{defn}{Definition}

%paul
%\linespread{1.8}
\sloppy  % #TODO begin sloppypar begin sloppypar

%\usepackage{showframe}

\usepackage{tgpagella}       % Majuskelziffern
%\usepackage[osf,sc]{mathpazo} % Minuskelziffern
\linespread{1.05}
\let\sffamily\rmfamily

\usepackage{hyperref}
\hypersetup{colorlinks=true, breaklinks=true, citecolor=black, linkcolor=black, menucolor=black, urlcolor=black}
\newcommand{\CtxMS}{CtxMS\xspace}

% \input{00_glossary}
% \loadglsentries{00_glossary}

%% keine Schusterjungen und Witwen

\widowpenalty=10000
\clubpenalty=10000 

\def\sectionautorefname{Section}
\def\subsectionautorefname{Section}
\def\subsubsectionautorefname{Section}
\def\tableautorefname{Table}
\def\figureautorefname{Figure}

\title{I am a Rocket Man}
\author{William Shatner}

\begin{document}
%
% preface.tex
% Common introduction preceeding theses
%
%##noBuild
\addglobaltitlecmark % additional mark in the TOC
%
\chapter*{Preface}
\bibliographystyle{splncs03}
\blindtext

\bigskip
\noindent June 20XX \hfill The Authors
\newpage
\nocite{*}
\bibliography{bibliography}
%%% Local Variables:
%%% mode: latex
%%% End:

%%
%% content.tex
%%
\authorrunning{Author}
\titlerunning{The importance of work}
\title{The importance of why and how to do work}
\subtitle{An imaginary paper to showcase a document}
\extratitle{\raggedleft Author\\ Importance of Work}
\author{Anton Author}
\institute{%
  Hasso-Plattner-Institut, Potsdam\\
  \email{\{firstname.lastname\}@student.hpi.uni-potsdam.de}
}
\supervisors{%
  Prof.\,Dr.\,William Withaname \and%
 Dr.\,John Doe}
\maketitle


% BAMA-O (2009) 21.7
%   (7) Die Bachelorarbeit ist eine Arbeit in deutscher Sprache. Mit Zustimmung
%   der/des Betreuerin/Betreuers kann die Arbeit auch in englischer Sprache
%   abgefasst werden. Erklären beide Gutachter/innen ihr Einverständnis,
%   kann der Prüfungsausschuss auch eine Anfertigung der Arbeit in einer
%   anderen Sprache zulassen. Ist die Arbeit in einer Fremdsprache verfasst,
%   muss sie als Anhang eine kurze Zusammenfassung in deutscher Sprache ent-
%   halten.
% BAMA-O (2012) 26.12
%  (12) Die Bachelorarbeit ist eine Arbeit in deutscher Sprache, sofern die
%  fachspezifische Ordnung keine andere Sprache bestimmt. Mit Zustimmung der
%  Betreuerin bzw. des Betreuers kann die Arbeit auch in englischer Sprache
%  abgefasst werden. Erklären beide Prüfer ihr Einverständnis, kann der
%  Prüfungs- ausschuss auch eine Anfertigung der Arbeit in einer anderen
%  Sprache zulassen. Ist die Arbeit nicht in deutscher Sprache verfasst, muss
%  sie als Anhang eine kurze Zusammenfassung in deutscher Sprache enthalten.

\begin{abstract}
  Hello, here is some text without a meaning. This text should show what a
  printed text will look like at this place. If you read this text, you will
  get no information. Really? Is there no information? Is there a difference
  between this text and some nonsense like “Huardest gefburn”? Kjift – not at
  all! A blind text like this gives you information about the selected font,
  how the letters are written and an impression of the look. This text should
  contain all letters of the alphabet and it should be written in of the
  original language. There is no need for special content, but the length of
  words should match the language.
\end{abstract}

\section{Introduction}
\label{sec:introduction}

Hello, here is some text without a meaning. This text should show what a
printed text will look like at this place. If you read this text, you will get
no information. Really? Is there no information? Is there a difference between
this text and some nonsense like “Huardest gefburn”? Kjift – not at all! A
blind text like this gives you information about the selected font, how the
letters are written and an impression of the look. This text should contain all
letters of the alphabet and it should be written in of the original language.
There is no need for special content, but the length of words should match the
language.\todo{cite}

Hello, here is some text without a meaning. This text should show what a
printed text will look like at this place. If you read this text, you will get
no information. Really? Is there no information? Is there a difference between
this text and some nonsense like “Huardest gefburn”? Kjift – not at all! A
blind text like this gives you information about the selected font, how the
letters are written and an impression of the look. This text should contain all
letters of the alphabet and it should be written in of the original language.
There is no need for special content, but the length of words should match the
language.

\section{Context}
\label{sec:context}

Hello, here is some text without a meaning. This text should show what a
printed text will look like at this place. If you read this text, you will get
no information. Really? Is there no information? Is there a difference between
this text and some nonsense like “Huardest gefburn”? Kjift – not at all! A
blind text like this gives you information about the selected font, how the
letters are written and an impression of the look. This text should contain all
letters of the alphabet and it should be written in of the original language.
There is no need for special content, but the length of words should match the
language.

\todosec{Is that redundant?}

Hello, here is some text without a meaning. This text should show what a
printed text will look like at this place. If you read this text, you will get
no information. Really? Is there no information? Is there a difference between
this text and some nonsense like “Huardest gefburn”? Kjift – not at all! A
blind text like this gives you information about the selected font, how the
letters are written and an impression of the look. This text should contain all
letters of the alphabet and it should be written in of the original language.
There is no need for special content, but the length of words should match the
language.

\section{Problem}
\label{sec:problem}

\secmissing{The Main Problem}

\section{Solution}
\label{sec:solution}

\todolist{anton}{
\item Put A onto B
\item Put B into C
\item Pull D from C
}

\section{Implementation}
\label{sec:implementation}

\missingfigure{Make a overview sketch of the whole system}

\section{Evaluation}
\label{sec:evaluation}

\lstset{language=Smalltalk}
\begin{code}[lst:example]{An example code snippet}{float,numbers=left}
exampleWithNumber: x

"A method that illustrates every part of Smalltalk method syntax
except primitives. It has unary, binary, and keyword messages,
declares arguments and temporaries, accesses a global variable
(but not and instance variable), uses literals (array, character,
symbol, string, integer, float), uses the pseudo variables
true false, nil, self, and super, and has sequence, assignment,
return and cascade. It has both zero argument and one argument blocks."

    |y|
    true & false not & (nil isNil) ifFalse: [self halt].
    y := self size + super size.
    #(#a "a" 1 1.0)
        do: [:each | Transcript show: (each class name);
                                 show: ' '].
     ^ x < y
\end{code}


\section{Related Work}
\label{sec:related-work}

\section{Conclusion}
\label{sec:conclusion}

\ctable%
[table,
caption={Differences between things projected and things achieved}
label=tbl:things]%
{>{}p{.4\linewidth}@{}c}%
{\tnote[a]{Just few things missing}}%
{%
\FL Part           & done
\ML Title          & \y
\NN Abstract       & \n
\NN Intro          & \y
\NN \multicolumn{2}{c}{Rest is not entirely true}
\ML Context        & \y
\NN Problem        & \n\tmark[a]
\NN Solution       & \y
\NN Implementation & \y
\NN Evaluation     & \n
\NN Related Work   & \n
\NN Conclusion     & \y
\LL
}

We have nice things: some code in \autoref{lst:example} and some information
in \autoref{tbl:things}.


\nocite{*}
\bibliography{references}

% BAMA-O (2009) §21.7 
%   Ist die Arbeit in einer Fremdsprache verfasst, muss sie als Anhang eine 
%   kurze Zusammenfassung in deutscher Sprache enthalten.
% BAMA-O (2013) §26.12
%   Ist die Arbeit nicht in deutscher Sprache verfasst, muss sie als Anhang 
%   eine kurze Zusammenfassung in deutscher Sprache enthalten.

\begin{zusammenfassung}
  Weit hinten, hinter den Wortbergen, fern der Länder Vokalien und Konsonantien
  leben die Blindtexte. Abgeschieden wohnen Sie in Buchstabhausen an der Küste
  des Semantik, eines großen Sprachozeans. Ein kleines Bächlein namens Duden
  fließt durch ihren Ort und versorgt sie mit den nötigen Regelialien. Es ist
  ein paradiesmatisches Land, in dem einem gebratene Satzteile in den Mund
  fliegen. Nicht einmal von der allmächtigen Interpunktion werden die
  Blindtexte beherrscht – ein geradezu unorthographisches Leben. Eines Tages
  aber beschloß eine kleine Zeile Blindtext, ihr Name war Lorem Ipsum, hinaus
  zu gehen in die weite Grammatik. Der große Oxmox riet ihr davon ab, da es
  dort wimmele von bösen Kommata, wilden Fragezeichen und hinterhältigen
  Semikoli, doch das Blindtextchen ließ sich nicht beirren. Es packte seine
  sieben Versalien, schob sich sein Initial in den Gürtel und machte sich auf
  den Weg. Als es die ersten Hügel des Kursivgebirges erklommen hatte, warf es
  einen letzten Blick zurück auf die Skyline seiner Heimatstadt Buchstabhausen,
  die Headline von Alphabetdorf und die Subline seiner eigenen Straße, der
  Zeilengasse. Wehmütig lief ihm eine rhetorische Frage über die Wange, dann
  setzte es seinen Weg fort. Unterwegs traf es eine Copy
\end{zusammenfassung}

%%% Local Variables: 
%%% mode: latex
%%% End: 

\chapter*{Publications}

\section*{Journals}

\begin{itemize}
	\item \bibentry{JournalPaper}
\end{itemize}

\section*{Conferences}

\begin{itemize}
	\item \bibentry{ConferencePaper}
\end{itemize}

\section*{Workshops}

\begin{itemize}
	\item \bibentry{WorkshopPaper}
\end{itemize}

\vbox{
\section*{Book Chapters}

\begin{itemize}
	\item \bibentry{BookEntry}
\end{itemize}
}

\section*{Technical Reports}
\begin{itemize}
	\item \bibentry{TechReport}
\end{itemize}

%%% Local Variables:
%%% mode: latex
%%% End:

\bibliographystyle{plain} % alpha
\bibliography{thesis}
\end{document}

%%% Local Variables:
%%% mode: latex
%%% TeX-master: t
%%% End:
