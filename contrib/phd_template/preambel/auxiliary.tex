 %TEX root = main.tex
%%%%%%%%%%%%%%%%%%%%%%%%%%%%%%%%%%%%%%%%%%%%%%%%%%%%%%%%%%%%%%%%%%%%%%%%%%%%%%%%%%%%%%%%%%%%%%%%%%%
%
% This file contains:
%
% Comment Mode Support 
% ==================
% The following features are only active while in comment mode. To enter this mode, simply 
% set \commentmode in the preamble. 
%
% \documentclass{...}
%  %TEX root = main.tex
%%%%%%%%%%%%%%%%%%%%%%%%%%%%%%%%%%%%%%%%%%%%%%%%%%%%%%%%%%%%%%%%%%%%%%%%%%%%%%%%%%%%%%%%%%%%%%%%%%%
%
% This file contains:
%
% Comment Mode Support 
% ==================
% The following features are only active while in comment mode. To enter this mode, simply 
% set \commentmode in the preamble. 
%
% \documentclass{...}
%  %TEX root = main.tex
%%%%%%%%%%%%%%%%%%%%%%%%%%%%%%%%%%%%%%%%%%%%%%%%%%%%%%%%%%%%%%%%%%%%%%%%%%%%%%%%%%%%%%%%%%%%%%%%%%%
%
% This file contains:
%
% Comment Mode Support 
% ==================
% The following features are only active while in comment mode. To enter this mode, simply 
% set \commentmode in the preamble. 
%
% \documentclass{...}
%  %TEX root = main.tex
%%%%%%%%%%%%%%%%%%%%%%%%%%%%%%%%%%%%%%%%%%%%%%%%%%%%%%%%%%%%%%%%%%%%%%%%%%%%%%%%%%%%%%%%%%%%%%%%%%%
%
% This file contains:
%
% Comment Mode Support 
% ==================
% The following features are only active while in comment mode. To enter this mode, simply 
% set \commentmode in the preamble. 
%
% \documentclass{...}
% \input{auxiliary}
% \commentmode
%
% - Draft Notice and Svn Info in a Framed Header on Every Page. This feature is automatically 
%   activated while in comment mode, but needs an additional declaration in the preamble to set 
%   the SVN Id: \def\svnId{$$Id:$$}
% 
%   \documentclass{...}
%   \def\svnId{$$Id:$$}
%   \input{auxiliary}
%   \commentmode
%
% - Comments Labeled with Author Id. Define your own author id before \begin{document}:
%   \newcommand\myAuthorID[1]{\nbnote{MAID}{#1}}
%   Use it anywhere in the text: 
%   ... \myAuthorID{this is a comment} ... 
%
% Listings Package Configuration for Several Languages
% ====================================================
% - Syntax highlighting for JCop, ContextJ, LogicAJ, LogicAJ2, GenTL, Prolog
% - Code makro to define code snippets independently of their appeariance. Use:
%   \code{public void foo(){}}
%
% Published Notice on Title Page
% ============================= 
% Paper reference feature to place a publication note on top of the title page of local downloads. 
% Use the following command after \begin{document}: 
% \paperref{<framed box width>}{<height>}{This paper appeared at the n-th XY Conference on Z}
%
% last modified: MA, 2009-08-17_14-47
%%%%%%%%%%%%%%%%%%%%%%%%%%%%%%%%%%%%%%%%%%%%%%%%%%%%%%%%%%%%%%%%%%%%%%%%%%%%%%%%%%%%%%%%%%%%%%%%%%%
\usepackage{amsmath}
\usepackage[usenames,dvipsnames]{color}

\usepackage{framed}
\usepackage{xspace}
\usepackage[hyphens]{url}
%\urlstyle{same}
\urlstyle{tt}
%\usepackage{caption}
\DeclareMathAlphabet{\mathpzc}{OT1}{pzc}{m}{it}
% General %%%%%%%%%%%%%%%%%%%%%%%%%%%%%%%%%%%%%%%%%%%%%%%%%%%%%%%%%%%%%%%%%%%%%%%%%%%%%%%%%%%%%%%%%

% API Table %%%%%%%%%%%%%%%%%%%%%%%%%%%%%%%%%%%%%%%%%%%%%%%%%%%%%%%%%%%%%%%%%%%%%%%%%%%%%%%
\usepackage{longtable}
\usepackage{multirow}

\newenvironment{api2}[1]{%
\begin{tabular}{p{.5cm} p{13.5cm}}%
\hline%
%\multicolumn{2}{|c|}{\code{#1}}\\ \hline %
}%
{\hline%
\end{tabular}%
}
\newcommand{\apidesctxt}[2]{\multicolumn{2}{l}{\lstinline[language=JCop, basicstyle=\listingsfont]{public #1}} \\ & #2 \\}

\newenvironment{api}[1]{%
\begin{longtable}{p{.5cm} p{12.5cm}}%
%\hline%
%\multicolumn{2}{|c|}{\code{#1}}\\ \hline %
}%
{%\hline%
\end{longtable}%
}

\newcommand{\apidesc}[2]{\multicolumn{2}{l}{\lstinline[language=JCop, basicstyle=\listingsfont]{public #1}} \\ & \footnotesize{#2} \\}


% COMMENTS & DRAFT  %%%%%%%%%%%%%%%%%%%%%%%%%%%%%%%%%%%%%%%%%%%%%%%%%%%%%%%%%%%%%%%%%%%%%%%%%%%%%%%

\usepackage{ednotes}
\usepackage{amssymb}
\usepackage{datetime}
\usepackage{everypage}

% date and timeformat
\newdateformat{simpledate}{\THEYEAR-\twodigit{\THEMONTH}-\twodigit{\THEDAY}}
\newtimeformat{simpletime}{\twodigit{\THEHOUR}:\twodigit{\THEMINUTE}}
\newcommand{\timestamp}{\simpledate\today~\simpletime}

% header for draft notice
\newcommand{\draftheader}{%
  %\framebox[\textwidth][t]{%
	  \centerline	
}

% the default definitions of \nbnote and \svnversion are empty	
\newcommand{\nbnote}[2]{}
\newcommand{\svnversion}{}

% the actions perfomed while in comment mode
\newcommand{\commentmode}{
		% \usepackage{draftwatermark} \SetWatermarkLightness{0.70} \SetWatermarkText{\Huge DRAFT}   	
		\renewcommand{\nbnote}[2]{%
		  \fbox{\bfseries\sffamily\scriptsize##1}
		  {\sf\small $\blacktriangleright$ \textit##2 $\blacktriangleleft$}
		  %{\sf\small$\blacktriangleright$\textit{##2}$\blacktriangleleft$}%
		}
	 \renewcommand{\svnversion}{\emph{\footnotesize\svnId}}
	 \AddEverypageHook{\paperref{21cm}{-3mm}{\draftheader}}      
	 \AddEverypageHook{\paperref{21cm}{28.5cm}{\draftheader}}      
}


\newcommand\todo[1]{\nbnote{TO DO}{#1}}

\newcommand{\here}{\nbnote{***}{CONTINUE HERE}}
\newcommand\fix[1]{\nbnote{FIX}{#1}}
\newcommand\jl[1]{\nbnote{JL}{#1}}

\newcommand\rw[1]{\color{blue}#1\color{black}}

% dangerous!
% \DeclareOption{comments}{\commentmode}
% \ProcessOptions 

% LISTINGS  %%%%%%%%%%%%%%%%%%%%%%%%%%%%%%%%%%%%%%%%%%%%%%%%%%%%%%%%%%%%%%%%%%%%%%%%%%%%%%%%%%%%%%%

\usepackage[formats]{listings}
\usepackage[T1]{fontenc}


%\usepackage{pifont}
%\usepackage{winfonts}
%\usepackage{windingbats}
%\usepackage{winpifont}
\usepackage{soul}
%\DisableLigatures{encoding = T1, family = courier }
%\usepackage{beramono}
\usepackage[scaled=.84]{DejaVuSansMono}
%\linespread{1.5}  % mathpazo

%\fontfamily{courier} \selectfont 

\newcommand{\code}[1]{\mbox{\texttt{#1}}} % code font
%\let\code\lstinline
\newcommand{\scode}[1]{\texttt{\small{#1}}} % code font
%\newcommand{\code}[1]{%
%  \lstinline[language=JCop,%
%  					 basicstyle={\ttfamily \footnotesize \selectfont},%
%             keywordstyle={\bfseries \color[rgb]{0,0,0}}]!#1!} % code font
\newcommand{\biglistingsfont}{ \ttfamily \normalsize \selectfont }
\newcommand{\listingsfont}{ \ttfamily \scriptsize \selectfont }
\newcommand{\tinylistingsfont}{\rmfamily \fontsize{6.5}{6.5} \selectfont}
%\newcommand{\listingsfont}{ \fontfamily{courier} \scalebox{.5}[1] \scriptsize \selectfont }
%\newcommand{\tinylistingsfont}{\fontfamily{courier} \fontsize{6.5}{5}  \selectfont \scalebox{.5}[1]}

%\newcommand{\listingsfont}{\ttfamily \scriptsize}

% COLORS
% eclipse keywords: \color[rgb]{0.49,0.00,0.33}
% eclipse strings:  \color[rgb]{0.16,0.2,1}
% eclipse comments: \color[rgb]{0.247,0.498,0.372},


% -- JCop  ----------------------------------------------------------------------------------------

\newcommand{\layerclass}{\code{jcop.lang.Layer}\xspace}
\newcommand{\with}{\code{with}\xspace}
\newcommand{\without}{\code{without}\xspace}
\newcommand{\withoutall}{\code{withoutall}\xspace}
\newcommand{\layer}{\code{layer}\xspace}
\newcommand{\before}{\code{before}\xspace}
\newcommand{\after}{\code{after}\xspace}
\newcommand{\finalmod}{\code{final}\xspace}
\newcommand{\abstractmod}{\code{abstract}\xspace}
\newcommand{\proceed}{\code{proceed}\xspace} 
\newcommand{\ind}{\hspace*{1.5em}}



\lstdefinelanguage{ContextJS} {
      classoffset=0,
      keywords={cop, proceed, withLayer, withoutLayer, with, break,%
                layerClass, layerObject, createLayer, %
        case, catch, continue, else, extends,%
                false, finally, function, for, if, var, instanceof,%
                new, %
                return, super, this, throw,%
                true, try, undefined, while, },
      morecomment=[l]//,%    
      morecomment=[s]{/*}{*/},%
      morestring=[b]",%
      % basicstyle=\listingsfont,
      % basicstyle= \ttfamily \scriptsize \selectfont
      keywordstyle = \color[rgb]{0.54,0.27,0.00},
      %commentstyle = \color[rgb]{0.0,0.533,0.8},
      stringstyle = \color[rgb]{0.0,0.533,0.8},
      identifierstyle=,
      showstringspaces=false,
      captionpos=b,
      tabsize=2,
      aboveskip=5mm,
      %numbersep=-10pt,      
      extendedchars=true,
      frame=tb,
      framexleftmargin=0pt,
      framexrightmargin=-2pt,
      numbers=left,
      numberstyle=\tiny    
}


% -- None ---

\lstdefinelanguage{todo} {
  basicstyle=\color{MidnightBlue}\scriptsize\ttfamily,
  breaklines=true,
  xleftmargin=0.5cm,
  extendedchars=true,
  frame=none
}

\lstdefinelanguage{none} {
  keywords={},
  alsoletter={+},
  morekeywords={}
  frame=none
}

% -----------
  
  % general config
  \lstset{
    framexrightmargin=0pt,      
    numberstyle=\tiny,
    numbersep=5pt,      
    frame=tb,
    basicstyle=\linespread{1} \listingsfont, % print whole listing small
    keywordstyle = \bfseries \color[rgb]{0.49,0.00,0.33},%\color[rgb]{0.84,0.27,0.40},
    commentstyle = \color[rgb]{0.247,0.498,0.372},
    stringstyle = \color[rgb]{0.16,0.2,1},    
    captionpos=b,
    tabsize=2,
    aboveskip=5mm,
    showlines=true,
    showstringspaces=false,
    escapechar=\§,
    breakautoindent=false    
  }
  
  
  \newcommand{\lstnolinenumbers}{
  	\lstset{
  	  numbers=none,   	  
      framexleftmargin=4pt,
      xleftmargin=4pt,   
    }
  }
  	
  
  \newcommand{\lstlinenumbers}{ 
  	\lstset{
  	  numbers=left,
      framexleftmargin=15pt,
      xleftmargin=15pt,            
    }
}



% Paperref ==========================================================================
% includes a published note at the top of the title page for local copies

\usepackage[absolute]{textpos}
%\setlength{\TPHorizModule}{0.8\textwidth}
\definecolor{shadecolor}{rgb}{0.8,0.8,0.8} 

\newcommand{\paperref}[3]{%  
  \setlength{\TPHorizModule}{#1}
  \textblockorigin{0mm}{#2}
  \begin{textblock}{1}(0,0)    
      \begin{shaded}#3\end{shaded}    
  \end{textblock}
}



% \usepackage{verbatim}% http://ctan.org/pkg/verbatim
% \newenvironment{note}{\endgraf\color{MidnightBlue}\scriptsize\verbatim}{\endverbatim}

\newenvironment{rewrite}{\endgraf\color{MidnightBlue}}{}


%\usepackage{fancyvrb, xstring}
%\DefineVerbatimEnvironment{note}{Verbatim}{
%    formatcom=\color{MidnightBlue}\scriptsize
%}

%\renewcommand{\FancyVerbFormatLine}[1]{
%   {\StrSubstitute[1]{#1}{☐}{-}}
%}

% \usepackage{listings}

\lstnewenvironment{note}%
  {%
    %\minipage{\textwidth} 
    \lstset{
      basicstyle=\color{MidnightBlue}\scriptsize\ttfamily,
      breaklines=true,
      xleftmargin=0.5cm,
      extendedchars=true,
      frame=none
    }
  }
  {
    %\endminipage
  }

\usepackage{grfext}
\PrependGraphicsExtensions*{.pdf}


% \usepackage[toc]{glossaries}
% \makeglossaries

\usepackage[nopostdot,nonumberlist,nomain,acronym,xindy%,toc
]{glossaries} % nomain, if you define glossaries in a file, and you use \include{INP-00-glossary}
\makeglossaries
\usepackage[xindy]{imakeidx}
\makeindex

\renewcommand*{\glsnamefont}[1]{\textbf{#1}}



% \commentmode
%
% - Draft Notice and Svn Info in a Framed Header on Every Page. This feature is automatically 
%   activated while in comment mode, but needs an additional declaration in the preamble to set 
%   the SVN Id: \def\svnId{$$Id:$$}
% 
%   \documentclass{...}
%   \def\svnId{$$Id:$$}
%    %TEX root = main.tex
%%%%%%%%%%%%%%%%%%%%%%%%%%%%%%%%%%%%%%%%%%%%%%%%%%%%%%%%%%%%%%%%%%%%%%%%%%%%%%%%%%%%%%%%%%%%%%%%%%%
%
% This file contains:
%
% Comment Mode Support 
% ==================
% The following features are only active while in comment mode. To enter this mode, simply 
% set \commentmode in the preamble. 
%
% \documentclass{...}
% \input{auxiliary}
% \commentmode
%
% - Draft Notice and Svn Info in a Framed Header on Every Page. This feature is automatically 
%   activated while in comment mode, but needs an additional declaration in the preamble to set 
%   the SVN Id: \def\svnId{$$Id:$$}
% 
%   \documentclass{...}
%   \def\svnId{$$Id:$$}
%   \input{auxiliary}
%   \commentmode
%
% - Comments Labeled with Author Id. Define your own author id before \begin{document}:
%   \newcommand\myAuthorID[1]{\nbnote{MAID}{#1}}
%   Use it anywhere in the text: 
%   ... \myAuthorID{this is a comment} ... 
%
% Listings Package Configuration for Several Languages
% ====================================================
% - Syntax highlighting for JCop, ContextJ, LogicAJ, LogicAJ2, GenTL, Prolog
% - Code makro to define code snippets independently of their appeariance. Use:
%   \code{public void foo(){}}
%
% Published Notice on Title Page
% ============================= 
% Paper reference feature to place a publication note on top of the title page of local downloads. 
% Use the following command after \begin{document}: 
% \paperref{<framed box width>}{<height>}{This paper appeared at the n-th XY Conference on Z}
%
% last modified: MA, 2009-08-17_14-47
%%%%%%%%%%%%%%%%%%%%%%%%%%%%%%%%%%%%%%%%%%%%%%%%%%%%%%%%%%%%%%%%%%%%%%%%%%%%%%%%%%%%%%%%%%%%%%%%%%%
\usepackage{amsmath}
\usepackage[usenames,dvipsnames]{color}

\usepackage{framed}
\usepackage{xspace}
\usepackage[hyphens]{url}
%\urlstyle{same}
\urlstyle{tt}
%\usepackage{caption}
\DeclareMathAlphabet{\mathpzc}{OT1}{pzc}{m}{it}
% General %%%%%%%%%%%%%%%%%%%%%%%%%%%%%%%%%%%%%%%%%%%%%%%%%%%%%%%%%%%%%%%%%%%%%%%%%%%%%%%%%%%%%%%%%

% API Table %%%%%%%%%%%%%%%%%%%%%%%%%%%%%%%%%%%%%%%%%%%%%%%%%%%%%%%%%%%%%%%%%%%%%%%%%%%%%%%
\usepackage{longtable}
\usepackage{multirow}

\newenvironment{api2}[1]{%
\begin{tabular}{p{.5cm} p{13.5cm}}%
\hline%
%\multicolumn{2}{|c|}{\code{#1}}\\ \hline %
}%
{\hline%
\end{tabular}%
}
\newcommand{\apidesctxt}[2]{\multicolumn{2}{l}{\lstinline[language=JCop, basicstyle=\listingsfont]{public #1}} \\ & #2 \\}

\newenvironment{api}[1]{%
\begin{longtable}{p{.5cm} p{12.5cm}}%
%\hline%
%\multicolumn{2}{|c|}{\code{#1}}\\ \hline %
}%
{%\hline%
\end{longtable}%
}

\newcommand{\apidesc}[2]{\multicolumn{2}{l}{\lstinline[language=JCop, basicstyle=\listingsfont]{public #1}} \\ & \footnotesize{#2} \\}


% COMMENTS & DRAFT  %%%%%%%%%%%%%%%%%%%%%%%%%%%%%%%%%%%%%%%%%%%%%%%%%%%%%%%%%%%%%%%%%%%%%%%%%%%%%%%

\usepackage{ednotes}
\usepackage{amssymb}
\usepackage{datetime}
\usepackage{everypage}

% date and timeformat
\newdateformat{simpledate}{\THEYEAR-\twodigit{\THEMONTH}-\twodigit{\THEDAY}}
\newtimeformat{simpletime}{\twodigit{\THEHOUR}:\twodigit{\THEMINUTE}}
\newcommand{\timestamp}{\simpledate\today~\simpletime}

% header for draft notice
\newcommand{\draftheader}{%
  %\framebox[\textwidth][t]{%
	  \centerline	
}

% the default definitions of \nbnote and \svnversion are empty	
\newcommand{\nbnote}[2]{}
\newcommand{\svnversion}{}

% the actions perfomed while in comment mode
\newcommand{\commentmode}{
		% \usepackage{draftwatermark} \SetWatermarkLightness{0.70} \SetWatermarkText{\Huge DRAFT}   	
		\renewcommand{\nbnote}[2]{%
		  \fbox{\bfseries\sffamily\scriptsize##1}
		  {\sf\small $\blacktriangleright$ \textit##2 $\blacktriangleleft$}
		  %{\sf\small$\blacktriangleright$\textit{##2}$\blacktriangleleft$}%
		}
	 \renewcommand{\svnversion}{\emph{\footnotesize\svnId}}
	 \AddEverypageHook{\paperref{21cm}{-3mm}{\draftheader}}      
	 \AddEverypageHook{\paperref{21cm}{28.5cm}{\draftheader}}      
}


\newcommand\todo[1]{\nbnote{TO DO}{#1}}

\newcommand{\here}{\nbnote{***}{CONTINUE HERE}}
\newcommand\fix[1]{\nbnote{FIX}{#1}}
\newcommand\jl[1]{\nbnote{JL}{#1}}

\newcommand\rw[1]{\color{blue}#1\color{black}}

% dangerous!
% \DeclareOption{comments}{\commentmode}
% \ProcessOptions 

% LISTINGS  %%%%%%%%%%%%%%%%%%%%%%%%%%%%%%%%%%%%%%%%%%%%%%%%%%%%%%%%%%%%%%%%%%%%%%%%%%%%%%%%%%%%%%%

\usepackage[formats]{listings}
\usepackage[T1]{fontenc}


%\usepackage{pifont}
%\usepackage{winfonts}
%\usepackage{windingbats}
%\usepackage{winpifont}
\usepackage{soul}
%\DisableLigatures{encoding = T1, family = courier }
%\usepackage{beramono}
\usepackage[scaled=.84]{DejaVuSansMono}
%\linespread{1.5}  % mathpazo

%\fontfamily{courier} \selectfont 

\newcommand{\code}[1]{\mbox{\texttt{#1}}} % code font
%\let\code\lstinline
\newcommand{\scode}[1]{\texttt{\small{#1}}} % code font
%\newcommand{\code}[1]{%
%  \lstinline[language=JCop,%
%  					 basicstyle={\ttfamily \footnotesize \selectfont},%
%             keywordstyle={\bfseries \color[rgb]{0,0,0}}]!#1!} % code font
\newcommand{\biglistingsfont}{ \ttfamily \normalsize \selectfont }
\newcommand{\listingsfont}{ \ttfamily \scriptsize \selectfont }
\newcommand{\tinylistingsfont}{\rmfamily \fontsize{6.5}{6.5} \selectfont}
%\newcommand{\listingsfont}{ \fontfamily{courier} \scalebox{.5}[1] \scriptsize \selectfont }
%\newcommand{\tinylistingsfont}{\fontfamily{courier} \fontsize{6.5}{5}  \selectfont \scalebox{.5}[1]}

%\newcommand{\listingsfont}{\ttfamily \scriptsize}

% COLORS
% eclipse keywords: \color[rgb]{0.49,0.00,0.33}
% eclipse strings:  \color[rgb]{0.16,0.2,1}
% eclipse comments: \color[rgb]{0.247,0.498,0.372},


% -- JCop  ----------------------------------------------------------------------------------------

\newcommand{\layerclass}{\code{jcop.lang.Layer}\xspace}
\newcommand{\with}{\code{with}\xspace}
\newcommand{\without}{\code{without}\xspace}
\newcommand{\withoutall}{\code{withoutall}\xspace}
\newcommand{\layer}{\code{layer}\xspace}
\newcommand{\before}{\code{before}\xspace}
\newcommand{\after}{\code{after}\xspace}
\newcommand{\finalmod}{\code{final}\xspace}
\newcommand{\abstractmod}{\code{abstract}\xspace}
\newcommand{\proceed}{\code{proceed}\xspace} 
\newcommand{\ind}{\hspace*{1.5em}}



\lstdefinelanguage{ContextJS} {
      classoffset=0,
      keywords={cop, proceed, withLayer, withoutLayer, with, break,%
                layerClass, layerObject, createLayer, %
        case, catch, continue, else, extends,%
                false, finally, function, for, if, var, instanceof,%
                new, %
                return, super, this, throw,%
                true, try, undefined, while, },
      morecomment=[l]//,%    
      morecomment=[s]{/*}{*/},%
      morestring=[b]",%
      % basicstyle=\listingsfont,
      % basicstyle= \ttfamily \scriptsize \selectfont
      keywordstyle = \color[rgb]{0.54,0.27,0.00},
      %commentstyle = \color[rgb]{0.0,0.533,0.8},
      stringstyle = \color[rgb]{0.0,0.533,0.8},
      identifierstyle=,
      showstringspaces=false,
      captionpos=b,
      tabsize=2,
      aboveskip=5mm,
      %numbersep=-10pt,      
      extendedchars=true,
      frame=tb,
      framexleftmargin=0pt,
      framexrightmargin=-2pt,
      numbers=left,
      numberstyle=\tiny    
}


% -- None ---

\lstdefinelanguage{todo} {
  basicstyle=\color{MidnightBlue}\scriptsize\ttfamily,
  breaklines=true,
  xleftmargin=0.5cm,
  extendedchars=true,
  frame=none
}

\lstdefinelanguage{none} {
  keywords={},
  alsoletter={+},
  morekeywords={}
  frame=none
}

% -----------
  
  % general config
  \lstset{
    framexrightmargin=0pt,      
    numberstyle=\tiny,
    numbersep=5pt,      
    frame=tb,
    basicstyle=\linespread{1} \listingsfont, % print whole listing small
    keywordstyle = \bfseries \color[rgb]{0.49,0.00,0.33},%\color[rgb]{0.84,0.27,0.40},
    commentstyle = \color[rgb]{0.247,0.498,0.372},
    stringstyle = \color[rgb]{0.16,0.2,1},    
    captionpos=b,
    tabsize=2,
    aboveskip=5mm,
    showlines=true,
    showstringspaces=false,
    escapechar=\§,
    breakautoindent=false    
  }
  
  
  \newcommand{\lstnolinenumbers}{
  	\lstset{
  	  numbers=none,   	  
      framexleftmargin=4pt,
      xleftmargin=4pt,   
    }
  }
  	
  
  \newcommand{\lstlinenumbers}{ 
  	\lstset{
  	  numbers=left,
      framexleftmargin=15pt,
      xleftmargin=15pt,            
    }
}



% Paperref ==========================================================================
% includes a published note at the top of the title page for local copies

\usepackage[absolute]{textpos}
%\setlength{\TPHorizModule}{0.8\textwidth}
\definecolor{shadecolor}{rgb}{0.8,0.8,0.8} 

\newcommand{\paperref}[3]{%  
  \setlength{\TPHorizModule}{#1}
  \textblockorigin{0mm}{#2}
  \begin{textblock}{1}(0,0)    
      \begin{shaded}#3\end{shaded}    
  \end{textblock}
}



% \usepackage{verbatim}% http://ctan.org/pkg/verbatim
% \newenvironment{note}{\endgraf\color{MidnightBlue}\scriptsize\verbatim}{\endverbatim}

\newenvironment{rewrite}{\endgraf\color{MidnightBlue}}{}


%\usepackage{fancyvrb, xstring}
%\DefineVerbatimEnvironment{note}{Verbatim}{
%    formatcom=\color{MidnightBlue}\scriptsize
%}

%\renewcommand{\FancyVerbFormatLine}[1]{
%   {\StrSubstitute[1]{#1}{☐}{-}}
%}

% \usepackage{listings}

\lstnewenvironment{note}%
  {%
    %\minipage{\textwidth} 
    \lstset{
      basicstyle=\color{MidnightBlue}\scriptsize\ttfamily,
      breaklines=true,
      xleftmargin=0.5cm,
      extendedchars=true,
      frame=none
    }
  }
  {
    %\endminipage
  }

\usepackage{grfext}
\PrependGraphicsExtensions*{.pdf}


% \usepackage[toc]{glossaries}
% \makeglossaries

\usepackage[nopostdot,nonumberlist,nomain,acronym,xindy%,toc
]{glossaries} % nomain, if you define glossaries in a file, and you use \include{INP-00-glossary}
\makeglossaries
\usepackage[xindy]{imakeidx}
\makeindex

\renewcommand*{\glsnamefont}[1]{\textbf{#1}}



%   \commentmode
%
% - Comments Labeled with Author Id. Define your own author id before \begin{document}:
%   \newcommand\myAuthorID[1]{\nbnote{MAID}{#1}}
%   Use it anywhere in the text: 
%   ... \myAuthorID{this is a comment} ... 
%
% Listings Package Configuration for Several Languages
% ====================================================
% - Syntax highlighting for JCop, ContextJ, LogicAJ, LogicAJ2, GenTL, Prolog
% - Code makro to define code snippets independently of their appeariance. Use:
%   \code{public void foo(){}}
%
% Published Notice on Title Page
% ============================= 
% Paper reference feature to place a publication note on top of the title page of local downloads. 
% Use the following command after \begin{document}: 
% \paperref{<framed box width>}{<height>}{This paper appeared at the n-th XY Conference on Z}
%
% last modified: MA, 2009-08-17_14-47
%%%%%%%%%%%%%%%%%%%%%%%%%%%%%%%%%%%%%%%%%%%%%%%%%%%%%%%%%%%%%%%%%%%%%%%%%%%%%%%%%%%%%%%%%%%%%%%%%%%
\usepackage{amsmath}
\usepackage[usenames,dvipsnames]{color}

\usepackage{framed}
\usepackage{xspace}
\usepackage[hyphens]{url}
%\urlstyle{same}
\urlstyle{tt}
%\usepackage{caption}
\DeclareMathAlphabet{\mathpzc}{OT1}{pzc}{m}{it}
% General %%%%%%%%%%%%%%%%%%%%%%%%%%%%%%%%%%%%%%%%%%%%%%%%%%%%%%%%%%%%%%%%%%%%%%%%%%%%%%%%%%%%%%%%%

% API Table %%%%%%%%%%%%%%%%%%%%%%%%%%%%%%%%%%%%%%%%%%%%%%%%%%%%%%%%%%%%%%%%%%%%%%%%%%%%%%%
\usepackage{longtable}
\usepackage{multirow}

\newenvironment{api2}[1]{%
\begin{tabular}{p{.5cm} p{13.5cm}}%
\hline%
%\multicolumn{2}{|c|}{\code{#1}}\\ \hline %
}%
{\hline%
\end{tabular}%
}
\newcommand{\apidesctxt}[2]{\multicolumn{2}{l}{\lstinline[language=JCop, basicstyle=\listingsfont]{public #1}} \\ & #2 \\}

\newenvironment{api}[1]{%
\begin{longtable}{p{.5cm} p{12.5cm}}%
%\hline%
%\multicolumn{2}{|c|}{\code{#1}}\\ \hline %
}%
{%\hline%
\end{longtable}%
}

\newcommand{\apidesc}[2]{\multicolumn{2}{l}{\lstinline[language=JCop, basicstyle=\listingsfont]{public #1}} \\ & \footnotesize{#2} \\}


% COMMENTS & DRAFT  %%%%%%%%%%%%%%%%%%%%%%%%%%%%%%%%%%%%%%%%%%%%%%%%%%%%%%%%%%%%%%%%%%%%%%%%%%%%%%%

\usepackage{ednotes}
\usepackage{amssymb}
\usepackage{datetime}
\usepackage{everypage}

% date and timeformat
\newdateformat{simpledate}{\THEYEAR-\twodigit{\THEMONTH}-\twodigit{\THEDAY}}
\newtimeformat{simpletime}{\twodigit{\THEHOUR}:\twodigit{\THEMINUTE}}
\newcommand{\timestamp}{\simpledate\today~\simpletime}

% header for draft notice
\newcommand{\draftheader}{%
  %\framebox[\textwidth][t]{%
	  \centerline	
}

% the default definitions of \nbnote and \svnversion are empty	
\newcommand{\nbnote}[2]{}
\newcommand{\svnversion}{}

% the actions perfomed while in comment mode
\newcommand{\commentmode}{
		% \usepackage{draftwatermark} \SetWatermarkLightness{0.70} \SetWatermarkText{\Huge DRAFT}   	
		\renewcommand{\nbnote}[2]{%
		  \fbox{\bfseries\sffamily\scriptsize##1}
		  {\sf\small $\blacktriangleright$ \textit##2 $\blacktriangleleft$}
		  %{\sf\small$\blacktriangleright$\textit{##2}$\blacktriangleleft$}%
		}
	 \renewcommand{\svnversion}{\emph{\footnotesize\svnId}}
	 \AddEverypageHook{\paperref{21cm}{-3mm}{\draftheader}}      
	 \AddEverypageHook{\paperref{21cm}{28.5cm}{\draftheader}}      
}


\newcommand\todo[1]{\nbnote{TO DO}{#1}}

\newcommand{\here}{\nbnote{***}{CONTINUE HERE}}
\newcommand\fix[1]{\nbnote{FIX}{#1}}
\newcommand\jl[1]{\nbnote{JL}{#1}}

\newcommand\rw[1]{\color{blue}#1\color{black}}

% dangerous!
% \DeclareOption{comments}{\commentmode}
% \ProcessOptions 

% LISTINGS  %%%%%%%%%%%%%%%%%%%%%%%%%%%%%%%%%%%%%%%%%%%%%%%%%%%%%%%%%%%%%%%%%%%%%%%%%%%%%%%%%%%%%%%

\usepackage[formats]{listings}
\usepackage[T1]{fontenc}


%\usepackage{pifont}
%\usepackage{winfonts}
%\usepackage{windingbats}
%\usepackage{winpifont}
\usepackage{soul}
%\DisableLigatures{encoding = T1, family = courier }
%\usepackage{beramono}
\usepackage[scaled=.84]{DejaVuSansMono}
%\linespread{1.5}  % mathpazo

%\fontfamily{courier} \selectfont 

\newcommand{\code}[1]{\mbox{\texttt{#1}}} % code font
%\let\code\lstinline
\newcommand{\scode}[1]{\texttt{\small{#1}}} % code font
%\newcommand{\code}[1]{%
%  \lstinline[language=JCop,%
%  					 basicstyle={\ttfamily \footnotesize \selectfont},%
%             keywordstyle={\bfseries \color[rgb]{0,0,0}}]!#1!} % code font
\newcommand{\biglistingsfont}{ \ttfamily \normalsize \selectfont }
\newcommand{\listingsfont}{ \ttfamily \scriptsize \selectfont }
\newcommand{\tinylistingsfont}{\rmfamily \fontsize{6.5}{6.5} \selectfont}
%\newcommand{\listingsfont}{ \fontfamily{courier} \scalebox{.5}[1] \scriptsize \selectfont }
%\newcommand{\tinylistingsfont}{\fontfamily{courier} \fontsize{6.5}{5}  \selectfont \scalebox{.5}[1]}

%\newcommand{\listingsfont}{\ttfamily \scriptsize}

% COLORS
% eclipse keywords: \color[rgb]{0.49,0.00,0.33}
% eclipse strings:  \color[rgb]{0.16,0.2,1}
% eclipse comments: \color[rgb]{0.247,0.498,0.372},


% -- JCop  ----------------------------------------------------------------------------------------

\newcommand{\layerclass}{\code{jcop.lang.Layer}\xspace}
\newcommand{\with}{\code{with}\xspace}
\newcommand{\without}{\code{without}\xspace}
\newcommand{\withoutall}{\code{withoutall}\xspace}
\newcommand{\layer}{\code{layer}\xspace}
\newcommand{\before}{\code{before}\xspace}
\newcommand{\after}{\code{after}\xspace}
\newcommand{\finalmod}{\code{final}\xspace}
\newcommand{\abstractmod}{\code{abstract}\xspace}
\newcommand{\proceed}{\code{proceed}\xspace} 
\newcommand{\ind}{\hspace*{1.5em}}



\lstdefinelanguage{ContextJS} {
      classoffset=0,
      keywords={cop, proceed, withLayer, withoutLayer, with, break,%
                layerClass, layerObject, createLayer, %
        case, catch, continue, else, extends,%
                false, finally, function, for, if, var, instanceof,%
                new, %
                return, super, this, throw,%
                true, try, undefined, while, },
      morecomment=[l]//,%    
      morecomment=[s]{/*}{*/},%
      morestring=[b]",%
      % basicstyle=\listingsfont,
      % basicstyle= \ttfamily \scriptsize \selectfont
      keywordstyle = \color[rgb]{0.54,0.27,0.00},
      %commentstyle = \color[rgb]{0.0,0.533,0.8},
      stringstyle = \color[rgb]{0.0,0.533,0.8},
      identifierstyle=,
      showstringspaces=false,
      captionpos=b,
      tabsize=2,
      aboveskip=5mm,
      %numbersep=-10pt,      
      extendedchars=true,
      frame=tb,
      framexleftmargin=0pt,
      framexrightmargin=-2pt,
      numbers=left,
      numberstyle=\tiny    
}


% -- None ---

\lstdefinelanguage{todo} {
  basicstyle=\color{MidnightBlue}\scriptsize\ttfamily,
  breaklines=true,
  xleftmargin=0.5cm,
  extendedchars=true,
  frame=none
}

\lstdefinelanguage{none} {
  keywords={},
  alsoletter={+},
  morekeywords={}
  frame=none
}

% -----------
  
  % general config
  \lstset{
    framexrightmargin=0pt,      
    numberstyle=\tiny,
    numbersep=5pt,      
    frame=tb,
    basicstyle=\linespread{1} \listingsfont, % print whole listing small
    keywordstyle = \bfseries \color[rgb]{0.49,0.00,0.33},%\color[rgb]{0.84,0.27,0.40},
    commentstyle = \color[rgb]{0.247,0.498,0.372},
    stringstyle = \color[rgb]{0.16,0.2,1},    
    captionpos=b,
    tabsize=2,
    aboveskip=5mm,
    showlines=true,
    showstringspaces=false,
    escapechar=\§,
    breakautoindent=false    
  }
  
  
  \newcommand{\lstnolinenumbers}{
  	\lstset{
  	  numbers=none,   	  
      framexleftmargin=4pt,
      xleftmargin=4pt,   
    }
  }
  	
  
  \newcommand{\lstlinenumbers}{ 
  	\lstset{
  	  numbers=left,
      framexleftmargin=15pt,
      xleftmargin=15pt,            
    }
}



% Paperref ==========================================================================
% includes a published note at the top of the title page for local copies

\usepackage[absolute]{textpos}
%\setlength{\TPHorizModule}{0.8\textwidth}
\definecolor{shadecolor}{rgb}{0.8,0.8,0.8} 

\newcommand{\paperref}[3]{%  
  \setlength{\TPHorizModule}{#1}
  \textblockorigin{0mm}{#2}
  \begin{textblock}{1}(0,0)    
      \begin{shaded}#3\end{shaded}    
  \end{textblock}
}



% \usepackage{verbatim}% http://ctan.org/pkg/verbatim
% \newenvironment{note}{\endgraf\color{MidnightBlue}\scriptsize\verbatim}{\endverbatim}

\newenvironment{rewrite}{\endgraf\color{MidnightBlue}}{}


%\usepackage{fancyvrb, xstring}
%\DefineVerbatimEnvironment{note}{Verbatim}{
%    formatcom=\color{MidnightBlue}\scriptsize
%}

%\renewcommand{\FancyVerbFormatLine}[1]{
%   {\StrSubstitute[1]{#1}{☐}{-}}
%}

% \usepackage{listings}

\lstnewenvironment{note}%
  {%
    %\minipage{\textwidth} 
    \lstset{
      basicstyle=\color{MidnightBlue}\scriptsize\ttfamily,
      breaklines=true,
      xleftmargin=0.5cm,
      extendedchars=true,
      frame=none
    }
  }
  {
    %\endminipage
  }

\usepackage{grfext}
\PrependGraphicsExtensions*{.pdf}


% \usepackage[toc]{glossaries}
% \makeglossaries

\usepackage[nopostdot,nonumberlist,nomain,acronym,xindy%,toc
]{glossaries} % nomain, if you define glossaries in a file, and you use \include{INP-00-glossary}
\makeglossaries
\usepackage[xindy]{imakeidx}
\makeindex

\renewcommand*{\glsnamefont}[1]{\textbf{#1}}



% \commentmode
%
% - Draft Notice and Svn Info in a Framed Header on Every Page. This feature is automatically 
%   activated while in comment mode, but needs an additional declaration in the preamble to set 
%   the SVN Id: \def\svnId{$$Id:$$}
% 
%   \documentclass{...}
%   \def\svnId{$$Id:$$}
%    %TEX root = main.tex
%%%%%%%%%%%%%%%%%%%%%%%%%%%%%%%%%%%%%%%%%%%%%%%%%%%%%%%%%%%%%%%%%%%%%%%%%%%%%%%%%%%%%%%%%%%%%%%%%%%
%
% This file contains:
%
% Comment Mode Support 
% ==================
% The following features are only active while in comment mode. To enter this mode, simply 
% set \commentmode in the preamble. 
%
% \documentclass{...}
%  %TEX root = main.tex
%%%%%%%%%%%%%%%%%%%%%%%%%%%%%%%%%%%%%%%%%%%%%%%%%%%%%%%%%%%%%%%%%%%%%%%%%%%%%%%%%%%%%%%%%%%%%%%%%%%
%
% This file contains:
%
% Comment Mode Support 
% ==================
% The following features are only active while in comment mode. To enter this mode, simply 
% set \commentmode in the preamble. 
%
% \documentclass{...}
% \input{auxiliary}
% \commentmode
%
% - Draft Notice and Svn Info in a Framed Header on Every Page. This feature is automatically 
%   activated while in comment mode, but needs an additional declaration in the preamble to set 
%   the SVN Id: \def\svnId{$$Id:$$}
% 
%   \documentclass{...}
%   \def\svnId{$$Id:$$}
%   \input{auxiliary}
%   \commentmode
%
% - Comments Labeled with Author Id. Define your own author id before \begin{document}:
%   \newcommand\myAuthorID[1]{\nbnote{MAID}{#1}}
%   Use it anywhere in the text: 
%   ... \myAuthorID{this is a comment} ... 
%
% Listings Package Configuration for Several Languages
% ====================================================
% - Syntax highlighting for JCop, ContextJ, LogicAJ, LogicAJ2, GenTL, Prolog
% - Code makro to define code snippets independently of their appeariance. Use:
%   \code{public void foo(){}}
%
% Published Notice on Title Page
% ============================= 
% Paper reference feature to place a publication note on top of the title page of local downloads. 
% Use the following command after \begin{document}: 
% \paperref{<framed box width>}{<height>}{This paper appeared at the n-th XY Conference on Z}
%
% last modified: MA, 2009-08-17_14-47
%%%%%%%%%%%%%%%%%%%%%%%%%%%%%%%%%%%%%%%%%%%%%%%%%%%%%%%%%%%%%%%%%%%%%%%%%%%%%%%%%%%%%%%%%%%%%%%%%%%
\usepackage{amsmath}
\usepackage[usenames,dvipsnames]{color}

\usepackage{framed}
\usepackage{xspace}
\usepackage[hyphens]{url}
%\urlstyle{same}
\urlstyle{tt}
%\usepackage{caption}
\DeclareMathAlphabet{\mathpzc}{OT1}{pzc}{m}{it}
% General %%%%%%%%%%%%%%%%%%%%%%%%%%%%%%%%%%%%%%%%%%%%%%%%%%%%%%%%%%%%%%%%%%%%%%%%%%%%%%%%%%%%%%%%%

% API Table %%%%%%%%%%%%%%%%%%%%%%%%%%%%%%%%%%%%%%%%%%%%%%%%%%%%%%%%%%%%%%%%%%%%%%%%%%%%%%%
\usepackage{longtable}
\usepackage{multirow}

\newenvironment{api2}[1]{%
\begin{tabular}{p{.5cm} p{13.5cm}}%
\hline%
%\multicolumn{2}{|c|}{\code{#1}}\\ \hline %
}%
{\hline%
\end{tabular}%
}
\newcommand{\apidesctxt}[2]{\multicolumn{2}{l}{\lstinline[language=JCop, basicstyle=\listingsfont]{public #1}} \\ & #2 \\}

\newenvironment{api}[1]{%
\begin{longtable}{p{.5cm} p{12.5cm}}%
%\hline%
%\multicolumn{2}{|c|}{\code{#1}}\\ \hline %
}%
{%\hline%
\end{longtable}%
}

\newcommand{\apidesc}[2]{\multicolumn{2}{l}{\lstinline[language=JCop, basicstyle=\listingsfont]{public #1}} \\ & \footnotesize{#2} \\}


% COMMENTS & DRAFT  %%%%%%%%%%%%%%%%%%%%%%%%%%%%%%%%%%%%%%%%%%%%%%%%%%%%%%%%%%%%%%%%%%%%%%%%%%%%%%%

\usepackage{ednotes}
\usepackage{amssymb}
\usepackage{datetime}
\usepackage{everypage}

% date and timeformat
\newdateformat{simpledate}{\THEYEAR-\twodigit{\THEMONTH}-\twodigit{\THEDAY}}
\newtimeformat{simpletime}{\twodigit{\THEHOUR}:\twodigit{\THEMINUTE}}
\newcommand{\timestamp}{\simpledate\today~\simpletime}

% header for draft notice
\newcommand{\draftheader}{%
  %\framebox[\textwidth][t]{%
	  \centerline	
}

% the default definitions of \nbnote and \svnversion are empty	
\newcommand{\nbnote}[2]{}
\newcommand{\svnversion}{}

% the actions perfomed while in comment mode
\newcommand{\commentmode}{
		% \usepackage{draftwatermark} \SetWatermarkLightness{0.70} \SetWatermarkText{\Huge DRAFT}   	
		\renewcommand{\nbnote}[2]{%
		  \fbox{\bfseries\sffamily\scriptsize##1}
		  {\sf\small $\blacktriangleright$ \textit##2 $\blacktriangleleft$}
		  %{\sf\small$\blacktriangleright$\textit{##2}$\blacktriangleleft$}%
		}
	 \renewcommand{\svnversion}{\emph{\footnotesize\svnId}}
	 \AddEverypageHook{\paperref{21cm}{-3mm}{\draftheader}}      
	 \AddEverypageHook{\paperref{21cm}{28.5cm}{\draftheader}}      
}


\newcommand\todo[1]{\nbnote{TO DO}{#1}}

\newcommand{\here}{\nbnote{***}{CONTINUE HERE}}
\newcommand\fix[1]{\nbnote{FIX}{#1}}
\newcommand\jl[1]{\nbnote{JL}{#1}}

\newcommand\rw[1]{\color{blue}#1\color{black}}

% dangerous!
% \DeclareOption{comments}{\commentmode}
% \ProcessOptions 

% LISTINGS  %%%%%%%%%%%%%%%%%%%%%%%%%%%%%%%%%%%%%%%%%%%%%%%%%%%%%%%%%%%%%%%%%%%%%%%%%%%%%%%%%%%%%%%

\usepackage[formats]{listings}
\usepackage[T1]{fontenc}


%\usepackage{pifont}
%\usepackage{winfonts}
%\usepackage{windingbats}
%\usepackage{winpifont}
\usepackage{soul}
%\DisableLigatures{encoding = T1, family = courier }
%\usepackage{beramono}
\usepackage[scaled=.84]{DejaVuSansMono}
%\linespread{1.5}  % mathpazo

%\fontfamily{courier} \selectfont 

\newcommand{\code}[1]{\mbox{\texttt{#1}}} % code font
%\let\code\lstinline
\newcommand{\scode}[1]{\texttt{\small{#1}}} % code font
%\newcommand{\code}[1]{%
%  \lstinline[language=JCop,%
%  					 basicstyle={\ttfamily \footnotesize \selectfont},%
%             keywordstyle={\bfseries \color[rgb]{0,0,0}}]!#1!} % code font
\newcommand{\biglistingsfont}{ \ttfamily \normalsize \selectfont }
\newcommand{\listingsfont}{ \ttfamily \scriptsize \selectfont }
\newcommand{\tinylistingsfont}{\rmfamily \fontsize{6.5}{6.5} \selectfont}
%\newcommand{\listingsfont}{ \fontfamily{courier} \scalebox{.5}[1] \scriptsize \selectfont }
%\newcommand{\tinylistingsfont}{\fontfamily{courier} \fontsize{6.5}{5}  \selectfont \scalebox{.5}[1]}

%\newcommand{\listingsfont}{\ttfamily \scriptsize}

% COLORS
% eclipse keywords: \color[rgb]{0.49,0.00,0.33}
% eclipse strings:  \color[rgb]{0.16,0.2,1}
% eclipse comments: \color[rgb]{0.247,0.498,0.372},


% -- JCop  ----------------------------------------------------------------------------------------

\newcommand{\layerclass}{\code{jcop.lang.Layer}\xspace}
\newcommand{\with}{\code{with}\xspace}
\newcommand{\without}{\code{without}\xspace}
\newcommand{\withoutall}{\code{withoutall}\xspace}
\newcommand{\layer}{\code{layer}\xspace}
\newcommand{\before}{\code{before}\xspace}
\newcommand{\after}{\code{after}\xspace}
\newcommand{\finalmod}{\code{final}\xspace}
\newcommand{\abstractmod}{\code{abstract}\xspace}
\newcommand{\proceed}{\code{proceed}\xspace} 
\newcommand{\ind}{\hspace*{1.5em}}



\lstdefinelanguage{ContextJS} {
      classoffset=0,
      keywords={cop, proceed, withLayer, withoutLayer, with, break,%
                layerClass, layerObject, createLayer, %
        case, catch, continue, else, extends,%
                false, finally, function, for, if, var, instanceof,%
                new, %
                return, super, this, throw,%
                true, try, undefined, while, },
      morecomment=[l]//,%    
      morecomment=[s]{/*}{*/},%
      morestring=[b]",%
      % basicstyle=\listingsfont,
      % basicstyle= \ttfamily \scriptsize \selectfont
      keywordstyle = \color[rgb]{0.54,0.27,0.00},
      %commentstyle = \color[rgb]{0.0,0.533,0.8},
      stringstyle = \color[rgb]{0.0,0.533,0.8},
      identifierstyle=,
      showstringspaces=false,
      captionpos=b,
      tabsize=2,
      aboveskip=5mm,
      %numbersep=-10pt,      
      extendedchars=true,
      frame=tb,
      framexleftmargin=0pt,
      framexrightmargin=-2pt,
      numbers=left,
      numberstyle=\tiny    
}


% -- None ---

\lstdefinelanguage{todo} {
  basicstyle=\color{MidnightBlue}\scriptsize\ttfamily,
  breaklines=true,
  xleftmargin=0.5cm,
  extendedchars=true,
  frame=none
}

\lstdefinelanguage{none} {
  keywords={},
  alsoletter={+},
  morekeywords={}
  frame=none
}

% -----------
  
  % general config
  \lstset{
    framexrightmargin=0pt,      
    numberstyle=\tiny,
    numbersep=5pt,      
    frame=tb,
    basicstyle=\linespread{1} \listingsfont, % print whole listing small
    keywordstyle = \bfseries \color[rgb]{0.49,0.00,0.33},%\color[rgb]{0.84,0.27,0.40},
    commentstyle = \color[rgb]{0.247,0.498,0.372},
    stringstyle = \color[rgb]{0.16,0.2,1},    
    captionpos=b,
    tabsize=2,
    aboveskip=5mm,
    showlines=true,
    showstringspaces=false,
    escapechar=\§,
    breakautoindent=false    
  }
  
  
  \newcommand{\lstnolinenumbers}{
  	\lstset{
  	  numbers=none,   	  
      framexleftmargin=4pt,
      xleftmargin=4pt,   
    }
  }
  	
  
  \newcommand{\lstlinenumbers}{ 
  	\lstset{
  	  numbers=left,
      framexleftmargin=15pt,
      xleftmargin=15pt,            
    }
}



% Paperref ==========================================================================
% includes a published note at the top of the title page for local copies

\usepackage[absolute]{textpos}
%\setlength{\TPHorizModule}{0.8\textwidth}
\definecolor{shadecolor}{rgb}{0.8,0.8,0.8} 

\newcommand{\paperref}[3]{%  
  \setlength{\TPHorizModule}{#1}
  \textblockorigin{0mm}{#2}
  \begin{textblock}{1}(0,0)    
      \begin{shaded}#3\end{shaded}    
  \end{textblock}
}



% \usepackage{verbatim}% http://ctan.org/pkg/verbatim
% \newenvironment{note}{\endgraf\color{MidnightBlue}\scriptsize\verbatim}{\endverbatim}

\newenvironment{rewrite}{\endgraf\color{MidnightBlue}}{}


%\usepackage{fancyvrb, xstring}
%\DefineVerbatimEnvironment{note}{Verbatim}{
%    formatcom=\color{MidnightBlue}\scriptsize
%}

%\renewcommand{\FancyVerbFormatLine}[1]{
%   {\StrSubstitute[1]{#1}{☐}{-}}
%}

% \usepackage{listings}

\lstnewenvironment{note}%
  {%
    %\minipage{\textwidth} 
    \lstset{
      basicstyle=\color{MidnightBlue}\scriptsize\ttfamily,
      breaklines=true,
      xleftmargin=0.5cm,
      extendedchars=true,
      frame=none
    }
  }
  {
    %\endminipage
  }

\usepackage{grfext}
\PrependGraphicsExtensions*{.pdf}


% \usepackage[toc]{glossaries}
% \makeglossaries

\usepackage[nopostdot,nonumberlist,nomain,acronym,xindy%,toc
]{glossaries} % nomain, if you define glossaries in a file, and you use \include{INP-00-glossary}
\makeglossaries
\usepackage[xindy]{imakeidx}
\makeindex

\renewcommand*{\glsnamefont}[1]{\textbf{#1}}



% \commentmode
%
% - Draft Notice and Svn Info in a Framed Header on Every Page. This feature is automatically 
%   activated while in comment mode, but needs an additional declaration in the preamble to set 
%   the SVN Id: \def\svnId{$$Id:$$}
% 
%   \documentclass{...}
%   \def\svnId{$$Id:$$}
%    %TEX root = main.tex
%%%%%%%%%%%%%%%%%%%%%%%%%%%%%%%%%%%%%%%%%%%%%%%%%%%%%%%%%%%%%%%%%%%%%%%%%%%%%%%%%%%%%%%%%%%%%%%%%%%
%
% This file contains:
%
% Comment Mode Support 
% ==================
% The following features are only active while in comment mode. To enter this mode, simply 
% set \commentmode in the preamble. 
%
% \documentclass{...}
% \input{auxiliary}
% \commentmode
%
% - Draft Notice and Svn Info in a Framed Header on Every Page. This feature is automatically 
%   activated while in comment mode, but needs an additional declaration in the preamble to set 
%   the SVN Id: \def\svnId{$$Id:$$}
% 
%   \documentclass{...}
%   \def\svnId{$$Id:$$}
%   \input{auxiliary}
%   \commentmode
%
% - Comments Labeled with Author Id. Define your own author id before \begin{document}:
%   \newcommand\myAuthorID[1]{\nbnote{MAID}{#1}}
%   Use it anywhere in the text: 
%   ... \myAuthorID{this is a comment} ... 
%
% Listings Package Configuration for Several Languages
% ====================================================
% - Syntax highlighting for JCop, ContextJ, LogicAJ, LogicAJ2, GenTL, Prolog
% - Code makro to define code snippets independently of their appeariance. Use:
%   \code{public void foo(){}}
%
% Published Notice on Title Page
% ============================= 
% Paper reference feature to place a publication note on top of the title page of local downloads. 
% Use the following command after \begin{document}: 
% \paperref{<framed box width>}{<height>}{This paper appeared at the n-th XY Conference on Z}
%
% last modified: MA, 2009-08-17_14-47
%%%%%%%%%%%%%%%%%%%%%%%%%%%%%%%%%%%%%%%%%%%%%%%%%%%%%%%%%%%%%%%%%%%%%%%%%%%%%%%%%%%%%%%%%%%%%%%%%%%
\usepackage{amsmath}
\usepackage[usenames,dvipsnames]{color}

\usepackage{framed}
\usepackage{xspace}
\usepackage[hyphens]{url}
%\urlstyle{same}
\urlstyle{tt}
%\usepackage{caption}
\DeclareMathAlphabet{\mathpzc}{OT1}{pzc}{m}{it}
% General %%%%%%%%%%%%%%%%%%%%%%%%%%%%%%%%%%%%%%%%%%%%%%%%%%%%%%%%%%%%%%%%%%%%%%%%%%%%%%%%%%%%%%%%%

% API Table %%%%%%%%%%%%%%%%%%%%%%%%%%%%%%%%%%%%%%%%%%%%%%%%%%%%%%%%%%%%%%%%%%%%%%%%%%%%%%%
\usepackage{longtable}
\usepackage{multirow}

\newenvironment{api2}[1]{%
\begin{tabular}{p{.5cm} p{13.5cm}}%
\hline%
%\multicolumn{2}{|c|}{\code{#1}}\\ \hline %
}%
{\hline%
\end{tabular}%
}
\newcommand{\apidesctxt}[2]{\multicolumn{2}{l}{\lstinline[language=JCop, basicstyle=\listingsfont]{public #1}} \\ & #2 \\}

\newenvironment{api}[1]{%
\begin{longtable}{p{.5cm} p{12.5cm}}%
%\hline%
%\multicolumn{2}{|c|}{\code{#1}}\\ \hline %
}%
{%\hline%
\end{longtable}%
}

\newcommand{\apidesc}[2]{\multicolumn{2}{l}{\lstinline[language=JCop, basicstyle=\listingsfont]{public #1}} \\ & \footnotesize{#2} \\}


% COMMENTS & DRAFT  %%%%%%%%%%%%%%%%%%%%%%%%%%%%%%%%%%%%%%%%%%%%%%%%%%%%%%%%%%%%%%%%%%%%%%%%%%%%%%%

\usepackage{ednotes}
\usepackage{amssymb}
\usepackage{datetime}
\usepackage{everypage}

% date and timeformat
\newdateformat{simpledate}{\THEYEAR-\twodigit{\THEMONTH}-\twodigit{\THEDAY}}
\newtimeformat{simpletime}{\twodigit{\THEHOUR}:\twodigit{\THEMINUTE}}
\newcommand{\timestamp}{\simpledate\today~\simpletime}

% header for draft notice
\newcommand{\draftheader}{%
  %\framebox[\textwidth][t]{%
	  \centerline	
}

% the default definitions of \nbnote and \svnversion are empty	
\newcommand{\nbnote}[2]{}
\newcommand{\svnversion}{}

% the actions perfomed while in comment mode
\newcommand{\commentmode}{
		% \usepackage{draftwatermark} \SetWatermarkLightness{0.70} \SetWatermarkText{\Huge DRAFT}   	
		\renewcommand{\nbnote}[2]{%
		  \fbox{\bfseries\sffamily\scriptsize##1}
		  {\sf\small $\blacktriangleright$ \textit##2 $\blacktriangleleft$}
		  %{\sf\small$\blacktriangleright$\textit{##2}$\blacktriangleleft$}%
		}
	 \renewcommand{\svnversion}{\emph{\footnotesize\svnId}}
	 \AddEverypageHook{\paperref{21cm}{-3mm}{\draftheader}}      
	 \AddEverypageHook{\paperref{21cm}{28.5cm}{\draftheader}}      
}


\newcommand\todo[1]{\nbnote{TO DO}{#1}}

\newcommand{\here}{\nbnote{***}{CONTINUE HERE}}
\newcommand\fix[1]{\nbnote{FIX}{#1}}
\newcommand\jl[1]{\nbnote{JL}{#1}}

\newcommand\rw[1]{\color{blue}#1\color{black}}

% dangerous!
% \DeclareOption{comments}{\commentmode}
% \ProcessOptions 

% LISTINGS  %%%%%%%%%%%%%%%%%%%%%%%%%%%%%%%%%%%%%%%%%%%%%%%%%%%%%%%%%%%%%%%%%%%%%%%%%%%%%%%%%%%%%%%

\usepackage[formats]{listings}
\usepackage[T1]{fontenc}


%\usepackage{pifont}
%\usepackage{winfonts}
%\usepackage{windingbats}
%\usepackage{winpifont}
\usepackage{soul}
%\DisableLigatures{encoding = T1, family = courier }
%\usepackage{beramono}
\usepackage[scaled=.84]{DejaVuSansMono}
%\linespread{1.5}  % mathpazo

%\fontfamily{courier} \selectfont 

\newcommand{\code}[1]{\mbox{\texttt{#1}}} % code font
%\let\code\lstinline
\newcommand{\scode}[1]{\texttt{\small{#1}}} % code font
%\newcommand{\code}[1]{%
%  \lstinline[language=JCop,%
%  					 basicstyle={\ttfamily \footnotesize \selectfont},%
%             keywordstyle={\bfseries \color[rgb]{0,0,0}}]!#1!} % code font
\newcommand{\biglistingsfont}{ \ttfamily \normalsize \selectfont }
\newcommand{\listingsfont}{ \ttfamily \scriptsize \selectfont }
\newcommand{\tinylistingsfont}{\rmfamily \fontsize{6.5}{6.5} \selectfont}
%\newcommand{\listingsfont}{ \fontfamily{courier} \scalebox{.5}[1] \scriptsize \selectfont }
%\newcommand{\tinylistingsfont}{\fontfamily{courier} \fontsize{6.5}{5}  \selectfont \scalebox{.5}[1]}

%\newcommand{\listingsfont}{\ttfamily \scriptsize}

% COLORS
% eclipse keywords: \color[rgb]{0.49,0.00,0.33}
% eclipse strings:  \color[rgb]{0.16,0.2,1}
% eclipse comments: \color[rgb]{0.247,0.498,0.372},


% -- JCop  ----------------------------------------------------------------------------------------

\newcommand{\layerclass}{\code{jcop.lang.Layer}\xspace}
\newcommand{\with}{\code{with}\xspace}
\newcommand{\without}{\code{without}\xspace}
\newcommand{\withoutall}{\code{withoutall}\xspace}
\newcommand{\layer}{\code{layer}\xspace}
\newcommand{\before}{\code{before}\xspace}
\newcommand{\after}{\code{after}\xspace}
\newcommand{\finalmod}{\code{final}\xspace}
\newcommand{\abstractmod}{\code{abstract}\xspace}
\newcommand{\proceed}{\code{proceed}\xspace} 
\newcommand{\ind}{\hspace*{1.5em}}



\lstdefinelanguage{ContextJS} {
      classoffset=0,
      keywords={cop, proceed, withLayer, withoutLayer, with, break,%
                layerClass, layerObject, createLayer, %
        case, catch, continue, else, extends,%
                false, finally, function, for, if, var, instanceof,%
                new, %
                return, super, this, throw,%
                true, try, undefined, while, },
      morecomment=[l]//,%    
      morecomment=[s]{/*}{*/},%
      morestring=[b]",%
      % basicstyle=\listingsfont,
      % basicstyle= \ttfamily \scriptsize \selectfont
      keywordstyle = \color[rgb]{0.54,0.27,0.00},
      %commentstyle = \color[rgb]{0.0,0.533,0.8},
      stringstyle = \color[rgb]{0.0,0.533,0.8},
      identifierstyle=,
      showstringspaces=false,
      captionpos=b,
      tabsize=2,
      aboveskip=5mm,
      %numbersep=-10pt,      
      extendedchars=true,
      frame=tb,
      framexleftmargin=0pt,
      framexrightmargin=-2pt,
      numbers=left,
      numberstyle=\tiny    
}


% -- None ---

\lstdefinelanguage{todo} {
  basicstyle=\color{MidnightBlue}\scriptsize\ttfamily,
  breaklines=true,
  xleftmargin=0.5cm,
  extendedchars=true,
  frame=none
}

\lstdefinelanguage{none} {
  keywords={},
  alsoletter={+},
  morekeywords={}
  frame=none
}

% -----------
  
  % general config
  \lstset{
    framexrightmargin=0pt,      
    numberstyle=\tiny,
    numbersep=5pt,      
    frame=tb,
    basicstyle=\linespread{1} \listingsfont, % print whole listing small
    keywordstyle = \bfseries \color[rgb]{0.49,0.00,0.33},%\color[rgb]{0.84,0.27,0.40},
    commentstyle = \color[rgb]{0.247,0.498,0.372},
    stringstyle = \color[rgb]{0.16,0.2,1},    
    captionpos=b,
    tabsize=2,
    aboveskip=5mm,
    showlines=true,
    showstringspaces=false,
    escapechar=\§,
    breakautoindent=false    
  }
  
  
  \newcommand{\lstnolinenumbers}{
  	\lstset{
  	  numbers=none,   	  
      framexleftmargin=4pt,
      xleftmargin=4pt,   
    }
  }
  	
  
  \newcommand{\lstlinenumbers}{ 
  	\lstset{
  	  numbers=left,
      framexleftmargin=15pt,
      xleftmargin=15pt,            
    }
}



% Paperref ==========================================================================
% includes a published note at the top of the title page for local copies

\usepackage[absolute]{textpos}
%\setlength{\TPHorizModule}{0.8\textwidth}
\definecolor{shadecolor}{rgb}{0.8,0.8,0.8} 

\newcommand{\paperref}[3]{%  
  \setlength{\TPHorizModule}{#1}
  \textblockorigin{0mm}{#2}
  \begin{textblock}{1}(0,0)    
      \begin{shaded}#3\end{shaded}    
  \end{textblock}
}



% \usepackage{verbatim}% http://ctan.org/pkg/verbatim
% \newenvironment{note}{\endgraf\color{MidnightBlue}\scriptsize\verbatim}{\endverbatim}

\newenvironment{rewrite}{\endgraf\color{MidnightBlue}}{}


%\usepackage{fancyvrb, xstring}
%\DefineVerbatimEnvironment{note}{Verbatim}{
%    formatcom=\color{MidnightBlue}\scriptsize
%}

%\renewcommand{\FancyVerbFormatLine}[1]{
%   {\StrSubstitute[1]{#1}{☐}{-}}
%}

% \usepackage{listings}

\lstnewenvironment{note}%
  {%
    %\minipage{\textwidth} 
    \lstset{
      basicstyle=\color{MidnightBlue}\scriptsize\ttfamily,
      breaklines=true,
      xleftmargin=0.5cm,
      extendedchars=true,
      frame=none
    }
  }
  {
    %\endminipage
  }

\usepackage{grfext}
\PrependGraphicsExtensions*{.pdf}


% \usepackage[toc]{glossaries}
% \makeglossaries

\usepackage[nopostdot,nonumberlist,nomain,acronym,xindy%,toc
]{glossaries} % nomain, if you define glossaries in a file, and you use \include{INP-00-glossary}
\makeglossaries
\usepackage[xindy]{imakeidx}
\makeindex

\renewcommand*{\glsnamefont}[1]{\textbf{#1}}



%   \commentmode
%
% - Comments Labeled with Author Id. Define your own author id before \begin{document}:
%   \newcommand\myAuthorID[1]{\nbnote{MAID}{#1}}
%   Use it anywhere in the text: 
%   ... \myAuthorID{this is a comment} ... 
%
% Listings Package Configuration for Several Languages
% ====================================================
% - Syntax highlighting for JCop, ContextJ, LogicAJ, LogicAJ2, GenTL, Prolog
% - Code makro to define code snippets independently of their appeariance. Use:
%   \code{public void foo(){}}
%
% Published Notice on Title Page
% ============================= 
% Paper reference feature to place a publication note on top of the title page of local downloads. 
% Use the following command after \begin{document}: 
% \paperref{<framed box width>}{<height>}{This paper appeared at the n-th XY Conference on Z}
%
% last modified: MA, 2009-08-17_14-47
%%%%%%%%%%%%%%%%%%%%%%%%%%%%%%%%%%%%%%%%%%%%%%%%%%%%%%%%%%%%%%%%%%%%%%%%%%%%%%%%%%%%%%%%%%%%%%%%%%%
\usepackage{amsmath}
\usepackage[usenames,dvipsnames]{color}

\usepackage{framed}
\usepackage{xspace}
\usepackage[hyphens]{url}
%\urlstyle{same}
\urlstyle{tt}
%\usepackage{caption}
\DeclareMathAlphabet{\mathpzc}{OT1}{pzc}{m}{it}
% General %%%%%%%%%%%%%%%%%%%%%%%%%%%%%%%%%%%%%%%%%%%%%%%%%%%%%%%%%%%%%%%%%%%%%%%%%%%%%%%%%%%%%%%%%

% API Table %%%%%%%%%%%%%%%%%%%%%%%%%%%%%%%%%%%%%%%%%%%%%%%%%%%%%%%%%%%%%%%%%%%%%%%%%%%%%%%
\usepackage{longtable}
\usepackage{multirow}

\newenvironment{api2}[1]{%
\begin{tabular}{p{.5cm} p{13.5cm}}%
\hline%
%\multicolumn{2}{|c|}{\code{#1}}\\ \hline %
}%
{\hline%
\end{tabular}%
}
\newcommand{\apidesctxt}[2]{\multicolumn{2}{l}{\lstinline[language=JCop, basicstyle=\listingsfont]{public #1}} \\ & #2 \\}

\newenvironment{api}[1]{%
\begin{longtable}{p{.5cm} p{12.5cm}}%
%\hline%
%\multicolumn{2}{|c|}{\code{#1}}\\ \hline %
}%
{%\hline%
\end{longtable}%
}

\newcommand{\apidesc}[2]{\multicolumn{2}{l}{\lstinline[language=JCop, basicstyle=\listingsfont]{public #1}} \\ & \footnotesize{#2} \\}


% COMMENTS & DRAFT  %%%%%%%%%%%%%%%%%%%%%%%%%%%%%%%%%%%%%%%%%%%%%%%%%%%%%%%%%%%%%%%%%%%%%%%%%%%%%%%

\usepackage{ednotes}
\usepackage{amssymb}
\usepackage{datetime}
\usepackage{everypage}

% date and timeformat
\newdateformat{simpledate}{\THEYEAR-\twodigit{\THEMONTH}-\twodigit{\THEDAY}}
\newtimeformat{simpletime}{\twodigit{\THEHOUR}:\twodigit{\THEMINUTE}}
\newcommand{\timestamp}{\simpledate\today~\simpletime}

% header for draft notice
\newcommand{\draftheader}{%
  %\framebox[\textwidth][t]{%
	  \centerline	
}

% the default definitions of \nbnote and \svnversion are empty	
\newcommand{\nbnote}[2]{}
\newcommand{\svnversion}{}

% the actions perfomed while in comment mode
\newcommand{\commentmode}{
		% \usepackage{draftwatermark} \SetWatermarkLightness{0.70} \SetWatermarkText{\Huge DRAFT}   	
		\renewcommand{\nbnote}[2]{%
		  \fbox{\bfseries\sffamily\scriptsize##1}
		  {\sf\small $\blacktriangleright$ \textit##2 $\blacktriangleleft$}
		  %{\sf\small$\blacktriangleright$\textit{##2}$\blacktriangleleft$}%
		}
	 \renewcommand{\svnversion}{\emph{\footnotesize\svnId}}
	 \AddEverypageHook{\paperref{21cm}{-3mm}{\draftheader}}      
	 \AddEverypageHook{\paperref{21cm}{28.5cm}{\draftheader}}      
}


\newcommand\todo[1]{\nbnote{TO DO}{#1}}

\newcommand{\here}{\nbnote{***}{CONTINUE HERE}}
\newcommand\fix[1]{\nbnote{FIX}{#1}}
\newcommand\jl[1]{\nbnote{JL}{#1}}

\newcommand\rw[1]{\color{blue}#1\color{black}}

% dangerous!
% \DeclareOption{comments}{\commentmode}
% \ProcessOptions 

% LISTINGS  %%%%%%%%%%%%%%%%%%%%%%%%%%%%%%%%%%%%%%%%%%%%%%%%%%%%%%%%%%%%%%%%%%%%%%%%%%%%%%%%%%%%%%%

\usepackage[formats]{listings}
\usepackage[T1]{fontenc}


%\usepackage{pifont}
%\usepackage{winfonts}
%\usepackage{windingbats}
%\usepackage{winpifont}
\usepackage{soul}
%\DisableLigatures{encoding = T1, family = courier }
%\usepackage{beramono}
\usepackage[scaled=.84]{DejaVuSansMono}
%\linespread{1.5}  % mathpazo

%\fontfamily{courier} \selectfont 

\newcommand{\code}[1]{\mbox{\texttt{#1}}} % code font
%\let\code\lstinline
\newcommand{\scode}[1]{\texttt{\small{#1}}} % code font
%\newcommand{\code}[1]{%
%  \lstinline[language=JCop,%
%  					 basicstyle={\ttfamily \footnotesize \selectfont},%
%             keywordstyle={\bfseries \color[rgb]{0,0,0}}]!#1!} % code font
\newcommand{\biglistingsfont}{ \ttfamily \normalsize \selectfont }
\newcommand{\listingsfont}{ \ttfamily \scriptsize \selectfont }
\newcommand{\tinylistingsfont}{\rmfamily \fontsize{6.5}{6.5} \selectfont}
%\newcommand{\listingsfont}{ \fontfamily{courier} \scalebox{.5}[1] \scriptsize \selectfont }
%\newcommand{\tinylistingsfont}{\fontfamily{courier} \fontsize{6.5}{5}  \selectfont \scalebox{.5}[1]}

%\newcommand{\listingsfont}{\ttfamily \scriptsize}

% COLORS
% eclipse keywords: \color[rgb]{0.49,0.00,0.33}
% eclipse strings:  \color[rgb]{0.16,0.2,1}
% eclipse comments: \color[rgb]{0.247,0.498,0.372},


% -- JCop  ----------------------------------------------------------------------------------------

\newcommand{\layerclass}{\code{jcop.lang.Layer}\xspace}
\newcommand{\with}{\code{with}\xspace}
\newcommand{\without}{\code{without}\xspace}
\newcommand{\withoutall}{\code{withoutall}\xspace}
\newcommand{\layer}{\code{layer}\xspace}
\newcommand{\before}{\code{before}\xspace}
\newcommand{\after}{\code{after}\xspace}
\newcommand{\finalmod}{\code{final}\xspace}
\newcommand{\abstractmod}{\code{abstract}\xspace}
\newcommand{\proceed}{\code{proceed}\xspace} 
\newcommand{\ind}{\hspace*{1.5em}}



\lstdefinelanguage{ContextJS} {
      classoffset=0,
      keywords={cop, proceed, withLayer, withoutLayer, with, break,%
                layerClass, layerObject, createLayer, %
        case, catch, continue, else, extends,%
                false, finally, function, for, if, var, instanceof,%
                new, %
                return, super, this, throw,%
                true, try, undefined, while, },
      morecomment=[l]//,%    
      morecomment=[s]{/*}{*/},%
      morestring=[b]",%
      % basicstyle=\listingsfont,
      % basicstyle= \ttfamily \scriptsize \selectfont
      keywordstyle = \color[rgb]{0.54,0.27,0.00},
      %commentstyle = \color[rgb]{0.0,0.533,0.8},
      stringstyle = \color[rgb]{0.0,0.533,0.8},
      identifierstyle=,
      showstringspaces=false,
      captionpos=b,
      tabsize=2,
      aboveskip=5mm,
      %numbersep=-10pt,      
      extendedchars=true,
      frame=tb,
      framexleftmargin=0pt,
      framexrightmargin=-2pt,
      numbers=left,
      numberstyle=\tiny    
}


% -- None ---

\lstdefinelanguage{todo} {
  basicstyle=\color{MidnightBlue}\scriptsize\ttfamily,
  breaklines=true,
  xleftmargin=0.5cm,
  extendedchars=true,
  frame=none
}

\lstdefinelanguage{none} {
  keywords={},
  alsoletter={+},
  morekeywords={}
  frame=none
}

% -----------
  
  % general config
  \lstset{
    framexrightmargin=0pt,      
    numberstyle=\tiny,
    numbersep=5pt,      
    frame=tb,
    basicstyle=\linespread{1} \listingsfont, % print whole listing small
    keywordstyle = \bfseries \color[rgb]{0.49,0.00,0.33},%\color[rgb]{0.84,0.27,0.40},
    commentstyle = \color[rgb]{0.247,0.498,0.372},
    stringstyle = \color[rgb]{0.16,0.2,1},    
    captionpos=b,
    tabsize=2,
    aboveskip=5mm,
    showlines=true,
    showstringspaces=false,
    escapechar=\§,
    breakautoindent=false    
  }
  
  
  \newcommand{\lstnolinenumbers}{
  	\lstset{
  	  numbers=none,   	  
      framexleftmargin=4pt,
      xleftmargin=4pt,   
    }
  }
  	
  
  \newcommand{\lstlinenumbers}{ 
  	\lstset{
  	  numbers=left,
      framexleftmargin=15pt,
      xleftmargin=15pt,            
    }
}



% Paperref ==========================================================================
% includes a published note at the top of the title page for local copies

\usepackage[absolute]{textpos}
%\setlength{\TPHorizModule}{0.8\textwidth}
\definecolor{shadecolor}{rgb}{0.8,0.8,0.8} 

\newcommand{\paperref}[3]{%  
  \setlength{\TPHorizModule}{#1}
  \textblockorigin{0mm}{#2}
  \begin{textblock}{1}(0,0)    
      \begin{shaded}#3\end{shaded}    
  \end{textblock}
}



% \usepackage{verbatim}% http://ctan.org/pkg/verbatim
% \newenvironment{note}{\endgraf\color{MidnightBlue}\scriptsize\verbatim}{\endverbatim}

\newenvironment{rewrite}{\endgraf\color{MidnightBlue}}{}


%\usepackage{fancyvrb, xstring}
%\DefineVerbatimEnvironment{note}{Verbatim}{
%    formatcom=\color{MidnightBlue}\scriptsize
%}

%\renewcommand{\FancyVerbFormatLine}[1]{
%   {\StrSubstitute[1]{#1}{☐}{-}}
%}

% \usepackage{listings}

\lstnewenvironment{note}%
  {%
    %\minipage{\textwidth} 
    \lstset{
      basicstyle=\color{MidnightBlue}\scriptsize\ttfamily,
      breaklines=true,
      xleftmargin=0.5cm,
      extendedchars=true,
      frame=none
    }
  }
  {
    %\endminipage
  }

\usepackage{grfext}
\PrependGraphicsExtensions*{.pdf}


% \usepackage[toc]{glossaries}
% \makeglossaries

\usepackage[nopostdot,nonumberlist,nomain,acronym,xindy%,toc
]{glossaries} % nomain, if you define glossaries in a file, and you use \include{INP-00-glossary}
\makeglossaries
\usepackage[xindy]{imakeidx}
\makeindex

\renewcommand*{\glsnamefont}[1]{\textbf{#1}}



%   \commentmode
%
% - Comments Labeled with Author Id. Define your own author id before \begin{document}:
%   \newcommand\myAuthorID[1]{\nbnote{MAID}{#1}}
%   Use it anywhere in the text: 
%   ... \myAuthorID{this is a comment} ... 
%
% Listings Package Configuration for Several Languages
% ====================================================
% - Syntax highlighting for JCop, ContextJ, LogicAJ, LogicAJ2, GenTL, Prolog
% - Code makro to define code snippets independently of their appeariance. Use:
%   \code{public void foo(){}}
%
% Published Notice on Title Page
% ============================= 
% Paper reference feature to place a publication note on top of the title page of local downloads. 
% Use the following command after \begin{document}: 
% \paperref{<framed box width>}{<height>}{This paper appeared at the n-th XY Conference on Z}
%
% last modified: MA, 2009-08-17_14-47
%%%%%%%%%%%%%%%%%%%%%%%%%%%%%%%%%%%%%%%%%%%%%%%%%%%%%%%%%%%%%%%%%%%%%%%%%%%%%%%%%%%%%%%%%%%%%%%%%%%
\usepackage{amsmath}
\usepackage[usenames,dvipsnames]{color}

\usepackage{framed}
\usepackage{xspace}
\usepackage[hyphens]{url}
%\urlstyle{same}
\urlstyle{tt}
%\usepackage{caption}
\DeclareMathAlphabet{\mathpzc}{OT1}{pzc}{m}{it}
% General %%%%%%%%%%%%%%%%%%%%%%%%%%%%%%%%%%%%%%%%%%%%%%%%%%%%%%%%%%%%%%%%%%%%%%%%%%%%%%%%%%%%%%%%%

% API Table %%%%%%%%%%%%%%%%%%%%%%%%%%%%%%%%%%%%%%%%%%%%%%%%%%%%%%%%%%%%%%%%%%%%%%%%%%%%%%%
\usepackage{longtable}
\usepackage{multirow}

\newenvironment{api2}[1]{%
\begin{tabular}{p{.5cm} p{13.5cm}}%
\hline%
%\multicolumn{2}{|c|}{\code{#1}}\\ \hline %
}%
{\hline%
\end{tabular}%
}
\newcommand{\apidesctxt}[2]{\multicolumn{2}{l}{\lstinline[language=JCop, basicstyle=\listingsfont]{public #1}} \\ & #2 \\}

\newenvironment{api}[1]{%
\begin{longtable}{p{.5cm} p{12.5cm}}%
%\hline%
%\multicolumn{2}{|c|}{\code{#1}}\\ \hline %
}%
{%\hline%
\end{longtable}%
}

\newcommand{\apidesc}[2]{\multicolumn{2}{l}{\lstinline[language=JCop, basicstyle=\listingsfont]{public #1}} \\ & \footnotesize{#2} \\}


% COMMENTS & DRAFT  %%%%%%%%%%%%%%%%%%%%%%%%%%%%%%%%%%%%%%%%%%%%%%%%%%%%%%%%%%%%%%%%%%%%%%%%%%%%%%%

\usepackage{ednotes}
\usepackage{amssymb}
\usepackage{datetime}
\usepackage{everypage}

% date and timeformat
\newdateformat{simpledate}{\THEYEAR-\twodigit{\THEMONTH}-\twodigit{\THEDAY}}
\newtimeformat{simpletime}{\twodigit{\THEHOUR}:\twodigit{\THEMINUTE}}
\newcommand{\timestamp}{\simpledate\today~\simpletime}

% header for draft notice
\newcommand{\draftheader}{%
  %\framebox[\textwidth][t]{%
	  \centerline	
}

% the default definitions of \nbnote and \svnversion are empty	
\newcommand{\nbnote}[2]{}
\newcommand{\svnversion}{}

% the actions perfomed while in comment mode
\newcommand{\commentmode}{
		% \usepackage{draftwatermark} \SetWatermarkLightness{0.70} \SetWatermarkText{\Huge DRAFT}   	
		\renewcommand{\nbnote}[2]{%
		  \fbox{\bfseries\sffamily\scriptsize##1}
		  {\sf\small $\blacktriangleright$ \textit##2 $\blacktriangleleft$}
		  %{\sf\small$\blacktriangleright$\textit{##2}$\blacktriangleleft$}%
		}
	 \renewcommand{\svnversion}{\emph{\footnotesize\svnId}}
	 \AddEverypageHook{\paperref{21cm}{-3mm}{\draftheader}}      
	 \AddEverypageHook{\paperref{21cm}{28.5cm}{\draftheader}}      
}


\newcommand\todo[1]{\nbnote{TO DO}{#1}}

\newcommand{\here}{\nbnote{***}{CONTINUE HERE}}
\newcommand\fix[1]{\nbnote{FIX}{#1}}
\newcommand\jl[1]{\nbnote{JL}{#1}}

\newcommand\rw[1]{\color{blue}#1\color{black}}

% dangerous!
% \DeclareOption{comments}{\commentmode}
% \ProcessOptions 

% LISTINGS  %%%%%%%%%%%%%%%%%%%%%%%%%%%%%%%%%%%%%%%%%%%%%%%%%%%%%%%%%%%%%%%%%%%%%%%%%%%%%%%%%%%%%%%

\usepackage[formats]{listings}
\usepackage[T1]{fontenc}


%\usepackage{pifont}
%\usepackage{winfonts}
%\usepackage{windingbats}
%\usepackage{winpifont}
\usepackage{soul}
%\DisableLigatures{encoding = T1, family = courier }
%\usepackage{beramono}
\usepackage[scaled=.84]{DejaVuSansMono}
%\linespread{1.5}  % mathpazo

%\fontfamily{courier} \selectfont 

\newcommand{\code}[1]{\mbox{\texttt{#1}}} % code font
%\let\code\lstinline
\newcommand{\scode}[1]{\texttt{\small{#1}}} % code font
%\newcommand{\code}[1]{%
%  \lstinline[language=JCop,%
%  					 basicstyle={\ttfamily \footnotesize \selectfont},%
%             keywordstyle={\bfseries \color[rgb]{0,0,0}}]!#1!} % code font
\newcommand{\biglistingsfont}{ \ttfamily \normalsize \selectfont }
\newcommand{\listingsfont}{ \ttfamily \scriptsize \selectfont }
\newcommand{\tinylistingsfont}{\rmfamily \fontsize{6.5}{6.5} \selectfont}
%\newcommand{\listingsfont}{ \fontfamily{courier} \scalebox{.5}[1] \scriptsize \selectfont }
%\newcommand{\tinylistingsfont}{\fontfamily{courier} \fontsize{6.5}{5}  \selectfont \scalebox{.5}[1]}

%\newcommand{\listingsfont}{\ttfamily \scriptsize}

% COLORS
% eclipse keywords: \color[rgb]{0.49,0.00,0.33}
% eclipse strings:  \color[rgb]{0.16,0.2,1}
% eclipse comments: \color[rgb]{0.247,0.498,0.372},


% -- JCop  ----------------------------------------------------------------------------------------

\newcommand{\layerclass}{\code{jcop.lang.Layer}\xspace}
\newcommand{\with}{\code{with}\xspace}
\newcommand{\without}{\code{without}\xspace}
\newcommand{\withoutall}{\code{withoutall}\xspace}
\newcommand{\layer}{\code{layer}\xspace}
\newcommand{\before}{\code{before}\xspace}
\newcommand{\after}{\code{after}\xspace}
\newcommand{\finalmod}{\code{final}\xspace}
\newcommand{\abstractmod}{\code{abstract}\xspace}
\newcommand{\proceed}{\code{proceed}\xspace} 
\newcommand{\ind}{\hspace*{1.5em}}



\lstdefinelanguage{ContextJS} {
      classoffset=0,
      keywords={cop, proceed, withLayer, withoutLayer, with, break,%
                layerClass, layerObject, createLayer, %
        case, catch, continue, else, extends,%
                false, finally, function, for, if, var, instanceof,%
                new, %
                return, super, this, throw,%
                true, try, undefined, while, },
      morecomment=[l]//,%    
      morecomment=[s]{/*}{*/},%
      morestring=[b]",%
      % basicstyle=\listingsfont,
      % basicstyle= \ttfamily \scriptsize \selectfont
      keywordstyle = \color[rgb]{0.54,0.27,0.00},
      %commentstyle = \color[rgb]{0.0,0.533,0.8},
      stringstyle = \color[rgb]{0.0,0.533,0.8},
      identifierstyle=,
      showstringspaces=false,
      captionpos=b,
      tabsize=2,
      aboveskip=5mm,
      %numbersep=-10pt,      
      extendedchars=true,
      frame=tb,
      framexleftmargin=0pt,
      framexrightmargin=-2pt,
      numbers=left,
      numberstyle=\tiny    
}


% -- None ---

\lstdefinelanguage{todo} {
  basicstyle=\color{MidnightBlue}\scriptsize\ttfamily,
  breaklines=true,
  xleftmargin=0.5cm,
  extendedchars=true,
  frame=none
}

\lstdefinelanguage{none} {
  keywords={},
  alsoletter={+},
  morekeywords={}
  frame=none
}

% -----------
  
  % general config
  \lstset{
    framexrightmargin=0pt,      
    numberstyle=\tiny,
    numbersep=5pt,      
    frame=tb,
    basicstyle=\linespread{1} \listingsfont, % print whole listing small
    keywordstyle = \bfseries \color[rgb]{0.49,0.00,0.33},%\color[rgb]{0.84,0.27,0.40},
    commentstyle = \color[rgb]{0.247,0.498,0.372},
    stringstyle = \color[rgb]{0.16,0.2,1},    
    captionpos=b,
    tabsize=2,
    aboveskip=5mm,
    showlines=true,
    showstringspaces=false,
    escapechar=\§,
    breakautoindent=false    
  }
  
  
  \newcommand{\lstnolinenumbers}{
  	\lstset{
  	  numbers=none,   	  
      framexleftmargin=4pt,
      xleftmargin=4pt,   
    }
  }
  	
  
  \newcommand{\lstlinenumbers}{ 
  	\lstset{
  	  numbers=left,
      framexleftmargin=15pt,
      xleftmargin=15pt,            
    }
}



% Paperref ==========================================================================
% includes a published note at the top of the title page for local copies

\usepackage[absolute]{textpos}
%\setlength{\TPHorizModule}{0.8\textwidth}
\definecolor{shadecolor}{rgb}{0.8,0.8,0.8} 

\newcommand{\paperref}[3]{%  
  \setlength{\TPHorizModule}{#1}
  \textblockorigin{0mm}{#2}
  \begin{textblock}{1}(0,0)    
      \begin{shaded}#3\end{shaded}    
  \end{textblock}
}



% \usepackage{verbatim}% http://ctan.org/pkg/verbatim
% \newenvironment{note}{\endgraf\color{MidnightBlue}\scriptsize\verbatim}{\endverbatim}

\newenvironment{rewrite}{\endgraf\color{MidnightBlue}}{}


%\usepackage{fancyvrb, xstring}
%\DefineVerbatimEnvironment{note}{Verbatim}{
%    formatcom=\color{MidnightBlue}\scriptsize
%}

%\renewcommand{\FancyVerbFormatLine}[1]{
%   {\StrSubstitute[1]{#1}{☐}{-}}
%}

% \usepackage{listings}

\lstnewenvironment{note}%
  {%
    %\minipage{\textwidth} 
    \lstset{
      basicstyle=\color{MidnightBlue}\scriptsize\ttfamily,
      breaklines=true,
      xleftmargin=0.5cm,
      extendedchars=true,
      frame=none
    }
  }
  {
    %\endminipage
  }

\usepackage{grfext}
\PrependGraphicsExtensions*{.pdf}


% \usepackage[toc]{glossaries}
% \makeglossaries

\usepackage[nopostdot,nonumberlist,nomain,acronym,xindy%,toc
]{glossaries} % nomain, if you define glossaries in a file, and you use \include{INP-00-glossary}
\makeglossaries
\usepackage[xindy]{imakeidx}
\makeindex

\renewcommand*{\glsnamefont}[1]{\textbf{#1}}



% \commentmode
%
% - Draft Notice and Svn Info in a Framed Header on Every Page. This feature is automatically 
%   activated while in comment mode, but needs an additional declaration in the preamble to set 
%   the SVN Id: \def\svnId{$$Id:$$}
% 
%   \documentclass{...}
%   \def\svnId{$$Id:$$}
%    %TEX root = main.tex
%%%%%%%%%%%%%%%%%%%%%%%%%%%%%%%%%%%%%%%%%%%%%%%%%%%%%%%%%%%%%%%%%%%%%%%%%%%%%%%%%%%%%%%%%%%%%%%%%%%
%
% This file contains:
%
% Comment Mode Support 
% ==================
% The following features are only active while in comment mode. To enter this mode, simply 
% set \commentmode in the preamble. 
%
% \documentclass{...}
%  %TEX root = main.tex
%%%%%%%%%%%%%%%%%%%%%%%%%%%%%%%%%%%%%%%%%%%%%%%%%%%%%%%%%%%%%%%%%%%%%%%%%%%%%%%%%%%%%%%%%%%%%%%%%%%
%
% This file contains:
%
% Comment Mode Support 
% ==================
% The following features are only active while in comment mode. To enter this mode, simply 
% set \commentmode in the preamble. 
%
% \documentclass{...}
%  %TEX root = main.tex
%%%%%%%%%%%%%%%%%%%%%%%%%%%%%%%%%%%%%%%%%%%%%%%%%%%%%%%%%%%%%%%%%%%%%%%%%%%%%%%%%%%%%%%%%%%%%%%%%%%
%
% This file contains:
%
% Comment Mode Support 
% ==================
% The following features are only active while in comment mode. To enter this mode, simply 
% set \commentmode in the preamble. 
%
% \documentclass{...}
% \input{auxiliary}
% \commentmode
%
% - Draft Notice and Svn Info in a Framed Header on Every Page. This feature is automatically 
%   activated while in comment mode, but needs an additional declaration in the preamble to set 
%   the SVN Id: \def\svnId{$$Id:$$}
% 
%   \documentclass{...}
%   \def\svnId{$$Id:$$}
%   \input{auxiliary}
%   \commentmode
%
% - Comments Labeled with Author Id. Define your own author id before \begin{document}:
%   \newcommand\myAuthorID[1]{\nbnote{MAID}{#1}}
%   Use it anywhere in the text: 
%   ... \myAuthorID{this is a comment} ... 
%
% Listings Package Configuration for Several Languages
% ====================================================
% - Syntax highlighting for JCop, ContextJ, LogicAJ, LogicAJ2, GenTL, Prolog
% - Code makro to define code snippets independently of their appeariance. Use:
%   \code{public void foo(){}}
%
% Published Notice on Title Page
% ============================= 
% Paper reference feature to place a publication note on top of the title page of local downloads. 
% Use the following command after \begin{document}: 
% \paperref{<framed box width>}{<height>}{This paper appeared at the n-th XY Conference on Z}
%
% last modified: MA, 2009-08-17_14-47
%%%%%%%%%%%%%%%%%%%%%%%%%%%%%%%%%%%%%%%%%%%%%%%%%%%%%%%%%%%%%%%%%%%%%%%%%%%%%%%%%%%%%%%%%%%%%%%%%%%
\usepackage{amsmath}
\usepackage[usenames,dvipsnames]{color}

\usepackage{framed}
\usepackage{xspace}
\usepackage[hyphens]{url}
%\urlstyle{same}
\urlstyle{tt}
%\usepackage{caption}
\DeclareMathAlphabet{\mathpzc}{OT1}{pzc}{m}{it}
% General %%%%%%%%%%%%%%%%%%%%%%%%%%%%%%%%%%%%%%%%%%%%%%%%%%%%%%%%%%%%%%%%%%%%%%%%%%%%%%%%%%%%%%%%%

% API Table %%%%%%%%%%%%%%%%%%%%%%%%%%%%%%%%%%%%%%%%%%%%%%%%%%%%%%%%%%%%%%%%%%%%%%%%%%%%%%%
\usepackage{longtable}
\usepackage{multirow}

\newenvironment{api2}[1]{%
\begin{tabular}{p{.5cm} p{13.5cm}}%
\hline%
%\multicolumn{2}{|c|}{\code{#1}}\\ \hline %
}%
{\hline%
\end{tabular}%
}
\newcommand{\apidesctxt}[2]{\multicolumn{2}{l}{\lstinline[language=JCop, basicstyle=\listingsfont]{public #1}} \\ & #2 \\}

\newenvironment{api}[1]{%
\begin{longtable}{p{.5cm} p{12.5cm}}%
%\hline%
%\multicolumn{2}{|c|}{\code{#1}}\\ \hline %
}%
{%\hline%
\end{longtable}%
}

\newcommand{\apidesc}[2]{\multicolumn{2}{l}{\lstinline[language=JCop, basicstyle=\listingsfont]{public #1}} \\ & \footnotesize{#2} \\}


% COMMENTS & DRAFT  %%%%%%%%%%%%%%%%%%%%%%%%%%%%%%%%%%%%%%%%%%%%%%%%%%%%%%%%%%%%%%%%%%%%%%%%%%%%%%%

\usepackage{ednotes}
\usepackage{amssymb}
\usepackage{datetime}
\usepackage{everypage}

% date and timeformat
\newdateformat{simpledate}{\THEYEAR-\twodigit{\THEMONTH}-\twodigit{\THEDAY}}
\newtimeformat{simpletime}{\twodigit{\THEHOUR}:\twodigit{\THEMINUTE}}
\newcommand{\timestamp}{\simpledate\today~\simpletime}

% header for draft notice
\newcommand{\draftheader}{%
  %\framebox[\textwidth][t]{%
	  \centerline	
}

% the default definitions of \nbnote and \svnversion are empty	
\newcommand{\nbnote}[2]{}
\newcommand{\svnversion}{}

% the actions perfomed while in comment mode
\newcommand{\commentmode}{
		% \usepackage{draftwatermark} \SetWatermarkLightness{0.70} \SetWatermarkText{\Huge DRAFT}   	
		\renewcommand{\nbnote}[2]{%
		  \fbox{\bfseries\sffamily\scriptsize##1}
		  {\sf\small $\blacktriangleright$ \textit##2 $\blacktriangleleft$}
		  %{\sf\small$\blacktriangleright$\textit{##2}$\blacktriangleleft$}%
		}
	 \renewcommand{\svnversion}{\emph{\footnotesize\svnId}}
	 \AddEverypageHook{\paperref{21cm}{-3mm}{\draftheader}}      
	 \AddEverypageHook{\paperref{21cm}{28.5cm}{\draftheader}}      
}


\newcommand\todo[1]{\nbnote{TO DO}{#1}}

\newcommand{\here}{\nbnote{***}{CONTINUE HERE}}
\newcommand\fix[1]{\nbnote{FIX}{#1}}
\newcommand\jl[1]{\nbnote{JL}{#1}}

\newcommand\rw[1]{\color{blue}#1\color{black}}

% dangerous!
% \DeclareOption{comments}{\commentmode}
% \ProcessOptions 

% LISTINGS  %%%%%%%%%%%%%%%%%%%%%%%%%%%%%%%%%%%%%%%%%%%%%%%%%%%%%%%%%%%%%%%%%%%%%%%%%%%%%%%%%%%%%%%

\usepackage[formats]{listings}
\usepackage[T1]{fontenc}


%\usepackage{pifont}
%\usepackage{winfonts}
%\usepackage{windingbats}
%\usepackage{winpifont}
\usepackage{soul}
%\DisableLigatures{encoding = T1, family = courier }
%\usepackage{beramono}
\usepackage[scaled=.84]{DejaVuSansMono}
%\linespread{1.5}  % mathpazo

%\fontfamily{courier} \selectfont 

\newcommand{\code}[1]{\mbox{\texttt{#1}}} % code font
%\let\code\lstinline
\newcommand{\scode}[1]{\texttt{\small{#1}}} % code font
%\newcommand{\code}[1]{%
%  \lstinline[language=JCop,%
%  					 basicstyle={\ttfamily \footnotesize \selectfont},%
%             keywordstyle={\bfseries \color[rgb]{0,0,0}}]!#1!} % code font
\newcommand{\biglistingsfont}{ \ttfamily \normalsize \selectfont }
\newcommand{\listingsfont}{ \ttfamily \scriptsize \selectfont }
\newcommand{\tinylistingsfont}{\rmfamily \fontsize{6.5}{6.5} \selectfont}
%\newcommand{\listingsfont}{ \fontfamily{courier} \scalebox{.5}[1] \scriptsize \selectfont }
%\newcommand{\tinylistingsfont}{\fontfamily{courier} \fontsize{6.5}{5}  \selectfont \scalebox{.5}[1]}

%\newcommand{\listingsfont}{\ttfamily \scriptsize}

% COLORS
% eclipse keywords: \color[rgb]{0.49,0.00,0.33}
% eclipse strings:  \color[rgb]{0.16,0.2,1}
% eclipse comments: \color[rgb]{0.247,0.498,0.372},


% -- JCop  ----------------------------------------------------------------------------------------

\newcommand{\layerclass}{\code{jcop.lang.Layer}\xspace}
\newcommand{\with}{\code{with}\xspace}
\newcommand{\without}{\code{without}\xspace}
\newcommand{\withoutall}{\code{withoutall}\xspace}
\newcommand{\layer}{\code{layer}\xspace}
\newcommand{\before}{\code{before}\xspace}
\newcommand{\after}{\code{after}\xspace}
\newcommand{\finalmod}{\code{final}\xspace}
\newcommand{\abstractmod}{\code{abstract}\xspace}
\newcommand{\proceed}{\code{proceed}\xspace} 
\newcommand{\ind}{\hspace*{1.5em}}



\lstdefinelanguage{ContextJS} {
      classoffset=0,
      keywords={cop, proceed, withLayer, withoutLayer, with, break,%
                layerClass, layerObject, createLayer, %
        case, catch, continue, else, extends,%
                false, finally, function, for, if, var, instanceof,%
                new, %
                return, super, this, throw,%
                true, try, undefined, while, },
      morecomment=[l]//,%    
      morecomment=[s]{/*}{*/},%
      morestring=[b]",%
      % basicstyle=\listingsfont,
      % basicstyle= \ttfamily \scriptsize \selectfont
      keywordstyle = \color[rgb]{0.54,0.27,0.00},
      %commentstyle = \color[rgb]{0.0,0.533,0.8},
      stringstyle = \color[rgb]{0.0,0.533,0.8},
      identifierstyle=,
      showstringspaces=false,
      captionpos=b,
      tabsize=2,
      aboveskip=5mm,
      %numbersep=-10pt,      
      extendedchars=true,
      frame=tb,
      framexleftmargin=0pt,
      framexrightmargin=-2pt,
      numbers=left,
      numberstyle=\tiny    
}


% -- None ---

\lstdefinelanguage{todo} {
  basicstyle=\color{MidnightBlue}\scriptsize\ttfamily,
  breaklines=true,
  xleftmargin=0.5cm,
  extendedchars=true,
  frame=none
}

\lstdefinelanguage{none} {
  keywords={},
  alsoletter={+},
  morekeywords={}
  frame=none
}

% -----------
  
  % general config
  \lstset{
    framexrightmargin=0pt,      
    numberstyle=\tiny,
    numbersep=5pt,      
    frame=tb,
    basicstyle=\linespread{1} \listingsfont, % print whole listing small
    keywordstyle = \bfseries \color[rgb]{0.49,0.00,0.33},%\color[rgb]{0.84,0.27,0.40},
    commentstyle = \color[rgb]{0.247,0.498,0.372},
    stringstyle = \color[rgb]{0.16,0.2,1},    
    captionpos=b,
    tabsize=2,
    aboveskip=5mm,
    showlines=true,
    showstringspaces=false,
    escapechar=\§,
    breakautoindent=false    
  }
  
  
  \newcommand{\lstnolinenumbers}{
  	\lstset{
  	  numbers=none,   	  
      framexleftmargin=4pt,
      xleftmargin=4pt,   
    }
  }
  	
  
  \newcommand{\lstlinenumbers}{ 
  	\lstset{
  	  numbers=left,
      framexleftmargin=15pt,
      xleftmargin=15pt,            
    }
}



% Paperref ==========================================================================
% includes a published note at the top of the title page for local copies

\usepackage[absolute]{textpos}
%\setlength{\TPHorizModule}{0.8\textwidth}
\definecolor{shadecolor}{rgb}{0.8,0.8,0.8} 

\newcommand{\paperref}[3]{%  
  \setlength{\TPHorizModule}{#1}
  \textblockorigin{0mm}{#2}
  \begin{textblock}{1}(0,0)    
      \begin{shaded}#3\end{shaded}    
  \end{textblock}
}



% \usepackage{verbatim}% http://ctan.org/pkg/verbatim
% \newenvironment{note}{\endgraf\color{MidnightBlue}\scriptsize\verbatim}{\endverbatim}

\newenvironment{rewrite}{\endgraf\color{MidnightBlue}}{}


%\usepackage{fancyvrb, xstring}
%\DefineVerbatimEnvironment{note}{Verbatim}{
%    formatcom=\color{MidnightBlue}\scriptsize
%}

%\renewcommand{\FancyVerbFormatLine}[1]{
%   {\StrSubstitute[1]{#1}{☐}{-}}
%}

% \usepackage{listings}

\lstnewenvironment{note}%
  {%
    %\minipage{\textwidth} 
    \lstset{
      basicstyle=\color{MidnightBlue}\scriptsize\ttfamily,
      breaklines=true,
      xleftmargin=0.5cm,
      extendedchars=true,
      frame=none
    }
  }
  {
    %\endminipage
  }

\usepackage{grfext}
\PrependGraphicsExtensions*{.pdf}


% \usepackage[toc]{glossaries}
% \makeglossaries

\usepackage[nopostdot,nonumberlist,nomain,acronym,xindy%,toc
]{glossaries} % nomain, if you define glossaries in a file, and you use \include{INP-00-glossary}
\makeglossaries
\usepackage[xindy]{imakeidx}
\makeindex

\renewcommand*{\glsnamefont}[1]{\textbf{#1}}



% \commentmode
%
% - Draft Notice and Svn Info in a Framed Header on Every Page. This feature is automatically 
%   activated while in comment mode, but needs an additional declaration in the preamble to set 
%   the SVN Id: \def\svnId{$$Id:$$}
% 
%   \documentclass{...}
%   \def\svnId{$$Id:$$}
%    %TEX root = main.tex
%%%%%%%%%%%%%%%%%%%%%%%%%%%%%%%%%%%%%%%%%%%%%%%%%%%%%%%%%%%%%%%%%%%%%%%%%%%%%%%%%%%%%%%%%%%%%%%%%%%
%
% This file contains:
%
% Comment Mode Support 
% ==================
% The following features are only active while in comment mode. To enter this mode, simply 
% set \commentmode in the preamble. 
%
% \documentclass{...}
% \input{auxiliary}
% \commentmode
%
% - Draft Notice and Svn Info in a Framed Header on Every Page. This feature is automatically 
%   activated while in comment mode, but needs an additional declaration in the preamble to set 
%   the SVN Id: \def\svnId{$$Id:$$}
% 
%   \documentclass{...}
%   \def\svnId{$$Id:$$}
%   \input{auxiliary}
%   \commentmode
%
% - Comments Labeled with Author Id. Define your own author id before \begin{document}:
%   \newcommand\myAuthorID[1]{\nbnote{MAID}{#1}}
%   Use it anywhere in the text: 
%   ... \myAuthorID{this is a comment} ... 
%
% Listings Package Configuration for Several Languages
% ====================================================
% - Syntax highlighting for JCop, ContextJ, LogicAJ, LogicAJ2, GenTL, Prolog
% - Code makro to define code snippets independently of their appeariance. Use:
%   \code{public void foo(){}}
%
% Published Notice on Title Page
% ============================= 
% Paper reference feature to place a publication note on top of the title page of local downloads. 
% Use the following command after \begin{document}: 
% \paperref{<framed box width>}{<height>}{This paper appeared at the n-th XY Conference on Z}
%
% last modified: MA, 2009-08-17_14-47
%%%%%%%%%%%%%%%%%%%%%%%%%%%%%%%%%%%%%%%%%%%%%%%%%%%%%%%%%%%%%%%%%%%%%%%%%%%%%%%%%%%%%%%%%%%%%%%%%%%
\usepackage{amsmath}
\usepackage[usenames,dvipsnames]{color}

\usepackage{framed}
\usepackage{xspace}
\usepackage[hyphens]{url}
%\urlstyle{same}
\urlstyle{tt}
%\usepackage{caption}
\DeclareMathAlphabet{\mathpzc}{OT1}{pzc}{m}{it}
% General %%%%%%%%%%%%%%%%%%%%%%%%%%%%%%%%%%%%%%%%%%%%%%%%%%%%%%%%%%%%%%%%%%%%%%%%%%%%%%%%%%%%%%%%%

% API Table %%%%%%%%%%%%%%%%%%%%%%%%%%%%%%%%%%%%%%%%%%%%%%%%%%%%%%%%%%%%%%%%%%%%%%%%%%%%%%%
\usepackage{longtable}
\usepackage{multirow}

\newenvironment{api2}[1]{%
\begin{tabular}{p{.5cm} p{13.5cm}}%
\hline%
%\multicolumn{2}{|c|}{\code{#1}}\\ \hline %
}%
{\hline%
\end{tabular}%
}
\newcommand{\apidesctxt}[2]{\multicolumn{2}{l}{\lstinline[language=JCop, basicstyle=\listingsfont]{public #1}} \\ & #2 \\}

\newenvironment{api}[1]{%
\begin{longtable}{p{.5cm} p{12.5cm}}%
%\hline%
%\multicolumn{2}{|c|}{\code{#1}}\\ \hline %
}%
{%\hline%
\end{longtable}%
}

\newcommand{\apidesc}[2]{\multicolumn{2}{l}{\lstinline[language=JCop, basicstyle=\listingsfont]{public #1}} \\ & \footnotesize{#2} \\}


% COMMENTS & DRAFT  %%%%%%%%%%%%%%%%%%%%%%%%%%%%%%%%%%%%%%%%%%%%%%%%%%%%%%%%%%%%%%%%%%%%%%%%%%%%%%%

\usepackage{ednotes}
\usepackage{amssymb}
\usepackage{datetime}
\usepackage{everypage}

% date and timeformat
\newdateformat{simpledate}{\THEYEAR-\twodigit{\THEMONTH}-\twodigit{\THEDAY}}
\newtimeformat{simpletime}{\twodigit{\THEHOUR}:\twodigit{\THEMINUTE}}
\newcommand{\timestamp}{\simpledate\today~\simpletime}

% header for draft notice
\newcommand{\draftheader}{%
  %\framebox[\textwidth][t]{%
	  \centerline	
}

% the default definitions of \nbnote and \svnversion are empty	
\newcommand{\nbnote}[2]{}
\newcommand{\svnversion}{}

% the actions perfomed while in comment mode
\newcommand{\commentmode}{
		% \usepackage{draftwatermark} \SetWatermarkLightness{0.70} \SetWatermarkText{\Huge DRAFT}   	
		\renewcommand{\nbnote}[2]{%
		  \fbox{\bfseries\sffamily\scriptsize##1}
		  {\sf\small $\blacktriangleright$ \textit##2 $\blacktriangleleft$}
		  %{\sf\small$\blacktriangleright$\textit{##2}$\blacktriangleleft$}%
		}
	 \renewcommand{\svnversion}{\emph{\footnotesize\svnId}}
	 \AddEverypageHook{\paperref{21cm}{-3mm}{\draftheader}}      
	 \AddEverypageHook{\paperref{21cm}{28.5cm}{\draftheader}}      
}


\newcommand\todo[1]{\nbnote{TO DO}{#1}}

\newcommand{\here}{\nbnote{***}{CONTINUE HERE}}
\newcommand\fix[1]{\nbnote{FIX}{#1}}
\newcommand\jl[1]{\nbnote{JL}{#1}}

\newcommand\rw[1]{\color{blue}#1\color{black}}

% dangerous!
% \DeclareOption{comments}{\commentmode}
% \ProcessOptions 

% LISTINGS  %%%%%%%%%%%%%%%%%%%%%%%%%%%%%%%%%%%%%%%%%%%%%%%%%%%%%%%%%%%%%%%%%%%%%%%%%%%%%%%%%%%%%%%

\usepackage[formats]{listings}
\usepackage[T1]{fontenc}


%\usepackage{pifont}
%\usepackage{winfonts}
%\usepackage{windingbats}
%\usepackage{winpifont}
\usepackage{soul}
%\DisableLigatures{encoding = T1, family = courier }
%\usepackage{beramono}
\usepackage[scaled=.84]{DejaVuSansMono}
%\linespread{1.5}  % mathpazo

%\fontfamily{courier} \selectfont 

\newcommand{\code}[1]{\mbox{\texttt{#1}}} % code font
%\let\code\lstinline
\newcommand{\scode}[1]{\texttt{\small{#1}}} % code font
%\newcommand{\code}[1]{%
%  \lstinline[language=JCop,%
%  					 basicstyle={\ttfamily \footnotesize \selectfont},%
%             keywordstyle={\bfseries \color[rgb]{0,0,0}}]!#1!} % code font
\newcommand{\biglistingsfont}{ \ttfamily \normalsize \selectfont }
\newcommand{\listingsfont}{ \ttfamily \scriptsize \selectfont }
\newcommand{\tinylistingsfont}{\rmfamily \fontsize{6.5}{6.5} \selectfont}
%\newcommand{\listingsfont}{ \fontfamily{courier} \scalebox{.5}[1] \scriptsize \selectfont }
%\newcommand{\tinylistingsfont}{\fontfamily{courier} \fontsize{6.5}{5}  \selectfont \scalebox{.5}[1]}

%\newcommand{\listingsfont}{\ttfamily \scriptsize}

% COLORS
% eclipse keywords: \color[rgb]{0.49,0.00,0.33}
% eclipse strings:  \color[rgb]{0.16,0.2,1}
% eclipse comments: \color[rgb]{0.247,0.498,0.372},


% -- JCop  ----------------------------------------------------------------------------------------

\newcommand{\layerclass}{\code{jcop.lang.Layer}\xspace}
\newcommand{\with}{\code{with}\xspace}
\newcommand{\without}{\code{without}\xspace}
\newcommand{\withoutall}{\code{withoutall}\xspace}
\newcommand{\layer}{\code{layer}\xspace}
\newcommand{\before}{\code{before}\xspace}
\newcommand{\after}{\code{after}\xspace}
\newcommand{\finalmod}{\code{final}\xspace}
\newcommand{\abstractmod}{\code{abstract}\xspace}
\newcommand{\proceed}{\code{proceed}\xspace} 
\newcommand{\ind}{\hspace*{1.5em}}



\lstdefinelanguage{ContextJS} {
      classoffset=0,
      keywords={cop, proceed, withLayer, withoutLayer, with, break,%
                layerClass, layerObject, createLayer, %
        case, catch, continue, else, extends,%
                false, finally, function, for, if, var, instanceof,%
                new, %
                return, super, this, throw,%
                true, try, undefined, while, },
      morecomment=[l]//,%    
      morecomment=[s]{/*}{*/},%
      morestring=[b]",%
      % basicstyle=\listingsfont,
      % basicstyle= \ttfamily \scriptsize \selectfont
      keywordstyle = \color[rgb]{0.54,0.27,0.00},
      %commentstyle = \color[rgb]{0.0,0.533,0.8},
      stringstyle = \color[rgb]{0.0,0.533,0.8},
      identifierstyle=,
      showstringspaces=false,
      captionpos=b,
      tabsize=2,
      aboveskip=5mm,
      %numbersep=-10pt,      
      extendedchars=true,
      frame=tb,
      framexleftmargin=0pt,
      framexrightmargin=-2pt,
      numbers=left,
      numberstyle=\tiny    
}


% -- None ---

\lstdefinelanguage{todo} {
  basicstyle=\color{MidnightBlue}\scriptsize\ttfamily,
  breaklines=true,
  xleftmargin=0.5cm,
  extendedchars=true,
  frame=none
}

\lstdefinelanguage{none} {
  keywords={},
  alsoletter={+},
  morekeywords={}
  frame=none
}

% -----------
  
  % general config
  \lstset{
    framexrightmargin=0pt,      
    numberstyle=\tiny,
    numbersep=5pt,      
    frame=tb,
    basicstyle=\linespread{1} \listingsfont, % print whole listing small
    keywordstyle = \bfseries \color[rgb]{0.49,0.00,0.33},%\color[rgb]{0.84,0.27,0.40},
    commentstyle = \color[rgb]{0.247,0.498,0.372},
    stringstyle = \color[rgb]{0.16,0.2,1},    
    captionpos=b,
    tabsize=2,
    aboveskip=5mm,
    showlines=true,
    showstringspaces=false,
    escapechar=\§,
    breakautoindent=false    
  }
  
  
  \newcommand{\lstnolinenumbers}{
  	\lstset{
  	  numbers=none,   	  
      framexleftmargin=4pt,
      xleftmargin=4pt,   
    }
  }
  	
  
  \newcommand{\lstlinenumbers}{ 
  	\lstset{
  	  numbers=left,
      framexleftmargin=15pt,
      xleftmargin=15pt,            
    }
}



% Paperref ==========================================================================
% includes a published note at the top of the title page for local copies

\usepackage[absolute]{textpos}
%\setlength{\TPHorizModule}{0.8\textwidth}
\definecolor{shadecolor}{rgb}{0.8,0.8,0.8} 

\newcommand{\paperref}[3]{%  
  \setlength{\TPHorizModule}{#1}
  \textblockorigin{0mm}{#2}
  \begin{textblock}{1}(0,0)    
      \begin{shaded}#3\end{shaded}    
  \end{textblock}
}



% \usepackage{verbatim}% http://ctan.org/pkg/verbatim
% \newenvironment{note}{\endgraf\color{MidnightBlue}\scriptsize\verbatim}{\endverbatim}

\newenvironment{rewrite}{\endgraf\color{MidnightBlue}}{}


%\usepackage{fancyvrb, xstring}
%\DefineVerbatimEnvironment{note}{Verbatim}{
%    formatcom=\color{MidnightBlue}\scriptsize
%}

%\renewcommand{\FancyVerbFormatLine}[1]{
%   {\StrSubstitute[1]{#1}{☐}{-}}
%}

% \usepackage{listings}

\lstnewenvironment{note}%
  {%
    %\minipage{\textwidth} 
    \lstset{
      basicstyle=\color{MidnightBlue}\scriptsize\ttfamily,
      breaklines=true,
      xleftmargin=0.5cm,
      extendedchars=true,
      frame=none
    }
  }
  {
    %\endminipage
  }

\usepackage{grfext}
\PrependGraphicsExtensions*{.pdf}


% \usepackage[toc]{glossaries}
% \makeglossaries

\usepackage[nopostdot,nonumberlist,nomain,acronym,xindy%,toc
]{glossaries} % nomain, if you define glossaries in a file, and you use \include{INP-00-glossary}
\makeglossaries
\usepackage[xindy]{imakeidx}
\makeindex

\renewcommand*{\glsnamefont}[1]{\textbf{#1}}



%   \commentmode
%
% - Comments Labeled with Author Id. Define your own author id before \begin{document}:
%   \newcommand\myAuthorID[1]{\nbnote{MAID}{#1}}
%   Use it anywhere in the text: 
%   ... \myAuthorID{this is a comment} ... 
%
% Listings Package Configuration for Several Languages
% ====================================================
% - Syntax highlighting for JCop, ContextJ, LogicAJ, LogicAJ2, GenTL, Prolog
% - Code makro to define code snippets independently of their appeariance. Use:
%   \code{public void foo(){}}
%
% Published Notice on Title Page
% ============================= 
% Paper reference feature to place a publication note on top of the title page of local downloads. 
% Use the following command after \begin{document}: 
% \paperref{<framed box width>}{<height>}{This paper appeared at the n-th XY Conference on Z}
%
% last modified: MA, 2009-08-17_14-47
%%%%%%%%%%%%%%%%%%%%%%%%%%%%%%%%%%%%%%%%%%%%%%%%%%%%%%%%%%%%%%%%%%%%%%%%%%%%%%%%%%%%%%%%%%%%%%%%%%%
\usepackage{amsmath}
\usepackage[usenames,dvipsnames]{color}

\usepackage{framed}
\usepackage{xspace}
\usepackage[hyphens]{url}
%\urlstyle{same}
\urlstyle{tt}
%\usepackage{caption}
\DeclareMathAlphabet{\mathpzc}{OT1}{pzc}{m}{it}
% General %%%%%%%%%%%%%%%%%%%%%%%%%%%%%%%%%%%%%%%%%%%%%%%%%%%%%%%%%%%%%%%%%%%%%%%%%%%%%%%%%%%%%%%%%

% API Table %%%%%%%%%%%%%%%%%%%%%%%%%%%%%%%%%%%%%%%%%%%%%%%%%%%%%%%%%%%%%%%%%%%%%%%%%%%%%%%
\usepackage{longtable}
\usepackage{multirow}

\newenvironment{api2}[1]{%
\begin{tabular}{p{.5cm} p{13.5cm}}%
\hline%
%\multicolumn{2}{|c|}{\code{#1}}\\ \hline %
}%
{\hline%
\end{tabular}%
}
\newcommand{\apidesctxt}[2]{\multicolumn{2}{l}{\lstinline[language=JCop, basicstyle=\listingsfont]{public #1}} \\ & #2 \\}

\newenvironment{api}[1]{%
\begin{longtable}{p{.5cm} p{12.5cm}}%
%\hline%
%\multicolumn{2}{|c|}{\code{#1}}\\ \hline %
}%
{%\hline%
\end{longtable}%
}

\newcommand{\apidesc}[2]{\multicolumn{2}{l}{\lstinline[language=JCop, basicstyle=\listingsfont]{public #1}} \\ & \footnotesize{#2} \\}


% COMMENTS & DRAFT  %%%%%%%%%%%%%%%%%%%%%%%%%%%%%%%%%%%%%%%%%%%%%%%%%%%%%%%%%%%%%%%%%%%%%%%%%%%%%%%

\usepackage{ednotes}
\usepackage{amssymb}
\usepackage{datetime}
\usepackage{everypage}

% date and timeformat
\newdateformat{simpledate}{\THEYEAR-\twodigit{\THEMONTH}-\twodigit{\THEDAY}}
\newtimeformat{simpletime}{\twodigit{\THEHOUR}:\twodigit{\THEMINUTE}}
\newcommand{\timestamp}{\simpledate\today~\simpletime}

% header for draft notice
\newcommand{\draftheader}{%
  %\framebox[\textwidth][t]{%
	  \centerline	
}

% the default definitions of \nbnote and \svnversion are empty	
\newcommand{\nbnote}[2]{}
\newcommand{\svnversion}{}

% the actions perfomed while in comment mode
\newcommand{\commentmode}{
		% \usepackage{draftwatermark} \SetWatermarkLightness{0.70} \SetWatermarkText{\Huge DRAFT}   	
		\renewcommand{\nbnote}[2]{%
		  \fbox{\bfseries\sffamily\scriptsize##1}
		  {\sf\small $\blacktriangleright$ \textit##2 $\blacktriangleleft$}
		  %{\sf\small$\blacktriangleright$\textit{##2}$\blacktriangleleft$}%
		}
	 \renewcommand{\svnversion}{\emph{\footnotesize\svnId}}
	 \AddEverypageHook{\paperref{21cm}{-3mm}{\draftheader}}      
	 \AddEverypageHook{\paperref{21cm}{28.5cm}{\draftheader}}      
}


\newcommand\todo[1]{\nbnote{TO DO}{#1}}

\newcommand{\here}{\nbnote{***}{CONTINUE HERE}}
\newcommand\fix[1]{\nbnote{FIX}{#1}}
\newcommand\jl[1]{\nbnote{JL}{#1}}

\newcommand\rw[1]{\color{blue}#1\color{black}}

% dangerous!
% \DeclareOption{comments}{\commentmode}
% \ProcessOptions 

% LISTINGS  %%%%%%%%%%%%%%%%%%%%%%%%%%%%%%%%%%%%%%%%%%%%%%%%%%%%%%%%%%%%%%%%%%%%%%%%%%%%%%%%%%%%%%%

\usepackage[formats]{listings}
\usepackage[T1]{fontenc}


%\usepackage{pifont}
%\usepackage{winfonts}
%\usepackage{windingbats}
%\usepackage{winpifont}
\usepackage{soul}
%\DisableLigatures{encoding = T1, family = courier }
%\usepackage{beramono}
\usepackage[scaled=.84]{DejaVuSansMono}
%\linespread{1.5}  % mathpazo

%\fontfamily{courier} \selectfont 

\newcommand{\code}[1]{\mbox{\texttt{#1}}} % code font
%\let\code\lstinline
\newcommand{\scode}[1]{\texttt{\small{#1}}} % code font
%\newcommand{\code}[1]{%
%  \lstinline[language=JCop,%
%  					 basicstyle={\ttfamily \footnotesize \selectfont},%
%             keywordstyle={\bfseries \color[rgb]{0,0,0}}]!#1!} % code font
\newcommand{\biglistingsfont}{ \ttfamily \normalsize \selectfont }
\newcommand{\listingsfont}{ \ttfamily \scriptsize \selectfont }
\newcommand{\tinylistingsfont}{\rmfamily \fontsize{6.5}{6.5} \selectfont}
%\newcommand{\listingsfont}{ \fontfamily{courier} \scalebox{.5}[1] \scriptsize \selectfont }
%\newcommand{\tinylistingsfont}{\fontfamily{courier} \fontsize{6.5}{5}  \selectfont \scalebox{.5}[1]}

%\newcommand{\listingsfont}{\ttfamily \scriptsize}

% COLORS
% eclipse keywords: \color[rgb]{0.49,0.00,0.33}
% eclipse strings:  \color[rgb]{0.16,0.2,1}
% eclipse comments: \color[rgb]{0.247,0.498,0.372},


% -- JCop  ----------------------------------------------------------------------------------------

\newcommand{\layerclass}{\code{jcop.lang.Layer}\xspace}
\newcommand{\with}{\code{with}\xspace}
\newcommand{\without}{\code{without}\xspace}
\newcommand{\withoutall}{\code{withoutall}\xspace}
\newcommand{\layer}{\code{layer}\xspace}
\newcommand{\before}{\code{before}\xspace}
\newcommand{\after}{\code{after}\xspace}
\newcommand{\finalmod}{\code{final}\xspace}
\newcommand{\abstractmod}{\code{abstract}\xspace}
\newcommand{\proceed}{\code{proceed}\xspace} 
\newcommand{\ind}{\hspace*{1.5em}}



\lstdefinelanguage{ContextJS} {
      classoffset=0,
      keywords={cop, proceed, withLayer, withoutLayer, with, break,%
                layerClass, layerObject, createLayer, %
        case, catch, continue, else, extends,%
                false, finally, function, for, if, var, instanceof,%
                new, %
                return, super, this, throw,%
                true, try, undefined, while, },
      morecomment=[l]//,%    
      morecomment=[s]{/*}{*/},%
      morestring=[b]",%
      % basicstyle=\listingsfont,
      % basicstyle= \ttfamily \scriptsize \selectfont
      keywordstyle = \color[rgb]{0.54,0.27,0.00},
      %commentstyle = \color[rgb]{0.0,0.533,0.8},
      stringstyle = \color[rgb]{0.0,0.533,0.8},
      identifierstyle=,
      showstringspaces=false,
      captionpos=b,
      tabsize=2,
      aboveskip=5mm,
      %numbersep=-10pt,      
      extendedchars=true,
      frame=tb,
      framexleftmargin=0pt,
      framexrightmargin=-2pt,
      numbers=left,
      numberstyle=\tiny    
}


% -- None ---

\lstdefinelanguage{todo} {
  basicstyle=\color{MidnightBlue}\scriptsize\ttfamily,
  breaklines=true,
  xleftmargin=0.5cm,
  extendedchars=true,
  frame=none
}

\lstdefinelanguage{none} {
  keywords={},
  alsoletter={+},
  morekeywords={}
  frame=none
}

% -----------
  
  % general config
  \lstset{
    framexrightmargin=0pt,      
    numberstyle=\tiny,
    numbersep=5pt,      
    frame=tb,
    basicstyle=\linespread{1} \listingsfont, % print whole listing small
    keywordstyle = \bfseries \color[rgb]{0.49,0.00,0.33},%\color[rgb]{0.84,0.27,0.40},
    commentstyle = \color[rgb]{0.247,0.498,0.372},
    stringstyle = \color[rgb]{0.16,0.2,1},    
    captionpos=b,
    tabsize=2,
    aboveskip=5mm,
    showlines=true,
    showstringspaces=false,
    escapechar=\§,
    breakautoindent=false    
  }
  
  
  \newcommand{\lstnolinenumbers}{
  	\lstset{
  	  numbers=none,   	  
      framexleftmargin=4pt,
      xleftmargin=4pt,   
    }
  }
  	
  
  \newcommand{\lstlinenumbers}{ 
  	\lstset{
  	  numbers=left,
      framexleftmargin=15pt,
      xleftmargin=15pt,            
    }
}



% Paperref ==========================================================================
% includes a published note at the top of the title page for local copies

\usepackage[absolute]{textpos}
%\setlength{\TPHorizModule}{0.8\textwidth}
\definecolor{shadecolor}{rgb}{0.8,0.8,0.8} 

\newcommand{\paperref}[3]{%  
  \setlength{\TPHorizModule}{#1}
  \textblockorigin{0mm}{#2}
  \begin{textblock}{1}(0,0)    
      \begin{shaded}#3\end{shaded}    
  \end{textblock}
}



% \usepackage{verbatim}% http://ctan.org/pkg/verbatim
% \newenvironment{note}{\endgraf\color{MidnightBlue}\scriptsize\verbatim}{\endverbatim}

\newenvironment{rewrite}{\endgraf\color{MidnightBlue}}{}


%\usepackage{fancyvrb, xstring}
%\DefineVerbatimEnvironment{note}{Verbatim}{
%    formatcom=\color{MidnightBlue}\scriptsize
%}

%\renewcommand{\FancyVerbFormatLine}[1]{
%   {\StrSubstitute[1]{#1}{☐}{-}}
%}

% \usepackage{listings}

\lstnewenvironment{note}%
  {%
    %\minipage{\textwidth} 
    \lstset{
      basicstyle=\color{MidnightBlue}\scriptsize\ttfamily,
      breaklines=true,
      xleftmargin=0.5cm,
      extendedchars=true,
      frame=none
    }
  }
  {
    %\endminipage
  }

\usepackage{grfext}
\PrependGraphicsExtensions*{.pdf}


% \usepackage[toc]{glossaries}
% \makeglossaries

\usepackage[nopostdot,nonumberlist,nomain,acronym,xindy%,toc
]{glossaries} % nomain, if you define glossaries in a file, and you use \include{INP-00-glossary}
\makeglossaries
\usepackage[xindy]{imakeidx}
\makeindex

\renewcommand*{\glsnamefont}[1]{\textbf{#1}}



% \commentmode
%
% - Draft Notice and Svn Info in a Framed Header on Every Page. This feature is automatically 
%   activated while in comment mode, but needs an additional declaration in the preamble to set 
%   the SVN Id: \def\svnId{$$Id:$$}
% 
%   \documentclass{...}
%   \def\svnId{$$Id:$$}
%    %TEX root = main.tex
%%%%%%%%%%%%%%%%%%%%%%%%%%%%%%%%%%%%%%%%%%%%%%%%%%%%%%%%%%%%%%%%%%%%%%%%%%%%%%%%%%%%%%%%%%%%%%%%%%%
%
% This file contains:
%
% Comment Mode Support 
% ==================
% The following features are only active while in comment mode. To enter this mode, simply 
% set \commentmode in the preamble. 
%
% \documentclass{...}
%  %TEX root = main.tex
%%%%%%%%%%%%%%%%%%%%%%%%%%%%%%%%%%%%%%%%%%%%%%%%%%%%%%%%%%%%%%%%%%%%%%%%%%%%%%%%%%%%%%%%%%%%%%%%%%%
%
% This file contains:
%
% Comment Mode Support 
% ==================
% The following features are only active while in comment mode. To enter this mode, simply 
% set \commentmode in the preamble. 
%
% \documentclass{...}
% \input{auxiliary}
% \commentmode
%
% - Draft Notice and Svn Info in a Framed Header on Every Page. This feature is automatically 
%   activated while in comment mode, but needs an additional declaration in the preamble to set 
%   the SVN Id: \def\svnId{$$Id:$$}
% 
%   \documentclass{...}
%   \def\svnId{$$Id:$$}
%   \input{auxiliary}
%   \commentmode
%
% - Comments Labeled with Author Id. Define your own author id before \begin{document}:
%   \newcommand\myAuthorID[1]{\nbnote{MAID}{#1}}
%   Use it anywhere in the text: 
%   ... \myAuthorID{this is a comment} ... 
%
% Listings Package Configuration for Several Languages
% ====================================================
% - Syntax highlighting for JCop, ContextJ, LogicAJ, LogicAJ2, GenTL, Prolog
% - Code makro to define code snippets independently of their appeariance. Use:
%   \code{public void foo(){}}
%
% Published Notice on Title Page
% ============================= 
% Paper reference feature to place a publication note on top of the title page of local downloads. 
% Use the following command after \begin{document}: 
% \paperref{<framed box width>}{<height>}{This paper appeared at the n-th XY Conference on Z}
%
% last modified: MA, 2009-08-17_14-47
%%%%%%%%%%%%%%%%%%%%%%%%%%%%%%%%%%%%%%%%%%%%%%%%%%%%%%%%%%%%%%%%%%%%%%%%%%%%%%%%%%%%%%%%%%%%%%%%%%%
\usepackage{amsmath}
\usepackage[usenames,dvipsnames]{color}

\usepackage{framed}
\usepackage{xspace}
\usepackage[hyphens]{url}
%\urlstyle{same}
\urlstyle{tt}
%\usepackage{caption}
\DeclareMathAlphabet{\mathpzc}{OT1}{pzc}{m}{it}
% General %%%%%%%%%%%%%%%%%%%%%%%%%%%%%%%%%%%%%%%%%%%%%%%%%%%%%%%%%%%%%%%%%%%%%%%%%%%%%%%%%%%%%%%%%

% API Table %%%%%%%%%%%%%%%%%%%%%%%%%%%%%%%%%%%%%%%%%%%%%%%%%%%%%%%%%%%%%%%%%%%%%%%%%%%%%%%
\usepackage{longtable}
\usepackage{multirow}

\newenvironment{api2}[1]{%
\begin{tabular}{p{.5cm} p{13.5cm}}%
\hline%
%\multicolumn{2}{|c|}{\code{#1}}\\ \hline %
}%
{\hline%
\end{tabular}%
}
\newcommand{\apidesctxt}[2]{\multicolumn{2}{l}{\lstinline[language=JCop, basicstyle=\listingsfont]{public #1}} \\ & #2 \\}

\newenvironment{api}[1]{%
\begin{longtable}{p{.5cm} p{12.5cm}}%
%\hline%
%\multicolumn{2}{|c|}{\code{#1}}\\ \hline %
}%
{%\hline%
\end{longtable}%
}

\newcommand{\apidesc}[2]{\multicolumn{2}{l}{\lstinline[language=JCop, basicstyle=\listingsfont]{public #1}} \\ & \footnotesize{#2} \\}


% COMMENTS & DRAFT  %%%%%%%%%%%%%%%%%%%%%%%%%%%%%%%%%%%%%%%%%%%%%%%%%%%%%%%%%%%%%%%%%%%%%%%%%%%%%%%

\usepackage{ednotes}
\usepackage{amssymb}
\usepackage{datetime}
\usepackage{everypage}

% date and timeformat
\newdateformat{simpledate}{\THEYEAR-\twodigit{\THEMONTH}-\twodigit{\THEDAY}}
\newtimeformat{simpletime}{\twodigit{\THEHOUR}:\twodigit{\THEMINUTE}}
\newcommand{\timestamp}{\simpledate\today~\simpletime}

% header for draft notice
\newcommand{\draftheader}{%
  %\framebox[\textwidth][t]{%
	  \centerline	
}

% the default definitions of \nbnote and \svnversion are empty	
\newcommand{\nbnote}[2]{}
\newcommand{\svnversion}{}

% the actions perfomed while in comment mode
\newcommand{\commentmode}{
		% \usepackage{draftwatermark} \SetWatermarkLightness{0.70} \SetWatermarkText{\Huge DRAFT}   	
		\renewcommand{\nbnote}[2]{%
		  \fbox{\bfseries\sffamily\scriptsize##1}
		  {\sf\small $\blacktriangleright$ \textit##2 $\blacktriangleleft$}
		  %{\sf\small$\blacktriangleright$\textit{##2}$\blacktriangleleft$}%
		}
	 \renewcommand{\svnversion}{\emph{\footnotesize\svnId}}
	 \AddEverypageHook{\paperref{21cm}{-3mm}{\draftheader}}      
	 \AddEverypageHook{\paperref{21cm}{28.5cm}{\draftheader}}      
}


\newcommand\todo[1]{\nbnote{TO DO}{#1}}

\newcommand{\here}{\nbnote{***}{CONTINUE HERE}}
\newcommand\fix[1]{\nbnote{FIX}{#1}}
\newcommand\jl[1]{\nbnote{JL}{#1}}

\newcommand\rw[1]{\color{blue}#1\color{black}}

% dangerous!
% \DeclareOption{comments}{\commentmode}
% \ProcessOptions 

% LISTINGS  %%%%%%%%%%%%%%%%%%%%%%%%%%%%%%%%%%%%%%%%%%%%%%%%%%%%%%%%%%%%%%%%%%%%%%%%%%%%%%%%%%%%%%%

\usepackage[formats]{listings}
\usepackage[T1]{fontenc}


%\usepackage{pifont}
%\usepackage{winfonts}
%\usepackage{windingbats}
%\usepackage{winpifont}
\usepackage{soul}
%\DisableLigatures{encoding = T1, family = courier }
%\usepackage{beramono}
\usepackage[scaled=.84]{DejaVuSansMono}
%\linespread{1.5}  % mathpazo

%\fontfamily{courier} \selectfont 

\newcommand{\code}[1]{\mbox{\texttt{#1}}} % code font
%\let\code\lstinline
\newcommand{\scode}[1]{\texttt{\small{#1}}} % code font
%\newcommand{\code}[1]{%
%  \lstinline[language=JCop,%
%  					 basicstyle={\ttfamily \footnotesize \selectfont},%
%             keywordstyle={\bfseries \color[rgb]{0,0,0}}]!#1!} % code font
\newcommand{\biglistingsfont}{ \ttfamily \normalsize \selectfont }
\newcommand{\listingsfont}{ \ttfamily \scriptsize \selectfont }
\newcommand{\tinylistingsfont}{\rmfamily \fontsize{6.5}{6.5} \selectfont}
%\newcommand{\listingsfont}{ \fontfamily{courier} \scalebox{.5}[1] \scriptsize \selectfont }
%\newcommand{\tinylistingsfont}{\fontfamily{courier} \fontsize{6.5}{5}  \selectfont \scalebox{.5}[1]}

%\newcommand{\listingsfont}{\ttfamily \scriptsize}

% COLORS
% eclipse keywords: \color[rgb]{0.49,0.00,0.33}
% eclipse strings:  \color[rgb]{0.16,0.2,1}
% eclipse comments: \color[rgb]{0.247,0.498,0.372},


% -- JCop  ----------------------------------------------------------------------------------------

\newcommand{\layerclass}{\code{jcop.lang.Layer}\xspace}
\newcommand{\with}{\code{with}\xspace}
\newcommand{\without}{\code{without}\xspace}
\newcommand{\withoutall}{\code{withoutall}\xspace}
\newcommand{\layer}{\code{layer}\xspace}
\newcommand{\before}{\code{before}\xspace}
\newcommand{\after}{\code{after}\xspace}
\newcommand{\finalmod}{\code{final}\xspace}
\newcommand{\abstractmod}{\code{abstract}\xspace}
\newcommand{\proceed}{\code{proceed}\xspace} 
\newcommand{\ind}{\hspace*{1.5em}}



\lstdefinelanguage{ContextJS} {
      classoffset=0,
      keywords={cop, proceed, withLayer, withoutLayer, with, break,%
                layerClass, layerObject, createLayer, %
        case, catch, continue, else, extends,%
                false, finally, function, for, if, var, instanceof,%
                new, %
                return, super, this, throw,%
                true, try, undefined, while, },
      morecomment=[l]//,%    
      morecomment=[s]{/*}{*/},%
      morestring=[b]",%
      % basicstyle=\listingsfont,
      % basicstyle= \ttfamily \scriptsize \selectfont
      keywordstyle = \color[rgb]{0.54,0.27,0.00},
      %commentstyle = \color[rgb]{0.0,0.533,0.8},
      stringstyle = \color[rgb]{0.0,0.533,0.8},
      identifierstyle=,
      showstringspaces=false,
      captionpos=b,
      tabsize=2,
      aboveskip=5mm,
      %numbersep=-10pt,      
      extendedchars=true,
      frame=tb,
      framexleftmargin=0pt,
      framexrightmargin=-2pt,
      numbers=left,
      numberstyle=\tiny    
}


% -- None ---

\lstdefinelanguage{todo} {
  basicstyle=\color{MidnightBlue}\scriptsize\ttfamily,
  breaklines=true,
  xleftmargin=0.5cm,
  extendedchars=true,
  frame=none
}

\lstdefinelanguage{none} {
  keywords={},
  alsoletter={+},
  morekeywords={}
  frame=none
}

% -----------
  
  % general config
  \lstset{
    framexrightmargin=0pt,      
    numberstyle=\tiny,
    numbersep=5pt,      
    frame=tb,
    basicstyle=\linespread{1} \listingsfont, % print whole listing small
    keywordstyle = \bfseries \color[rgb]{0.49,0.00,0.33},%\color[rgb]{0.84,0.27,0.40},
    commentstyle = \color[rgb]{0.247,0.498,0.372},
    stringstyle = \color[rgb]{0.16,0.2,1},    
    captionpos=b,
    tabsize=2,
    aboveskip=5mm,
    showlines=true,
    showstringspaces=false,
    escapechar=\§,
    breakautoindent=false    
  }
  
  
  \newcommand{\lstnolinenumbers}{
  	\lstset{
  	  numbers=none,   	  
      framexleftmargin=4pt,
      xleftmargin=4pt,   
    }
  }
  	
  
  \newcommand{\lstlinenumbers}{ 
  	\lstset{
  	  numbers=left,
      framexleftmargin=15pt,
      xleftmargin=15pt,            
    }
}



% Paperref ==========================================================================
% includes a published note at the top of the title page for local copies

\usepackage[absolute]{textpos}
%\setlength{\TPHorizModule}{0.8\textwidth}
\definecolor{shadecolor}{rgb}{0.8,0.8,0.8} 

\newcommand{\paperref}[3]{%  
  \setlength{\TPHorizModule}{#1}
  \textblockorigin{0mm}{#2}
  \begin{textblock}{1}(0,0)    
      \begin{shaded}#3\end{shaded}    
  \end{textblock}
}



% \usepackage{verbatim}% http://ctan.org/pkg/verbatim
% \newenvironment{note}{\endgraf\color{MidnightBlue}\scriptsize\verbatim}{\endverbatim}

\newenvironment{rewrite}{\endgraf\color{MidnightBlue}}{}


%\usepackage{fancyvrb, xstring}
%\DefineVerbatimEnvironment{note}{Verbatim}{
%    formatcom=\color{MidnightBlue}\scriptsize
%}

%\renewcommand{\FancyVerbFormatLine}[1]{
%   {\StrSubstitute[1]{#1}{☐}{-}}
%}

% \usepackage{listings}

\lstnewenvironment{note}%
  {%
    %\minipage{\textwidth} 
    \lstset{
      basicstyle=\color{MidnightBlue}\scriptsize\ttfamily,
      breaklines=true,
      xleftmargin=0.5cm,
      extendedchars=true,
      frame=none
    }
  }
  {
    %\endminipage
  }

\usepackage{grfext}
\PrependGraphicsExtensions*{.pdf}


% \usepackage[toc]{glossaries}
% \makeglossaries

\usepackage[nopostdot,nonumberlist,nomain,acronym,xindy%,toc
]{glossaries} % nomain, if you define glossaries in a file, and you use \include{INP-00-glossary}
\makeglossaries
\usepackage[xindy]{imakeidx}
\makeindex

\renewcommand*{\glsnamefont}[1]{\textbf{#1}}



% \commentmode
%
% - Draft Notice and Svn Info in a Framed Header on Every Page. This feature is automatically 
%   activated while in comment mode, but needs an additional declaration in the preamble to set 
%   the SVN Id: \def\svnId{$$Id:$$}
% 
%   \documentclass{...}
%   \def\svnId{$$Id:$$}
%    %TEX root = main.tex
%%%%%%%%%%%%%%%%%%%%%%%%%%%%%%%%%%%%%%%%%%%%%%%%%%%%%%%%%%%%%%%%%%%%%%%%%%%%%%%%%%%%%%%%%%%%%%%%%%%
%
% This file contains:
%
% Comment Mode Support 
% ==================
% The following features are only active while in comment mode. To enter this mode, simply 
% set \commentmode in the preamble. 
%
% \documentclass{...}
% \input{auxiliary}
% \commentmode
%
% - Draft Notice and Svn Info in a Framed Header on Every Page. This feature is automatically 
%   activated while in comment mode, but needs an additional declaration in the preamble to set 
%   the SVN Id: \def\svnId{$$Id:$$}
% 
%   \documentclass{...}
%   \def\svnId{$$Id:$$}
%   \input{auxiliary}
%   \commentmode
%
% - Comments Labeled with Author Id. Define your own author id before \begin{document}:
%   \newcommand\myAuthorID[1]{\nbnote{MAID}{#1}}
%   Use it anywhere in the text: 
%   ... \myAuthorID{this is a comment} ... 
%
% Listings Package Configuration for Several Languages
% ====================================================
% - Syntax highlighting for JCop, ContextJ, LogicAJ, LogicAJ2, GenTL, Prolog
% - Code makro to define code snippets independently of their appeariance. Use:
%   \code{public void foo(){}}
%
% Published Notice on Title Page
% ============================= 
% Paper reference feature to place a publication note on top of the title page of local downloads. 
% Use the following command after \begin{document}: 
% \paperref{<framed box width>}{<height>}{This paper appeared at the n-th XY Conference on Z}
%
% last modified: MA, 2009-08-17_14-47
%%%%%%%%%%%%%%%%%%%%%%%%%%%%%%%%%%%%%%%%%%%%%%%%%%%%%%%%%%%%%%%%%%%%%%%%%%%%%%%%%%%%%%%%%%%%%%%%%%%
\usepackage{amsmath}
\usepackage[usenames,dvipsnames]{color}

\usepackage{framed}
\usepackage{xspace}
\usepackage[hyphens]{url}
%\urlstyle{same}
\urlstyle{tt}
%\usepackage{caption}
\DeclareMathAlphabet{\mathpzc}{OT1}{pzc}{m}{it}
% General %%%%%%%%%%%%%%%%%%%%%%%%%%%%%%%%%%%%%%%%%%%%%%%%%%%%%%%%%%%%%%%%%%%%%%%%%%%%%%%%%%%%%%%%%

% API Table %%%%%%%%%%%%%%%%%%%%%%%%%%%%%%%%%%%%%%%%%%%%%%%%%%%%%%%%%%%%%%%%%%%%%%%%%%%%%%%
\usepackage{longtable}
\usepackage{multirow}

\newenvironment{api2}[1]{%
\begin{tabular}{p{.5cm} p{13.5cm}}%
\hline%
%\multicolumn{2}{|c|}{\code{#1}}\\ \hline %
}%
{\hline%
\end{tabular}%
}
\newcommand{\apidesctxt}[2]{\multicolumn{2}{l}{\lstinline[language=JCop, basicstyle=\listingsfont]{public #1}} \\ & #2 \\}

\newenvironment{api}[1]{%
\begin{longtable}{p{.5cm} p{12.5cm}}%
%\hline%
%\multicolumn{2}{|c|}{\code{#1}}\\ \hline %
}%
{%\hline%
\end{longtable}%
}

\newcommand{\apidesc}[2]{\multicolumn{2}{l}{\lstinline[language=JCop, basicstyle=\listingsfont]{public #1}} \\ & \footnotesize{#2} \\}


% COMMENTS & DRAFT  %%%%%%%%%%%%%%%%%%%%%%%%%%%%%%%%%%%%%%%%%%%%%%%%%%%%%%%%%%%%%%%%%%%%%%%%%%%%%%%

\usepackage{ednotes}
\usepackage{amssymb}
\usepackage{datetime}
\usepackage{everypage}

% date and timeformat
\newdateformat{simpledate}{\THEYEAR-\twodigit{\THEMONTH}-\twodigit{\THEDAY}}
\newtimeformat{simpletime}{\twodigit{\THEHOUR}:\twodigit{\THEMINUTE}}
\newcommand{\timestamp}{\simpledate\today~\simpletime}

% header for draft notice
\newcommand{\draftheader}{%
  %\framebox[\textwidth][t]{%
	  \centerline	
}

% the default definitions of \nbnote and \svnversion are empty	
\newcommand{\nbnote}[2]{}
\newcommand{\svnversion}{}

% the actions perfomed while in comment mode
\newcommand{\commentmode}{
		% \usepackage{draftwatermark} \SetWatermarkLightness{0.70} \SetWatermarkText{\Huge DRAFT}   	
		\renewcommand{\nbnote}[2]{%
		  \fbox{\bfseries\sffamily\scriptsize##1}
		  {\sf\small $\blacktriangleright$ \textit##2 $\blacktriangleleft$}
		  %{\sf\small$\blacktriangleright$\textit{##2}$\blacktriangleleft$}%
		}
	 \renewcommand{\svnversion}{\emph{\footnotesize\svnId}}
	 \AddEverypageHook{\paperref{21cm}{-3mm}{\draftheader}}      
	 \AddEverypageHook{\paperref{21cm}{28.5cm}{\draftheader}}      
}


\newcommand\todo[1]{\nbnote{TO DO}{#1}}

\newcommand{\here}{\nbnote{***}{CONTINUE HERE}}
\newcommand\fix[1]{\nbnote{FIX}{#1}}
\newcommand\jl[1]{\nbnote{JL}{#1}}

\newcommand\rw[1]{\color{blue}#1\color{black}}

% dangerous!
% \DeclareOption{comments}{\commentmode}
% \ProcessOptions 

% LISTINGS  %%%%%%%%%%%%%%%%%%%%%%%%%%%%%%%%%%%%%%%%%%%%%%%%%%%%%%%%%%%%%%%%%%%%%%%%%%%%%%%%%%%%%%%

\usepackage[formats]{listings}
\usepackage[T1]{fontenc}


%\usepackage{pifont}
%\usepackage{winfonts}
%\usepackage{windingbats}
%\usepackage{winpifont}
\usepackage{soul}
%\DisableLigatures{encoding = T1, family = courier }
%\usepackage{beramono}
\usepackage[scaled=.84]{DejaVuSansMono}
%\linespread{1.5}  % mathpazo

%\fontfamily{courier} \selectfont 

\newcommand{\code}[1]{\mbox{\texttt{#1}}} % code font
%\let\code\lstinline
\newcommand{\scode}[1]{\texttt{\small{#1}}} % code font
%\newcommand{\code}[1]{%
%  \lstinline[language=JCop,%
%  					 basicstyle={\ttfamily \footnotesize \selectfont},%
%             keywordstyle={\bfseries \color[rgb]{0,0,0}}]!#1!} % code font
\newcommand{\biglistingsfont}{ \ttfamily \normalsize \selectfont }
\newcommand{\listingsfont}{ \ttfamily \scriptsize \selectfont }
\newcommand{\tinylistingsfont}{\rmfamily \fontsize{6.5}{6.5} \selectfont}
%\newcommand{\listingsfont}{ \fontfamily{courier} \scalebox{.5}[1] \scriptsize \selectfont }
%\newcommand{\tinylistingsfont}{\fontfamily{courier} \fontsize{6.5}{5}  \selectfont \scalebox{.5}[1]}

%\newcommand{\listingsfont}{\ttfamily \scriptsize}

% COLORS
% eclipse keywords: \color[rgb]{0.49,0.00,0.33}
% eclipse strings:  \color[rgb]{0.16,0.2,1}
% eclipse comments: \color[rgb]{0.247,0.498,0.372},


% -- JCop  ----------------------------------------------------------------------------------------

\newcommand{\layerclass}{\code{jcop.lang.Layer}\xspace}
\newcommand{\with}{\code{with}\xspace}
\newcommand{\without}{\code{without}\xspace}
\newcommand{\withoutall}{\code{withoutall}\xspace}
\newcommand{\layer}{\code{layer}\xspace}
\newcommand{\before}{\code{before}\xspace}
\newcommand{\after}{\code{after}\xspace}
\newcommand{\finalmod}{\code{final}\xspace}
\newcommand{\abstractmod}{\code{abstract}\xspace}
\newcommand{\proceed}{\code{proceed}\xspace} 
\newcommand{\ind}{\hspace*{1.5em}}



\lstdefinelanguage{ContextJS} {
      classoffset=0,
      keywords={cop, proceed, withLayer, withoutLayer, with, break,%
                layerClass, layerObject, createLayer, %
        case, catch, continue, else, extends,%
                false, finally, function, for, if, var, instanceof,%
                new, %
                return, super, this, throw,%
                true, try, undefined, while, },
      morecomment=[l]//,%    
      morecomment=[s]{/*}{*/},%
      morestring=[b]",%
      % basicstyle=\listingsfont,
      % basicstyle= \ttfamily \scriptsize \selectfont
      keywordstyle = \color[rgb]{0.54,0.27,0.00},
      %commentstyle = \color[rgb]{0.0,0.533,0.8},
      stringstyle = \color[rgb]{0.0,0.533,0.8},
      identifierstyle=,
      showstringspaces=false,
      captionpos=b,
      tabsize=2,
      aboveskip=5mm,
      %numbersep=-10pt,      
      extendedchars=true,
      frame=tb,
      framexleftmargin=0pt,
      framexrightmargin=-2pt,
      numbers=left,
      numberstyle=\tiny    
}


% -- None ---

\lstdefinelanguage{todo} {
  basicstyle=\color{MidnightBlue}\scriptsize\ttfamily,
  breaklines=true,
  xleftmargin=0.5cm,
  extendedchars=true,
  frame=none
}

\lstdefinelanguage{none} {
  keywords={},
  alsoletter={+},
  morekeywords={}
  frame=none
}

% -----------
  
  % general config
  \lstset{
    framexrightmargin=0pt,      
    numberstyle=\tiny,
    numbersep=5pt,      
    frame=tb,
    basicstyle=\linespread{1} \listingsfont, % print whole listing small
    keywordstyle = \bfseries \color[rgb]{0.49,0.00,0.33},%\color[rgb]{0.84,0.27,0.40},
    commentstyle = \color[rgb]{0.247,0.498,0.372},
    stringstyle = \color[rgb]{0.16,0.2,1},    
    captionpos=b,
    tabsize=2,
    aboveskip=5mm,
    showlines=true,
    showstringspaces=false,
    escapechar=\§,
    breakautoindent=false    
  }
  
  
  \newcommand{\lstnolinenumbers}{
  	\lstset{
  	  numbers=none,   	  
      framexleftmargin=4pt,
      xleftmargin=4pt,   
    }
  }
  	
  
  \newcommand{\lstlinenumbers}{ 
  	\lstset{
  	  numbers=left,
      framexleftmargin=15pt,
      xleftmargin=15pt,            
    }
}



% Paperref ==========================================================================
% includes a published note at the top of the title page for local copies

\usepackage[absolute]{textpos}
%\setlength{\TPHorizModule}{0.8\textwidth}
\definecolor{shadecolor}{rgb}{0.8,0.8,0.8} 

\newcommand{\paperref}[3]{%  
  \setlength{\TPHorizModule}{#1}
  \textblockorigin{0mm}{#2}
  \begin{textblock}{1}(0,0)    
      \begin{shaded}#3\end{shaded}    
  \end{textblock}
}



% \usepackage{verbatim}% http://ctan.org/pkg/verbatim
% \newenvironment{note}{\endgraf\color{MidnightBlue}\scriptsize\verbatim}{\endverbatim}

\newenvironment{rewrite}{\endgraf\color{MidnightBlue}}{}


%\usepackage{fancyvrb, xstring}
%\DefineVerbatimEnvironment{note}{Verbatim}{
%    formatcom=\color{MidnightBlue}\scriptsize
%}

%\renewcommand{\FancyVerbFormatLine}[1]{
%   {\StrSubstitute[1]{#1}{☐}{-}}
%}

% \usepackage{listings}

\lstnewenvironment{note}%
  {%
    %\minipage{\textwidth} 
    \lstset{
      basicstyle=\color{MidnightBlue}\scriptsize\ttfamily,
      breaklines=true,
      xleftmargin=0.5cm,
      extendedchars=true,
      frame=none
    }
  }
  {
    %\endminipage
  }

\usepackage{grfext}
\PrependGraphicsExtensions*{.pdf}


% \usepackage[toc]{glossaries}
% \makeglossaries

\usepackage[nopostdot,nonumberlist,nomain,acronym,xindy%,toc
]{glossaries} % nomain, if you define glossaries in a file, and you use \include{INP-00-glossary}
\makeglossaries
\usepackage[xindy]{imakeidx}
\makeindex

\renewcommand*{\glsnamefont}[1]{\textbf{#1}}



%   \commentmode
%
% - Comments Labeled with Author Id. Define your own author id before \begin{document}:
%   \newcommand\myAuthorID[1]{\nbnote{MAID}{#1}}
%   Use it anywhere in the text: 
%   ... \myAuthorID{this is a comment} ... 
%
% Listings Package Configuration for Several Languages
% ====================================================
% - Syntax highlighting for JCop, ContextJ, LogicAJ, LogicAJ2, GenTL, Prolog
% - Code makro to define code snippets independently of their appeariance. Use:
%   \code{public void foo(){}}
%
% Published Notice on Title Page
% ============================= 
% Paper reference feature to place a publication note on top of the title page of local downloads. 
% Use the following command after \begin{document}: 
% \paperref{<framed box width>}{<height>}{This paper appeared at the n-th XY Conference on Z}
%
% last modified: MA, 2009-08-17_14-47
%%%%%%%%%%%%%%%%%%%%%%%%%%%%%%%%%%%%%%%%%%%%%%%%%%%%%%%%%%%%%%%%%%%%%%%%%%%%%%%%%%%%%%%%%%%%%%%%%%%
\usepackage{amsmath}
\usepackage[usenames,dvipsnames]{color}

\usepackage{framed}
\usepackage{xspace}
\usepackage[hyphens]{url}
%\urlstyle{same}
\urlstyle{tt}
%\usepackage{caption}
\DeclareMathAlphabet{\mathpzc}{OT1}{pzc}{m}{it}
% General %%%%%%%%%%%%%%%%%%%%%%%%%%%%%%%%%%%%%%%%%%%%%%%%%%%%%%%%%%%%%%%%%%%%%%%%%%%%%%%%%%%%%%%%%

% API Table %%%%%%%%%%%%%%%%%%%%%%%%%%%%%%%%%%%%%%%%%%%%%%%%%%%%%%%%%%%%%%%%%%%%%%%%%%%%%%%
\usepackage{longtable}
\usepackage{multirow}

\newenvironment{api2}[1]{%
\begin{tabular}{p{.5cm} p{13.5cm}}%
\hline%
%\multicolumn{2}{|c|}{\code{#1}}\\ \hline %
}%
{\hline%
\end{tabular}%
}
\newcommand{\apidesctxt}[2]{\multicolumn{2}{l}{\lstinline[language=JCop, basicstyle=\listingsfont]{public #1}} \\ & #2 \\}

\newenvironment{api}[1]{%
\begin{longtable}{p{.5cm} p{12.5cm}}%
%\hline%
%\multicolumn{2}{|c|}{\code{#1}}\\ \hline %
}%
{%\hline%
\end{longtable}%
}

\newcommand{\apidesc}[2]{\multicolumn{2}{l}{\lstinline[language=JCop, basicstyle=\listingsfont]{public #1}} \\ & \footnotesize{#2} \\}


% COMMENTS & DRAFT  %%%%%%%%%%%%%%%%%%%%%%%%%%%%%%%%%%%%%%%%%%%%%%%%%%%%%%%%%%%%%%%%%%%%%%%%%%%%%%%

\usepackage{ednotes}
\usepackage{amssymb}
\usepackage{datetime}
\usepackage{everypage}

% date and timeformat
\newdateformat{simpledate}{\THEYEAR-\twodigit{\THEMONTH}-\twodigit{\THEDAY}}
\newtimeformat{simpletime}{\twodigit{\THEHOUR}:\twodigit{\THEMINUTE}}
\newcommand{\timestamp}{\simpledate\today~\simpletime}

% header for draft notice
\newcommand{\draftheader}{%
  %\framebox[\textwidth][t]{%
	  \centerline	
}

% the default definitions of \nbnote and \svnversion are empty	
\newcommand{\nbnote}[2]{}
\newcommand{\svnversion}{}

% the actions perfomed while in comment mode
\newcommand{\commentmode}{
		% \usepackage{draftwatermark} \SetWatermarkLightness{0.70} \SetWatermarkText{\Huge DRAFT}   	
		\renewcommand{\nbnote}[2]{%
		  \fbox{\bfseries\sffamily\scriptsize##1}
		  {\sf\small $\blacktriangleright$ \textit##2 $\blacktriangleleft$}
		  %{\sf\small$\blacktriangleright$\textit{##2}$\blacktriangleleft$}%
		}
	 \renewcommand{\svnversion}{\emph{\footnotesize\svnId}}
	 \AddEverypageHook{\paperref{21cm}{-3mm}{\draftheader}}      
	 \AddEverypageHook{\paperref{21cm}{28.5cm}{\draftheader}}      
}


\newcommand\todo[1]{\nbnote{TO DO}{#1}}

\newcommand{\here}{\nbnote{***}{CONTINUE HERE}}
\newcommand\fix[1]{\nbnote{FIX}{#1}}
\newcommand\jl[1]{\nbnote{JL}{#1}}

\newcommand\rw[1]{\color{blue}#1\color{black}}

% dangerous!
% \DeclareOption{comments}{\commentmode}
% \ProcessOptions 

% LISTINGS  %%%%%%%%%%%%%%%%%%%%%%%%%%%%%%%%%%%%%%%%%%%%%%%%%%%%%%%%%%%%%%%%%%%%%%%%%%%%%%%%%%%%%%%

\usepackage[formats]{listings}
\usepackage[T1]{fontenc}


%\usepackage{pifont}
%\usepackage{winfonts}
%\usepackage{windingbats}
%\usepackage{winpifont}
\usepackage{soul}
%\DisableLigatures{encoding = T1, family = courier }
%\usepackage{beramono}
\usepackage[scaled=.84]{DejaVuSansMono}
%\linespread{1.5}  % mathpazo

%\fontfamily{courier} \selectfont 

\newcommand{\code}[1]{\mbox{\texttt{#1}}} % code font
%\let\code\lstinline
\newcommand{\scode}[1]{\texttt{\small{#1}}} % code font
%\newcommand{\code}[1]{%
%  \lstinline[language=JCop,%
%  					 basicstyle={\ttfamily \footnotesize \selectfont},%
%             keywordstyle={\bfseries \color[rgb]{0,0,0}}]!#1!} % code font
\newcommand{\biglistingsfont}{ \ttfamily \normalsize \selectfont }
\newcommand{\listingsfont}{ \ttfamily \scriptsize \selectfont }
\newcommand{\tinylistingsfont}{\rmfamily \fontsize{6.5}{6.5} \selectfont}
%\newcommand{\listingsfont}{ \fontfamily{courier} \scalebox{.5}[1] \scriptsize \selectfont }
%\newcommand{\tinylistingsfont}{\fontfamily{courier} \fontsize{6.5}{5}  \selectfont \scalebox{.5}[1]}

%\newcommand{\listingsfont}{\ttfamily \scriptsize}

% COLORS
% eclipse keywords: \color[rgb]{0.49,0.00,0.33}
% eclipse strings:  \color[rgb]{0.16,0.2,1}
% eclipse comments: \color[rgb]{0.247,0.498,0.372},


% -- JCop  ----------------------------------------------------------------------------------------

\newcommand{\layerclass}{\code{jcop.lang.Layer}\xspace}
\newcommand{\with}{\code{with}\xspace}
\newcommand{\without}{\code{without}\xspace}
\newcommand{\withoutall}{\code{withoutall}\xspace}
\newcommand{\layer}{\code{layer}\xspace}
\newcommand{\before}{\code{before}\xspace}
\newcommand{\after}{\code{after}\xspace}
\newcommand{\finalmod}{\code{final}\xspace}
\newcommand{\abstractmod}{\code{abstract}\xspace}
\newcommand{\proceed}{\code{proceed}\xspace} 
\newcommand{\ind}{\hspace*{1.5em}}



\lstdefinelanguage{ContextJS} {
      classoffset=0,
      keywords={cop, proceed, withLayer, withoutLayer, with, break,%
                layerClass, layerObject, createLayer, %
        case, catch, continue, else, extends,%
                false, finally, function, for, if, var, instanceof,%
                new, %
                return, super, this, throw,%
                true, try, undefined, while, },
      morecomment=[l]//,%    
      morecomment=[s]{/*}{*/},%
      morestring=[b]",%
      % basicstyle=\listingsfont,
      % basicstyle= \ttfamily \scriptsize \selectfont
      keywordstyle = \color[rgb]{0.54,0.27,0.00},
      %commentstyle = \color[rgb]{0.0,0.533,0.8},
      stringstyle = \color[rgb]{0.0,0.533,0.8},
      identifierstyle=,
      showstringspaces=false,
      captionpos=b,
      tabsize=2,
      aboveskip=5mm,
      %numbersep=-10pt,      
      extendedchars=true,
      frame=tb,
      framexleftmargin=0pt,
      framexrightmargin=-2pt,
      numbers=left,
      numberstyle=\tiny    
}


% -- None ---

\lstdefinelanguage{todo} {
  basicstyle=\color{MidnightBlue}\scriptsize\ttfamily,
  breaklines=true,
  xleftmargin=0.5cm,
  extendedchars=true,
  frame=none
}

\lstdefinelanguage{none} {
  keywords={},
  alsoletter={+},
  morekeywords={}
  frame=none
}

% -----------
  
  % general config
  \lstset{
    framexrightmargin=0pt,      
    numberstyle=\tiny,
    numbersep=5pt,      
    frame=tb,
    basicstyle=\linespread{1} \listingsfont, % print whole listing small
    keywordstyle = \bfseries \color[rgb]{0.49,0.00,0.33},%\color[rgb]{0.84,0.27,0.40},
    commentstyle = \color[rgb]{0.247,0.498,0.372},
    stringstyle = \color[rgb]{0.16,0.2,1},    
    captionpos=b,
    tabsize=2,
    aboveskip=5mm,
    showlines=true,
    showstringspaces=false,
    escapechar=\§,
    breakautoindent=false    
  }
  
  
  \newcommand{\lstnolinenumbers}{
  	\lstset{
  	  numbers=none,   	  
      framexleftmargin=4pt,
      xleftmargin=4pt,   
    }
  }
  	
  
  \newcommand{\lstlinenumbers}{ 
  	\lstset{
  	  numbers=left,
      framexleftmargin=15pt,
      xleftmargin=15pt,            
    }
}



% Paperref ==========================================================================
% includes a published note at the top of the title page for local copies

\usepackage[absolute]{textpos}
%\setlength{\TPHorizModule}{0.8\textwidth}
\definecolor{shadecolor}{rgb}{0.8,0.8,0.8} 

\newcommand{\paperref}[3]{%  
  \setlength{\TPHorizModule}{#1}
  \textblockorigin{0mm}{#2}
  \begin{textblock}{1}(0,0)    
      \begin{shaded}#3\end{shaded}    
  \end{textblock}
}



% \usepackage{verbatim}% http://ctan.org/pkg/verbatim
% \newenvironment{note}{\endgraf\color{MidnightBlue}\scriptsize\verbatim}{\endverbatim}

\newenvironment{rewrite}{\endgraf\color{MidnightBlue}}{}


%\usepackage{fancyvrb, xstring}
%\DefineVerbatimEnvironment{note}{Verbatim}{
%    formatcom=\color{MidnightBlue}\scriptsize
%}

%\renewcommand{\FancyVerbFormatLine}[1]{
%   {\StrSubstitute[1]{#1}{☐}{-}}
%}

% \usepackage{listings}

\lstnewenvironment{note}%
  {%
    %\minipage{\textwidth} 
    \lstset{
      basicstyle=\color{MidnightBlue}\scriptsize\ttfamily,
      breaklines=true,
      xleftmargin=0.5cm,
      extendedchars=true,
      frame=none
    }
  }
  {
    %\endminipage
  }

\usepackage{grfext}
\PrependGraphicsExtensions*{.pdf}


% \usepackage[toc]{glossaries}
% \makeglossaries

\usepackage[nopostdot,nonumberlist,nomain,acronym,xindy%,toc
]{glossaries} % nomain, if you define glossaries in a file, and you use \include{INP-00-glossary}
\makeglossaries
\usepackage[xindy]{imakeidx}
\makeindex

\renewcommand*{\glsnamefont}[1]{\textbf{#1}}



%   \commentmode
%
% - Comments Labeled with Author Id. Define your own author id before \begin{document}:
%   \newcommand\myAuthorID[1]{\nbnote{MAID}{#1}}
%   Use it anywhere in the text: 
%   ... \myAuthorID{this is a comment} ... 
%
% Listings Package Configuration for Several Languages
% ====================================================
% - Syntax highlighting for JCop, ContextJ, LogicAJ, LogicAJ2, GenTL, Prolog
% - Code makro to define code snippets independently of their appeariance. Use:
%   \code{public void foo(){}}
%
% Published Notice on Title Page
% ============================= 
% Paper reference feature to place a publication note on top of the title page of local downloads. 
% Use the following command after \begin{document}: 
% \paperref{<framed box width>}{<height>}{This paper appeared at the n-th XY Conference on Z}
%
% last modified: MA, 2009-08-17_14-47
%%%%%%%%%%%%%%%%%%%%%%%%%%%%%%%%%%%%%%%%%%%%%%%%%%%%%%%%%%%%%%%%%%%%%%%%%%%%%%%%%%%%%%%%%%%%%%%%%%%
\usepackage{amsmath}
\usepackage[usenames,dvipsnames]{color}

\usepackage{framed}
\usepackage{xspace}
\usepackage[hyphens]{url}
%\urlstyle{same}
\urlstyle{tt}
%\usepackage{caption}
\DeclareMathAlphabet{\mathpzc}{OT1}{pzc}{m}{it}
% General %%%%%%%%%%%%%%%%%%%%%%%%%%%%%%%%%%%%%%%%%%%%%%%%%%%%%%%%%%%%%%%%%%%%%%%%%%%%%%%%%%%%%%%%%

% API Table %%%%%%%%%%%%%%%%%%%%%%%%%%%%%%%%%%%%%%%%%%%%%%%%%%%%%%%%%%%%%%%%%%%%%%%%%%%%%%%
\usepackage{longtable}
\usepackage{multirow}

\newenvironment{api2}[1]{%
\begin{tabular}{p{.5cm} p{13.5cm}}%
\hline%
%\multicolumn{2}{|c|}{\code{#1}}\\ \hline %
}%
{\hline%
\end{tabular}%
}
\newcommand{\apidesctxt}[2]{\multicolumn{2}{l}{\lstinline[language=JCop, basicstyle=\listingsfont]{public #1}} \\ & #2 \\}

\newenvironment{api}[1]{%
\begin{longtable}{p{.5cm} p{12.5cm}}%
%\hline%
%\multicolumn{2}{|c|}{\code{#1}}\\ \hline %
}%
{%\hline%
\end{longtable}%
}

\newcommand{\apidesc}[2]{\multicolumn{2}{l}{\lstinline[language=JCop, basicstyle=\listingsfont]{public #1}} \\ & \footnotesize{#2} \\}


% COMMENTS & DRAFT  %%%%%%%%%%%%%%%%%%%%%%%%%%%%%%%%%%%%%%%%%%%%%%%%%%%%%%%%%%%%%%%%%%%%%%%%%%%%%%%

\usepackage{ednotes}
\usepackage{amssymb}
\usepackage{datetime}
\usepackage{everypage}

% date and timeformat
\newdateformat{simpledate}{\THEYEAR-\twodigit{\THEMONTH}-\twodigit{\THEDAY}}
\newtimeformat{simpletime}{\twodigit{\THEHOUR}:\twodigit{\THEMINUTE}}
\newcommand{\timestamp}{\simpledate\today~\simpletime}

% header for draft notice
\newcommand{\draftheader}{%
  %\framebox[\textwidth][t]{%
	  \centerline	
}

% the default definitions of \nbnote and \svnversion are empty	
\newcommand{\nbnote}[2]{}
\newcommand{\svnversion}{}

% the actions perfomed while in comment mode
\newcommand{\commentmode}{
		% \usepackage{draftwatermark} \SetWatermarkLightness{0.70} \SetWatermarkText{\Huge DRAFT}   	
		\renewcommand{\nbnote}[2]{%
		  \fbox{\bfseries\sffamily\scriptsize##1}
		  {\sf\small $\blacktriangleright$ \textit##2 $\blacktriangleleft$}
		  %{\sf\small$\blacktriangleright$\textit{##2}$\blacktriangleleft$}%
		}
	 \renewcommand{\svnversion}{\emph{\footnotesize\svnId}}
	 \AddEverypageHook{\paperref{21cm}{-3mm}{\draftheader}}      
	 \AddEverypageHook{\paperref{21cm}{28.5cm}{\draftheader}}      
}


\newcommand\todo[1]{\nbnote{TO DO}{#1}}

\newcommand{\here}{\nbnote{***}{CONTINUE HERE}}
\newcommand\fix[1]{\nbnote{FIX}{#1}}
\newcommand\jl[1]{\nbnote{JL}{#1}}

\newcommand\rw[1]{\color{blue}#1\color{black}}

% dangerous!
% \DeclareOption{comments}{\commentmode}
% \ProcessOptions 

% LISTINGS  %%%%%%%%%%%%%%%%%%%%%%%%%%%%%%%%%%%%%%%%%%%%%%%%%%%%%%%%%%%%%%%%%%%%%%%%%%%%%%%%%%%%%%%

\usepackage[formats]{listings}
\usepackage[T1]{fontenc}


%\usepackage{pifont}
%\usepackage{winfonts}
%\usepackage{windingbats}
%\usepackage{winpifont}
\usepackage{soul}
%\DisableLigatures{encoding = T1, family = courier }
%\usepackage{beramono}
\usepackage[scaled=.84]{DejaVuSansMono}
%\linespread{1.5}  % mathpazo

%\fontfamily{courier} \selectfont 

\newcommand{\code}[1]{\mbox{\texttt{#1}}} % code font
%\let\code\lstinline
\newcommand{\scode}[1]{\texttt{\small{#1}}} % code font
%\newcommand{\code}[1]{%
%  \lstinline[language=JCop,%
%  					 basicstyle={\ttfamily \footnotesize \selectfont},%
%             keywordstyle={\bfseries \color[rgb]{0,0,0}}]!#1!} % code font
\newcommand{\biglistingsfont}{ \ttfamily \normalsize \selectfont }
\newcommand{\listingsfont}{ \ttfamily \scriptsize \selectfont }
\newcommand{\tinylistingsfont}{\rmfamily \fontsize{6.5}{6.5} \selectfont}
%\newcommand{\listingsfont}{ \fontfamily{courier} \scalebox{.5}[1] \scriptsize \selectfont }
%\newcommand{\tinylistingsfont}{\fontfamily{courier} \fontsize{6.5}{5}  \selectfont \scalebox{.5}[1]}

%\newcommand{\listingsfont}{\ttfamily \scriptsize}

% COLORS
% eclipse keywords: \color[rgb]{0.49,0.00,0.33}
% eclipse strings:  \color[rgb]{0.16,0.2,1}
% eclipse comments: \color[rgb]{0.247,0.498,0.372},


% -- JCop  ----------------------------------------------------------------------------------------

\newcommand{\layerclass}{\code{jcop.lang.Layer}\xspace}
\newcommand{\with}{\code{with}\xspace}
\newcommand{\without}{\code{without}\xspace}
\newcommand{\withoutall}{\code{withoutall}\xspace}
\newcommand{\layer}{\code{layer}\xspace}
\newcommand{\before}{\code{before}\xspace}
\newcommand{\after}{\code{after}\xspace}
\newcommand{\finalmod}{\code{final}\xspace}
\newcommand{\abstractmod}{\code{abstract}\xspace}
\newcommand{\proceed}{\code{proceed}\xspace} 
\newcommand{\ind}{\hspace*{1.5em}}



\lstdefinelanguage{ContextJS} {
      classoffset=0,
      keywords={cop, proceed, withLayer, withoutLayer, with, break,%
                layerClass, layerObject, createLayer, %
        case, catch, continue, else, extends,%
                false, finally, function, for, if, var, instanceof,%
                new, %
                return, super, this, throw,%
                true, try, undefined, while, },
      morecomment=[l]//,%    
      morecomment=[s]{/*}{*/},%
      morestring=[b]",%
      % basicstyle=\listingsfont,
      % basicstyle= \ttfamily \scriptsize \selectfont
      keywordstyle = \color[rgb]{0.54,0.27,0.00},
      %commentstyle = \color[rgb]{0.0,0.533,0.8},
      stringstyle = \color[rgb]{0.0,0.533,0.8},
      identifierstyle=,
      showstringspaces=false,
      captionpos=b,
      tabsize=2,
      aboveskip=5mm,
      %numbersep=-10pt,      
      extendedchars=true,
      frame=tb,
      framexleftmargin=0pt,
      framexrightmargin=-2pt,
      numbers=left,
      numberstyle=\tiny    
}


% -- None ---

\lstdefinelanguage{todo} {
  basicstyle=\color{MidnightBlue}\scriptsize\ttfamily,
  breaklines=true,
  xleftmargin=0.5cm,
  extendedchars=true,
  frame=none
}

\lstdefinelanguage{none} {
  keywords={},
  alsoletter={+},
  morekeywords={}
  frame=none
}

% -----------
  
  % general config
  \lstset{
    framexrightmargin=0pt,      
    numberstyle=\tiny,
    numbersep=5pt,      
    frame=tb,
    basicstyle=\linespread{1} \listingsfont, % print whole listing small
    keywordstyle = \bfseries \color[rgb]{0.49,0.00,0.33},%\color[rgb]{0.84,0.27,0.40},
    commentstyle = \color[rgb]{0.247,0.498,0.372},
    stringstyle = \color[rgb]{0.16,0.2,1},    
    captionpos=b,
    tabsize=2,
    aboveskip=5mm,
    showlines=true,
    showstringspaces=false,
    escapechar=\§,
    breakautoindent=false    
  }
  
  
  \newcommand{\lstnolinenumbers}{
  	\lstset{
  	  numbers=none,   	  
      framexleftmargin=4pt,
      xleftmargin=4pt,   
    }
  }
  	
  
  \newcommand{\lstlinenumbers}{ 
  	\lstset{
  	  numbers=left,
      framexleftmargin=15pt,
      xleftmargin=15pt,            
    }
}



% Paperref ==========================================================================
% includes a published note at the top of the title page for local copies

\usepackage[absolute]{textpos}
%\setlength{\TPHorizModule}{0.8\textwidth}
\definecolor{shadecolor}{rgb}{0.8,0.8,0.8} 

\newcommand{\paperref}[3]{%  
  \setlength{\TPHorizModule}{#1}
  \textblockorigin{0mm}{#2}
  \begin{textblock}{1}(0,0)    
      \begin{shaded}#3\end{shaded}    
  \end{textblock}
}



% \usepackage{verbatim}% http://ctan.org/pkg/verbatim
% \newenvironment{note}{\endgraf\color{MidnightBlue}\scriptsize\verbatim}{\endverbatim}

\newenvironment{rewrite}{\endgraf\color{MidnightBlue}}{}


%\usepackage{fancyvrb, xstring}
%\DefineVerbatimEnvironment{note}{Verbatim}{
%    formatcom=\color{MidnightBlue}\scriptsize
%}

%\renewcommand{\FancyVerbFormatLine}[1]{
%   {\StrSubstitute[1]{#1}{☐}{-}}
%}

% \usepackage{listings}

\lstnewenvironment{note}%
  {%
    %\minipage{\textwidth} 
    \lstset{
      basicstyle=\color{MidnightBlue}\scriptsize\ttfamily,
      breaklines=true,
      xleftmargin=0.5cm,
      extendedchars=true,
      frame=none
    }
  }
  {
    %\endminipage
  }

\usepackage{grfext}
\PrependGraphicsExtensions*{.pdf}


% \usepackage[toc]{glossaries}
% \makeglossaries

\usepackage[nopostdot,nonumberlist,nomain,acronym,xindy%,toc
]{glossaries} % nomain, if you define glossaries in a file, and you use \include{INP-00-glossary}
\makeglossaries
\usepackage[xindy]{imakeidx}
\makeindex

\renewcommand*{\glsnamefont}[1]{\textbf{#1}}



%   \commentmode
%
% - Comments Labeled with Author Id. Define your own author id before \begin{document}:
%   \newcommand\myAuthorID[1]{\nbnote{MAID}{#1}}
%   Use it anywhere in the text: 
%   ... \myAuthorID{this is a comment} ... 
%
% Listings Package Configuration for Several Languages
% ====================================================
% - Syntax highlighting for JCop, ContextJ, LogicAJ, LogicAJ2, GenTL, Prolog
% - Code makro to define code snippets independently of their appeariance. Use:
%   \code{public void foo(){}}
%
% Published Notice on Title Page
% ============================= 
% Paper reference feature to place a publication note on top of the title page of local downloads. 
% Use the following command after \begin{document}: 
% \paperref{<framed box width>}{<height>}{This paper appeared at the n-th XY Conference on Z}
%
% last modified: MA, 2009-08-17_14-47
%%%%%%%%%%%%%%%%%%%%%%%%%%%%%%%%%%%%%%%%%%%%%%%%%%%%%%%%%%%%%%%%%%%%%%%%%%%%%%%%%%%%%%%%%%%%%%%%%%%
\usepackage{amsmath}
\usepackage[usenames,dvipsnames]{color}

\usepackage{framed}
\usepackage{xspace}
\usepackage[hyphens]{url}
%\urlstyle{same}
\urlstyle{tt}
%\usepackage{caption}
\DeclareMathAlphabet{\mathpzc}{OT1}{pzc}{m}{it}
% General %%%%%%%%%%%%%%%%%%%%%%%%%%%%%%%%%%%%%%%%%%%%%%%%%%%%%%%%%%%%%%%%%%%%%%%%%%%%%%%%%%%%%%%%%

% API Table %%%%%%%%%%%%%%%%%%%%%%%%%%%%%%%%%%%%%%%%%%%%%%%%%%%%%%%%%%%%%%%%%%%%%%%%%%%%%%%
\usepackage{longtable}
\usepackage{multirow}

\newenvironment{api2}[1]{%
\begin{tabular}{p{.5cm} p{13.5cm}}%
\hline%
%\multicolumn{2}{|c|}{\code{#1}}\\ \hline %
}%
{\hline%
\end{tabular}%
}
\newcommand{\apidesctxt}[2]{\multicolumn{2}{l}{\lstinline[language=JCop, basicstyle=\listingsfont]{public #1}} \\ & #2 \\}

\newenvironment{api}[1]{%
\begin{longtable}{p{.5cm} p{12.5cm}}%
%\hline%
%\multicolumn{2}{|c|}{\code{#1}}\\ \hline %
}%
{%\hline%
\end{longtable}%
}

\newcommand{\apidesc}[2]{\multicolumn{2}{l}{\lstinline[language=JCop, basicstyle=\listingsfont]{public #1}} \\ & \footnotesize{#2} \\}


% COMMENTS & DRAFT  %%%%%%%%%%%%%%%%%%%%%%%%%%%%%%%%%%%%%%%%%%%%%%%%%%%%%%%%%%%%%%%%%%%%%%%%%%%%%%%

\usepackage{ednotes}
\usepackage{amssymb}
\usepackage{datetime}
\usepackage{everypage}

% date and timeformat
\newdateformat{simpledate}{\THEYEAR-\twodigit{\THEMONTH}-\twodigit{\THEDAY}}
\newtimeformat{simpletime}{\twodigit{\THEHOUR}:\twodigit{\THEMINUTE}}
\newcommand{\timestamp}{\simpledate\today~\simpletime}

% header for draft notice
\newcommand{\draftheader}{%
  %\framebox[\textwidth][t]{%
	  \centerline	
}

% the default definitions of \nbnote and \svnversion are empty	
\newcommand{\nbnote}[2]{}
\newcommand{\svnversion}{}

% the actions perfomed while in comment mode
\newcommand{\commentmode}{
		% \usepackage{draftwatermark} \SetWatermarkLightness{0.70} \SetWatermarkText{\Huge DRAFT}   	
		\renewcommand{\nbnote}[2]{%
		  \fbox{\bfseries\sffamily\scriptsize##1}
		  {\sf\small $\blacktriangleright$ \textit##2 $\blacktriangleleft$}
		  %{\sf\small$\blacktriangleright$\textit{##2}$\blacktriangleleft$}%
		}
	 \renewcommand{\svnversion}{\emph{\footnotesize\svnId}}
	 \AddEverypageHook{\paperref{21cm}{-3mm}{\draftheader}}      
	 \AddEverypageHook{\paperref{21cm}{28.5cm}{\draftheader}}      
}


\newcommand\todo[1]{\nbnote{TO DO}{#1}}

\newcommand{\here}{\nbnote{***}{CONTINUE HERE}}
\newcommand\fix[1]{\nbnote{FIX}{#1}}
\newcommand\jl[1]{\nbnote{JL}{#1}}

\newcommand\rw[1]{\color{blue}#1\color{black}}

% dangerous!
% \DeclareOption{comments}{\commentmode}
% \ProcessOptions 

% LISTINGS  %%%%%%%%%%%%%%%%%%%%%%%%%%%%%%%%%%%%%%%%%%%%%%%%%%%%%%%%%%%%%%%%%%%%%%%%%%%%%%%%%%%%%%%

\usepackage[formats]{listings}
\usepackage[T1]{fontenc}


%\usepackage{pifont}
%\usepackage{winfonts}
%\usepackage{windingbats}
%\usepackage{winpifont}
\usepackage{soul}
%\DisableLigatures{encoding = T1, family = courier }
%\usepackage{beramono}
\usepackage[scaled=.84]{DejaVuSansMono}
%\linespread{1.5}  % mathpazo

%\fontfamily{courier} \selectfont 

\newcommand{\code}[1]{\mbox{\texttt{#1}}} % code font
%\let\code\lstinline
\newcommand{\scode}[1]{\texttt{\small{#1}}} % code font
%\newcommand{\code}[1]{%
%  \lstinline[language=JCop,%
%  					 basicstyle={\ttfamily \footnotesize \selectfont},%
%             keywordstyle={\bfseries \color[rgb]{0,0,0}}]!#1!} % code font
\newcommand{\biglistingsfont}{ \ttfamily \normalsize \selectfont }
\newcommand{\listingsfont}{ \ttfamily \scriptsize \selectfont }
\newcommand{\tinylistingsfont}{\rmfamily \fontsize{6.5}{6.5} \selectfont}
%\newcommand{\listingsfont}{ \fontfamily{courier} \scalebox{.5}[1] \scriptsize \selectfont }
%\newcommand{\tinylistingsfont}{\fontfamily{courier} \fontsize{6.5}{5}  \selectfont \scalebox{.5}[1]}

%\newcommand{\listingsfont}{\ttfamily \scriptsize}

% COLORS
% eclipse keywords: \color[rgb]{0.49,0.00,0.33}
% eclipse strings:  \color[rgb]{0.16,0.2,1}
% eclipse comments: \color[rgb]{0.247,0.498,0.372},


% -- JCop  ----------------------------------------------------------------------------------------

\newcommand{\layerclass}{\code{jcop.lang.Layer}\xspace}
\newcommand{\with}{\code{with}\xspace}
\newcommand{\without}{\code{without}\xspace}
\newcommand{\withoutall}{\code{withoutall}\xspace}
\newcommand{\layer}{\code{layer}\xspace}
\newcommand{\before}{\code{before}\xspace}
\newcommand{\after}{\code{after}\xspace}
\newcommand{\finalmod}{\code{final}\xspace}
\newcommand{\abstractmod}{\code{abstract}\xspace}
\newcommand{\proceed}{\code{proceed}\xspace} 
\newcommand{\ind}{\hspace*{1.5em}}



\lstdefinelanguage{ContextJS} {
      classoffset=0,
      keywords={cop, proceed, withLayer, withoutLayer, with, break,%
                layerClass, layerObject, createLayer, %
        case, catch, continue, else, extends,%
                false, finally, function, for, if, var, instanceof,%
                new, %
                return, super, this, throw,%
                true, try, undefined, while, },
      morecomment=[l]//,%    
      morecomment=[s]{/*}{*/},%
      morestring=[b]",%
      % basicstyle=\listingsfont,
      % basicstyle= \ttfamily \scriptsize \selectfont
      keywordstyle = \color[rgb]{0.54,0.27,0.00},
      %commentstyle = \color[rgb]{0.0,0.533,0.8},
      stringstyle = \color[rgb]{0.0,0.533,0.8},
      identifierstyle=,
      showstringspaces=false,
      captionpos=b,
      tabsize=2,
      aboveskip=5mm,
      %numbersep=-10pt,      
      extendedchars=true,
      frame=tb,
      framexleftmargin=0pt,
      framexrightmargin=-2pt,
      numbers=left,
      numberstyle=\tiny    
}


% -- None ---

\lstdefinelanguage{todo} {
  basicstyle=\color{MidnightBlue}\scriptsize\ttfamily,
  breaklines=true,
  xleftmargin=0.5cm,
  extendedchars=true,
  frame=none
}

\lstdefinelanguage{none} {
  keywords={},
  alsoletter={+},
  morekeywords={}
  frame=none
}

% -----------
  
  % general config
  \lstset{
    framexrightmargin=0pt,      
    numberstyle=\tiny,
    numbersep=5pt,      
    frame=tb,
    basicstyle=\linespread{1} \listingsfont, % print whole listing small
    keywordstyle = \bfseries \color[rgb]{0.49,0.00,0.33},%\color[rgb]{0.84,0.27,0.40},
    commentstyle = \color[rgb]{0.247,0.498,0.372},
    stringstyle = \color[rgb]{0.16,0.2,1},    
    captionpos=b,
    tabsize=2,
    aboveskip=5mm,
    showlines=true,
    showstringspaces=false,
    escapechar=\§,
    breakautoindent=false    
  }
  
  
  \newcommand{\lstnolinenumbers}{
  	\lstset{
  	  numbers=none,   	  
      framexleftmargin=4pt,
      xleftmargin=4pt,   
    }
  }
  	
  
  \newcommand{\lstlinenumbers}{ 
  	\lstset{
  	  numbers=left,
      framexleftmargin=15pt,
      xleftmargin=15pt,            
    }
}



% Paperref ==========================================================================
% includes a published note at the top of the title page for local copies

\usepackage[absolute]{textpos}
%\setlength{\TPHorizModule}{0.8\textwidth}
\definecolor{shadecolor}{rgb}{0.8,0.8,0.8} 

\newcommand{\paperref}[3]{%  
  \setlength{\TPHorizModule}{#1}
  \textblockorigin{0mm}{#2}
  \begin{textblock}{1}(0,0)    
      \begin{shaded}#3\end{shaded}    
  \end{textblock}
}



% \usepackage{verbatim}% http://ctan.org/pkg/verbatim
% \newenvironment{note}{\endgraf\color{MidnightBlue}\scriptsize\verbatim}{\endverbatim}

\newenvironment{rewrite}{\endgraf\color{MidnightBlue}}{}


%\usepackage{fancyvrb, xstring}
%\DefineVerbatimEnvironment{note}{Verbatim}{
%    formatcom=\color{MidnightBlue}\scriptsize
%}

%\renewcommand{\FancyVerbFormatLine}[1]{
%   {\StrSubstitute[1]{#1}{☐}{-}}
%}

% \usepackage{listings}

\lstnewenvironment{note}%
  {%
    %\minipage{\textwidth} 
    \lstset{
      basicstyle=\color{MidnightBlue}\scriptsize\ttfamily,
      breaklines=true,
      xleftmargin=0.5cm,
      extendedchars=true,
      frame=none
    }
  }
  {
    %\endminipage
  }

\usepackage{grfext}
\PrependGraphicsExtensions*{.pdf}


% \usepackage[toc]{glossaries}
% \makeglossaries

\usepackage[nopostdot,nonumberlist,nomain,acronym,xindy%,toc
]{glossaries} % nomain, if you define glossaries in a file, and you use \include{INP-00-glossary}
\makeglossaries
\usepackage[xindy]{imakeidx}
\makeindex

\renewcommand*{\glsnamefont}[1]{\textbf{#1}}


